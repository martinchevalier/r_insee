\documentclass[12pt,ignorenonframetext,]{beamer}
\setbeamertemplate{caption}[numbered]
\setbeamertemplate{caption label separator}{: }
\setbeamercolor{caption name}{fg=normal text.fg}
\beamertemplatenavigationsymbolsempty
\usepackage{lmodern}
\usepackage{amssymb,amsmath}
\usepackage{ifxetex,ifluatex}
\usepackage{fixltx2e} % provides \textsubscript
\ifnum 0\ifxetex 1\fi\ifluatex 1\fi=0 % if pdftex
\usepackage[T1]{fontenc}
\usepackage[utf8]{inputenc}
\else % if luatex or xelatex
\ifxetex
\usepackage{mathspec}
\else
\usepackage{fontspec}
\fi
\defaultfontfeatures{Ligatures=TeX,Scale=MatchLowercase}
\fi
% use upquote if available, for straight quotes in verbatim environments
\IfFileExists{upquote.sty}{\usepackage{upquote}}{}
% use microtype if available
\IfFileExists{microtype.sty}{%
\usepackage{microtype}
\UseMicrotypeSet[protrusion]{basicmath} % disable protrusion for tt fonts
}{}
\newif\ifbibliography
\usepackage{color}
\usepackage{fancyvrb}
\newcommand{\VerbBar}{|}
\newcommand{\VERB}{\Verb[commandchars=\\\{\}]}
\DefineVerbatimEnvironment{Highlighting}{Verbatim}{commandchars=\\\{\}}
% Add ',fontsize=\small' for more characters per line
\newenvironment{Shaded}{}{}
\newcommand{\KeywordTok}[1]{\textcolor[rgb]{0.00,0.00,1.00}{{#1}}}
\newcommand{\DataTypeTok}[1]{{#1}}
\newcommand{\DecValTok}[1]{{#1}}
\newcommand{\BaseNTok}[1]{{#1}}
\newcommand{\FloatTok}[1]{{#1}}
\newcommand{\ConstantTok}[1]{{#1}}
\newcommand{\CharTok}[1]{\textcolor[rgb]{0.00,0.50,0.50}{{#1}}}
\newcommand{\SpecialCharTok}[1]{\textcolor[rgb]{0.00,0.50,0.50}{{#1}}}
\newcommand{\StringTok}[1]{\textcolor[rgb]{0.00,0.50,0.50}{{#1}}}
\newcommand{\VerbatimStringTok}[1]{\textcolor[rgb]{0.00,0.50,0.50}{{#1}}}
\newcommand{\SpecialStringTok}[1]{\textcolor[rgb]{0.00,0.50,0.50}{{#1}}}
\newcommand{\ImportTok}[1]{{#1}}
\newcommand{\CommentTok}[1]{\textcolor[rgb]{0.00,0.50,0.00}{{#1}}}
\newcommand{\DocumentationTok}[1]{\textcolor[rgb]{0.00,0.50,0.00}{{#1}}}
\newcommand{\AnnotationTok}[1]{\textcolor[rgb]{0.00,0.50,0.00}{{#1}}}
\newcommand{\CommentVarTok}[1]{\textcolor[rgb]{0.00,0.50,0.00}{{#1}}}
\newcommand{\OtherTok}[1]{\textcolor[rgb]{1.00,0.25,0.00}{{#1}}}
\newcommand{\FunctionTok}[1]{{#1}}
\newcommand{\VariableTok}[1]{{#1}}
\newcommand{\ControlFlowTok}[1]{\textcolor[rgb]{0.00,0.00,1.00}{{#1}}}
\newcommand{\OperatorTok}[1]{{#1}}
\newcommand{\BuiltInTok}[1]{{#1}}
\newcommand{\ExtensionTok}[1]{{#1}}
\newcommand{\PreprocessorTok}[1]{\textcolor[rgb]{1.00,0.25,0.00}{{#1}}}
\newcommand{\AttributeTok}[1]{{#1}}
\newcommand{\RegionMarkerTok}[1]{{#1}}
\newcommand{\InformationTok}[1]{\textcolor[rgb]{0.00,0.50,0.00}{{#1}}}
\newcommand{\WarningTok}[1]{\textcolor[rgb]{0.00,0.50,0.00}{\textbf{{#1}}}}
\newcommand{\AlertTok}[1]{\textcolor[rgb]{1.00,0.00,0.00}{{#1}}}
\newcommand{\ErrorTok}[1]{\textcolor[rgb]{1.00,0.00,0.00}{\textbf{{#1}}}}
\newcommand{\NormalTok}[1]{{#1}}

% Prevent slide breaks in the middle of a paragraph:
\widowpenalties 1 10000
\raggedbottom

\AtBeginPart{
\let\insertpartnumber\relax
\let\partname\relax
\frame{\partpage}
}
\AtBeginSection{
\ifbibliography
\else
\let\insertsectionnumber\relax
\let\sectionname\relax
\frame{\sectionpage}
\fi
}
\AtBeginSubsection{
\let\insertsubsectionnumber\relax
\let\subsectionname\relax
\frame{\subsectionpage}
}

\setlength{\parindent}{0pt}
\setlength{\parskip}{6pt plus 2pt minus 1pt}
\setlength{\emergencystretch}{3em}  % prevent overfull lines
\providecommand{\tightlist}{%
\setlength{\itemsep}{0pt}\setlength{\parskip}{0pt}}
\setcounter{secnumdepth}{0}

\usepackage[french]{babel}
\usepackage{lmodern}
\usepackage{graphicx}
\usepackage{xcolor}
\usepackage{textcomp} 
\usepackage{amsmath, amsfonts, amssymb, amsthm}
\usepackage{booktabs,multirow}
\usepackage{setspace}
\usepackage{float}
\usepackage{pgfpages}
\usepackage{colortbl}
\usepackage{epstopdf}
\usepackage{framed}

\definecolor{shadecolor}{RGB}{248,248,248}
\definecolor{grayInsee}{HTML}{5a5758}
\definecolor{redInsee}{HTML}{ed1443}

%Everything about the notes and formatting of the notepage
\setbeamertemplate{note page}{%
	Notes personnelles
	\insertnote%
}


\setbeamertemplate{navigation symbols}{}
\usetheme{default} %Malmoe not bad
\setbeamertemplate{footline}[frame number]


%\setbeamerfont{frametitle}{size=\normalsize}
%\setbeamerfont{framesubtitle}{size=\Large}
%\setbeamercolor{frametitle}{fg=grayInsee}
%\setbeamercolor{framesubtitle}{fg=redInsee}
\setbeamercolor{title}{fg=grayInsee}
\setbeamercolor{subsection in toc}{fg=grayInsee}
\setbeamertemplate{frametitle}{%
	\large \textcolor{grayInsee}{\subsecname}
	\\ \vspace{0.1cm} \Large \textcolor{redInsee}{\insertframetitle}
}
%\setbeamertemplate{frametitle}{%
%	\large \textcolor{grayInsee}{
%		\ifx\intertsubsection\emptyset
%			\secname \\ \vspace{0.1cm} 
%		\else 
%			\subsecname \\ \vspace{0.1cm}
%		\fi
%	}
%	\Large \textcolor{redInsee}{\insertframetitle}
%}
\setbeamercolor{local structure}{fg=redInsee}

\AtBeginSection[]
{\ifnum \thesection>1
  \begin{frame}
  \vfill
  \begin{center}
  \LARGE
  \textcolor{grayInsee}{\insertsectionhead}
  \end{center}
  \vfill
  \end{frame}
\else
\fi
}

\AtBeginSubsection[]{}

\title{\Large Formation \textbf{R} Perfectionnement}

\institute{ \includegraphics[height = 2.5cm]{../figures/Logo_Insee.png}\\ ~ \\ \normalsize Martin \textsc{Chevalier} (Insee)}

\author{21-22 juin 2017}

\date{}

\renewenvironment{Shaded}{\begin{snugshade}}{\end{snugshade}}

\newcommand{\aparte}[2]{
	{\small\textsf{\textbf{#1} #2}}
}

%\usepackage{enumitem}
%\setlist{nolistsep}

\usepackage{tikz}
\usetikzlibrary{shapes,arrows,calc, positioning}
\tikzstyle{input} = [draw, rectangle,rounded corners, text width=2.5cm, fill=green!20, node distance=0.5cm, minimum height=2em, text centered]
\tikzstyle{output} = [draw, ellipse,fill=red!20, node distance=0.5cm, minimum height=2em, text centered]
\tikzstyle{block} = [rectangle, draw, fill=blue!20, 
    text width=1.5cm, text centered, minimum height=2em, node distance = 0.5cm]
\tikzstyle{line} = [draw, -latex', shorten >=2pt, shorten <=2pt]
\tikzset{
  invisible/.style={opacity=0},
  visible on/.style={alt={#1{}{invisible}}},
  alt/.code args={<#1>#2#3}{%
    \alt<#1>{\pgfkeysalso{#2}}{\pgfkeysalso{#3}} % \pgfkeysalso doesn't change the path
  },
}

%\usepackage{pgfpages}
%\mode<handout>{
%	%\setbeamercolor{background canvas}{bg=black!20}
%	\pgfpagesuselayout{2 on 1}[border shrink=2mm]
%}

\title{Formation R Perfectionnement}
\date{}

\begin{document}
\frame{\titlepage}

\section{Introduction :
Se~perfectionner~avec~R}\label{introduction-seperfectionneravecr}

\subsection*{Introduction : Se\ perfectionner\ avec\ R}

\begin{frame}{Connaître plus ou connaître mieux ?}

Comme tout langage statistique ou de programmation, R repose sur un
ensemble d'instructions plus ou moins complexes.

\pause \bigskip
Se perfectionner dans la maîtrise de R peut donc signifier deux choses :

\begin{itemize}
\tightlist
\item
  étendre son \og vocabulaire \fg{} d'instructions connues ;
\item
  mieux comprendre les instructions déjà connues.
\end{itemize}

\pause \bigskip
En pratique, les deux \textbf{vont de pair} : en découvrant de nouvelles
fonctions, on est souvent amené à mieux comprendre le fonctionnement de
celles que l'on croyait maîtriser.

\end{frame}

\begin{frame}{Plan de la partie}

\Large 
\tableofcontents[currentsection, sectionstyle = hide, subsectionstyle = show/show/hide]

\end{frame}

\subsection{Chercher (et trouver !) de
l'aide}\label{chercher-et-trouver-de-laide}

\begin{frame}[fragile]{Savoir utiliser l'aide du logiciel}

À tout moment, taper \texttt{help(nomFonction)} ou
\texttt{?\ nomFonction} affiche l'aide de la fonction
\texttt{nomFonction}.

\begin{Shaded}
\begin{Highlighting}[]
\CommentTok{# Aide de la fonction read.csv}
\NormalTok{? read.csv}
\end{Highlighting}
\end{Shaded}

\pause 

\textbf{Remarque} Pour afficher l'aide sur une fonction d'un
\emph{package}, il faut que celui-ci soit au préalable chargé (avec
\texttt{library()} ou \texttt{require()}).

\pause La fonction \texttt{help.search()} ou la commande \texttt{??}
permettent d'effectuer une recherche approximative:

\begin{Shaded}
\begin{Highlighting}[]
\CommentTok{# Recherche à partir du mot-clé csv}
\NormalTok{?? csv}
\end{Highlighting}
\end{Shaded}

\end{frame}

\begin{frame}{Chercher de l'aide en ligne}

Bien souvent, le problème que l'on rencontre a \textbf{déjà été
rencontré par d'autres}.

\bigskip
Pour progresser dans la maîtrise de R, il ne faut donc surtout pas
hésiter à s'appuyer sur les forums de discussion, comme par exemple
\href{http://stackoverflow.com/questions/tagged/r}{\underline{stackoverflow}}.

\pause \bigskip
On gagne ainsi souvent beaucoup de temps en formulant le problème que
l'on rencontre dans un \textbf{moteur de recherche} pour consulter
certaines réponses.

\bigskip  Quand une question semble ne pas avoir été déjà posée, ne pas
hésiter à la poser soi-même, en joignant alors un \textbf{exemple
reproductible} (\emph{minimal working example} ou MWE).

\end{frame}

\begin{frame}[fragile]{Afficher le code d'une fonction}

Quand l'utilisation d'une fonction pose problème (message d'erreur
inattendu), il est souvent utile d'\textbf{afficher son code} pour
comprendre d'où vient le problème.

\pause Pour ce faire, il suffit de saisir son nom sans parenthèses.

\footnotesize

\begin{Shaded}
\begin{Highlighting}[]
\CommentTok{# Code de la fonction read.csv}
\NormalTok{read.csv}
  \NormalTok{## function (file, header = TRUE, sep = ",", quote = "\textbackslash{}"", dec = ".", }
  \NormalTok{##     fill = TRUE, comment.char = "", ...) }
  \NormalTok{## read.table(file = file, header = header, sep = sep, quote = quote, }
  \NormalTok{##     dec = dec, fill = fill, comment.char = comment.char, ...)}
  \NormalTok{## <bytecode: 0x37ccb28>}
  \NormalTok{## <environment: namespace:utils>}
\end{Highlighting}
\end{Shaded}

\pause \normalsize
Afficher le code d'une fonction est dans certains cas plus difficile,
\emph{cf.}
\href{http://stackoverflow.com/questions/19226816/how-can-i-view-the-source-code-for-a-function}{\underline{stackoverflow}}.

\end{frame}

\subsection{Découvrir de nouvelles
fonctionnalités}\label{decouvrir-de-nouvelles-fonctionnalites}

\begin{frame}{Se repérer dans les CRAN \protect\textit{Task Views}}

Les CRAN \emph{Task Views} recensent les fonctions et \emph{packages} de
façon thématique. Elles sont mises à jour régulièrement et portent sur
des thèmes variés:

\pause

\footnotesize \href{https://cran.r-project.org/web/views/Bayesian.html}{Bayesian},
\href{https://cran.r-project.org/web/views/ChemPhys.html}{ChemPhys},
\href{https://cran.r-project.org/web/views/ClinicalTrials.html}{ClinicalTrials},
\href{https://cran.r-project.org/web/views/Cluster.html}{Cluster},
\href{https://cran.r-project.org/web/views/DifferentialEquations.html}{DifferentialEquations},
\href{https://cran.r-project.org/web/views/Distributions.html}{Distributions},
\href{https://cran.r-project.org/web/views/Econometrics.html}{Econometrics},
\href{https://cran.r-project.org/web/views/Environmetrics.html}{Environmetrics},
\href{https://cran.r-project.org/web/views/ExperimentalDesign.html}{ExperimentalDesign},
\href{https://cran.r-project.org/web/views/ExtremeValue.html}{ExtremeValue},
\href{https://cran.r-project.org/web/views/Finance.html}{Finance},
\href{https://cran.r-project.org/web/views/FunctionalData.html}{FunctionalData},
\href{https://cran.r-project.org/web/views/Genetics.html}{Genetics},
\href{https://cran.r-project.org/web/views/Graphics.html}{Graphics},
\href{https://cran.r-project.org/web/views/HighPerformanceComputing.html}{HighPerformanceComputing},
\href{https://cran.r-project.org/web/views/MachineLearning.html}{MachineLearning},
\href{https://cran.r-project.org/web/views/MedicalImaging.html}{MedicalImaging},
\href{https://cran.r-project.org/web/views/MetaAnalysis.html}{MetaAnalysis},
\href{https://cran.r-project.org/web/views/Multivariate.html}{Multivariate},
\href{https://cran.r-project.org/web/views/NaturalLanguageProcessing.html}{NaturalLanguageProcessing},
\href{https://cran.r-project.org/web/views/NumericalMathematics.html}{NumericalMathematics},
\href{https://cran.r-project.org/web/views/OfficialStatistics.html}{OfficialStatistics},
\href{https://cran.r-project.org/web/views/Optimization.html}{Optimization},
\href{https://cran.r-project.org/web/views/Pharmacokinetics.html}{Pharmacokinetics},
\href{https://cran.r-project.org/web/views/Phylogenetics.html}{Phylogenetics},
\href{https://cran.r-project.org/web/views/Psychometrics.html}{Psychometrics},
\href{https://cran.r-project.org/web/views/ReproducibleResearch.html}{ReproducibleResearch},
\href{https://cran.r-project.org/web/views/Robust.html}{Robust},
\href{https://cran.r-project.org/web/views/SocialSciences.html}{SocialSciences},
\href{https://cran.r-project.org/web/views/Spatial.html}{Spatial},
\href{https://cran.r-project.org/web/views/SpatioTemporal.html}{SpatioTemporal},
\href{https://cran.r-project.org/web/views/Survival.html}{Survival},
\href{https://cran.r-project.org/web/views/TimeSeries.html}{TimeSeries},
\href{https://cran.r-project.org/web/views/WebTechnologies.html}{WebTechnologies},
\href{https://cran.r-project.org/web/views/gR.html}{gR}

\pause \bigskip \normalsize
La liste de toutes les \emph{Task Views} est accessible à la page :
\href{https://cran.r-project.org/web/views}{\underline{https://cran.r-project.org/web/views}}.

\end{frame}

\begin{frame}{Consulter des sites, tutoriels, livres}

De plus en plus de supports sont consacrés à la présentation et à
l'enseignement des fonctionnalités de R, comme par exemple :

\begin{itemize}
\item
  \pause le site
  \href{https://www.r-bloggers.com}{\underline{R-bloggers}}: articles en
  général courts sur des exemples d'applications (de qualité inégale);
\item
  \pause le site \href{https://bookdown.org}{\underline{bookdown.org}}:
  dépôt de livres numériques consacrés à R élaborés avec R Markdown
  (très riches et très complets);
\item
  \pause le site de \href{https://www.rstudio.com}{\underline{RStudio}}:
  nombreux
  \href{https://www.rstudio.com/resources/cheatsheets/}{\underline{aides-mémoires}}
  ou articles présentant les fonctionnalités de l'écosystème RStudio;
\item
  \pause les ouvrages de
  \href{http://hadley.nz}{\underline{Hadley Wickham}}:
  \href{https://github.com/hadley/ggplot2-book}{\underline{ggplot2: elegant graphics for data analysis}}
  (à compiler soi-même),
  \href{http:/:adv-r.had.co.nz}{\underline{Advanced R}}.
\end{itemize}

\end{frame}

\subsection{\texorpdfstring{Utiliser de nouveaux
\protect\textit{packages}}{Utiliser de nouveaux }}\label{utiliser-de-nouveaux}

\begin{frame}{Accéder à la documentation d'un \emph{package}}

Une des principales forces de R est d'être un langage hautement
modulaire comptant \textbf{plusieurs milliers de \emph{packages}} (10
865 au 21/06/2017).

\pause Toutes les informations sur un \emph{package} sont accessibles
sur sa page du \emph{Comprehensive R Archive Network} (CRAN).

\textbf{Exemple} \url{https://CRAN.R-project.org/package=haven}

\pause \bigskip On trouve en particulier sur cette page:

\begin{itemize}
\tightlist
\item
  les \textbf{dépendances} du \emph{package} (\emph{Depends} et
  \emph{Imports});
\item
  un lien vers sa \textbf{page de développement} (\emph{URL});
\item
  une \textbf{version .pdf de son aide} (\emph{Reference manual})
\item
  éventuellement un ou plusieurs \textbf{documents de démonstration}
  (\emph{Vignettes}).
\end{itemize}

\end{frame}

\begin{frame}[fragile]{Installer un \emph{package} automatiquement}

La fonction \texttt{install.packages("nomPackage")} permet d'installer
automatiquement le \emph{package} \texttt{nomPackage}.

Les données nécessaires sont téléchargées depuis un des dépôts du CRAN
(\emph{repositories} ou en abrégé \texttt{repos}).

C'est la \textbf{méthode à privilégier}: les dépendances nécessaires au
bon fonctionnement du \emph{package} sont détectées et automatiquement
installées.

\pause 

\textbf{Remarque} Cette méthode fonctionne à l'Insee:

\begin{itemize}
\tightlist
\item
  pour les installations locales de R sur les postes de travail;
\item
  sur AUS, \emph{via} un dépôt local spécifique;
\item
  mais PAS sur les sessions des postes de formation.
\end{itemize}

\end{frame}

\begin{frame}[fragile]{Installer un \emph{package} manuellement}

La page d'information d'un \emph{package} comporte également des liens
vers les fichiers qui le composent.

Quand l'installation directe depuis un dépôt du CRAN est indisponible,
il suffit de \textbf{télécharger ces fichiers} et d'\textbf{installer
manuellement le \emph{package}}.

Pour une installation sous Windows, il faut privilégier les
\textbf{fichiers compilés} (\emph{Windows binaries}).

\pause \small

\begin{Shaded}
\begin{Highlighting}[]
\CommentTok{# Note : Le fichier haven._1.0.0.zip est situé }
\CommentTok{# dans le répertoire de travail}
\KeywordTok{install.packages}\NormalTok{(}
  \StringTok{"haven_1.0.0.zip"}\NormalTok{, }\DataTypeTok{repos =} \OtherTok{NULL}\NormalTok{, }\DataTypeTok{type =} \StringTok{"binaries"}
\NormalTok{)}
\end{Highlighting}
\end{Shaded}

\end{frame}

\begin{frame}[fragile]{Installer des \emph{packages} depuis github}

En règle générale, le développement de \emph{packages} s'appuie sur des
plate-formes de \textbf{développement collaboratif} comme
\href{https://github.com}{\underline{Github}}.

\pause La \textbf{page de développement} d'un \emph{package} comporte
plusieurs informations préciseuses :

\begin{itemize}
\tightlist
\item
  la dernière version du \emph{package} et de sa documentation;
\item
  des informations sur son développement;
\item
  une zone pour rapporter d'éventuels \emph{bugs} (\emph{bug reports}).
\end{itemize}

\textbf{Exemple} \url{https://github.com/tidyverse/haven}

\pause La fonction \texttt{install\_github()} du \emph{package}
\texttt{devtools} permet d'installer un \emph{package} directement
depuis GitHub.

\begin{Shaded}
\begin{Highlighting}[]
\KeywordTok{library}\NormalTok{(devtools)}
\KeywordTok{install_github}\NormalTok{(}\StringTok{"tidyverse/haven"}\NormalTok{)}
\end{Highlighting}
\end{Shaded}

\end{frame}

\begin{frame}[fragile]{Utiliser les données d'exemples d'un
\emph{package}}

La plupart des \textbf{packages} contiennent des \textbf{données
d'exemples} utilisées notamment dans son aide ou ses vignettes.

Une fois le \emph{package} installé, il suffit d'utiliser la fonction
\texttt{data(package\ =\ "nomPackage")} pour afficher les données qu'il
contient.

\begin{Shaded}
\begin{Highlighting}[]
\KeywordTok{library}\NormalTok{(ggplot2)}
\KeywordTok{data}\NormalTok{(}\DataTypeTok{package =} \StringTok{"ggplot2"}\NormalTok{)}
\end{Highlighting}
\end{Shaded}

\pause Pour \og rapatrier \fg{} dans l'environnement global les données
d'un \emph{package}, c'est de nouveau la fonction \texttt{data()} qu'il
faut utiliser.

\begin{Shaded}
\begin{Highlighting}[]
\KeywordTok{data}\NormalTok{(mpg)}
\end{Highlighting}
\end{Shaded}

\end{frame}

\end{document}
