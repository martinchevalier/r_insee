\documentclass[12pt,handout,ignorenonframetext,]{beamer}
\setbeamertemplate{caption}[numbered]
\setbeamertemplate{caption label separator}{: }
\setbeamercolor{caption name}{fg=normal text.fg}
\beamertemplatenavigationsymbolsempty
\usepackage{lmodern}
\usepackage{amssymb,amsmath}
\usepackage{ifxetex,ifluatex}
\usepackage{fixltx2e} % provides \textsubscript
\ifnum 0\ifxetex 1\fi\ifluatex 1\fi=0 % if pdftex
\usepackage[T1]{fontenc}
\usepackage[utf8]{inputenc}
\else % if luatex or xelatex
\ifxetex
\usepackage{mathspec}
\else
\usepackage{fontspec}
\fi
\defaultfontfeatures{Ligatures=TeX,Scale=MatchLowercase}
\fi
% use upquote if available, for straight quotes in verbatim environments
\IfFileExists{upquote.sty}{\usepackage{upquote}}{}
% use microtype if available
\IfFileExists{microtype.sty}{%
\usepackage{microtype}
\UseMicrotypeSet[protrusion]{basicmath} % disable protrusion for tt fonts
}{}
\newif\ifbibliography
\usepackage{color}
\usepackage{fancyvrb}
\newcommand{\VerbBar}{|}
\newcommand{\VERB}{\Verb[commandchars=\\\{\}]}
\DefineVerbatimEnvironment{Highlighting}{Verbatim}{commandchars=\\\{\}}
% Add ',fontsize=\small' for more characters per line
\newenvironment{Shaded}{}{}
\newcommand{\KeywordTok}[1]{\textcolor[rgb]{0.00,0.00,1.00}{{#1}}}
\newcommand{\DataTypeTok}[1]{{#1}}
\newcommand{\DecValTok}[1]{{#1}}
\newcommand{\BaseNTok}[1]{{#1}}
\newcommand{\FloatTok}[1]{{#1}}
\newcommand{\ConstantTok}[1]{{#1}}
\newcommand{\CharTok}[1]{\textcolor[rgb]{0.00,0.50,0.50}{{#1}}}
\newcommand{\SpecialCharTok}[1]{\textcolor[rgb]{0.00,0.50,0.50}{{#1}}}
\newcommand{\StringTok}[1]{\textcolor[rgb]{0.00,0.50,0.50}{{#1}}}
\newcommand{\VerbatimStringTok}[1]{\textcolor[rgb]{0.00,0.50,0.50}{{#1}}}
\newcommand{\SpecialStringTok}[1]{\textcolor[rgb]{0.00,0.50,0.50}{{#1}}}
\newcommand{\ImportTok}[1]{{#1}}
\newcommand{\CommentTok}[1]{\textcolor[rgb]{0.00,0.50,0.00}{{#1}}}
\newcommand{\DocumentationTok}[1]{\textcolor[rgb]{0.00,0.50,0.00}{{#1}}}
\newcommand{\AnnotationTok}[1]{\textcolor[rgb]{0.00,0.50,0.00}{{#1}}}
\newcommand{\CommentVarTok}[1]{\textcolor[rgb]{0.00,0.50,0.00}{{#1}}}
\newcommand{\OtherTok}[1]{\textcolor[rgb]{1.00,0.25,0.00}{{#1}}}
\newcommand{\FunctionTok}[1]{{#1}}
\newcommand{\VariableTok}[1]{{#1}}
\newcommand{\ControlFlowTok}[1]{\textcolor[rgb]{0.00,0.00,1.00}{{#1}}}
\newcommand{\OperatorTok}[1]{{#1}}
\newcommand{\BuiltInTok}[1]{{#1}}
\newcommand{\ExtensionTok}[1]{{#1}}
\newcommand{\PreprocessorTok}[1]{\textcolor[rgb]{1.00,0.25,0.00}{{#1}}}
\newcommand{\AttributeTok}[1]{{#1}}
\newcommand{\RegionMarkerTok}[1]{{#1}}
\newcommand{\InformationTok}[1]{\textcolor[rgb]{0.00,0.50,0.00}{{#1}}}
\newcommand{\WarningTok}[1]{\textcolor[rgb]{0.00,0.50,0.00}{\textbf{{#1}}}}
\newcommand{\AlertTok}[1]{\textcolor[rgb]{1.00,0.00,0.00}{{#1}}}
\newcommand{\ErrorTok}[1]{\textcolor[rgb]{1.00,0.00,0.00}{\textbf{{#1}}}}
\newcommand{\NormalTok}[1]{{#1}}
\usepackage{longtable,booktabs}
\usepackage{caption}
% These lines are needed to make table captions work with longtable:
\makeatletter
\def\fnum@table{\tablename~\thetable}
\makeatother
\usepackage{graphicx,grffile}
\makeatletter
\def\maxwidth{\ifdim\Gin@nat@width>\linewidth\linewidth\else\Gin@nat@width\fi}
\def\maxheight{\ifdim\Gin@nat@height>\textheight0.8\textheight\else\Gin@nat@height\fi}
\makeatother
% Scale images if necessary, so that they will not overflow the page
% margins by default, and it is still possible to overwrite the defaults
% using explicit options in \includegraphics[width, height, ...]{}
\setkeys{Gin}{width=\maxwidth,height=\maxheight,keepaspectratio}

% Prevent slide breaks in the middle of a paragraph:
\widowpenalties 1 10000
\raggedbottom

\AtBeginPart{
\let\insertpartnumber\relax
\let\partname\relax
\frame{\partpage}
}
\AtBeginSection{
\ifbibliography
\else
\let\insertsectionnumber\relax
\let\sectionname\relax
\frame{\sectionpage}
\fi
}
\AtBeginSubsection{
\let\insertsubsectionnumber\relax
\let\subsectionname\relax
\frame{\subsectionpage}
}

\setlength{\parindent}{0pt}
\setlength{\parskip}{6pt plus 2pt minus 1pt}
\setlength{\emergencystretch}{3em}  % prevent overfull lines
\providecommand{\tightlist}{%
\setlength{\itemsep}{0pt}\setlength{\parskip}{0pt}}
\setcounter{secnumdepth}{0}

\usepackage[french]{babel}
\usepackage{lmodern}
\usepackage{graphicx}
\usepackage{xcolor}
\usepackage{textcomp} 
\usepackage{amsmath, amsfonts, amssymb, amsthm}
\usepackage{booktabs,multirow}
\usepackage{setspace}
\usepackage{float}
\usepackage{pgfpages}
\usepackage{colortbl}
\usepackage{epstopdf}
\usepackage{framed}



\definecolor{shadecolor}{RGB}{248,248,248}
\definecolor{grayInsee}{HTML}{5a5758}
\definecolor{redInsee}{HTML}{ed1443}

%Everything about the notes and formatting of the notepage
\setbeamertemplate{note page}{%
	Notes personnelles
	\insertnote%
}


\setbeamertemplate{navigation symbols}{}
\usetheme{default} %Malmoe not bad
\setbeamertemplate{footline}[frame number]


%\setbeamerfont{frametitle}{size=\normalsize}
%\setbeamerfont{framesubtitle}{size=\Large}
%\setbeamercolor{frametitle}{fg=grayInsee}
%\setbeamercolor{framesubtitle}{fg=redInsee}
\setbeamercolor{title}{fg=grayInsee}
\setbeamercolor{subsection in toc}{fg=grayInsee}
\setbeamertemplate{frametitle}{%
	\large \textcolor{grayInsee}{\subsecname}
	\\ \vspace{0.1cm} \Large \textcolor{redInsee}{\insertframetitle}
}
%\setbeamertemplate{frametitle}{%
%	\large \textcolor{grayInsee}{
%		\ifx\intertsubsection\emptyset
%			\secname \\ \vspace{0.1cm} 
%		\else 
%			\subsecname \\ \vspace{0.1cm}
%		\fi
%	}
%	\Large \textcolor{redInsee}{\insertframetitle}
%}
\setbeamercolor{local structure}{fg=redInsee}

\AtBeginSection[]
{\ifnum \thesection>1
  \begin{frame}
  \vfill
  \begin{center}
  \LARGE
  \textcolor{grayInsee}{\insertsectionhead}
  \end{center}
  \vfill
  \end{frame}
\else
\fi
}

\AtBeginSubsection[]{}

\title{\Large Formation \textbf{R} Perfectionnement}

\institute{ \includegraphics[height = 2.5cm]{../figures/Logo_Insee.png}\\ ~ \\ \normalsize Martin \textsc{Chevalier} (Insee)}

\author{21-22 juin 2017}

\date{}

\renewenvironment{Shaded}{\begin{snugshade}}{\end{snugshade}}

\newcommand{\aparte}[2]{
	{\small\textsf{\textbf{#1} #2}}
}

%\usepackage{enumitem}
%\setlist{nolistsep}

\usepackage{tikz}
\usetikzlibrary{shapes,arrows,calc, positioning}
\tikzstyle{input} = [draw, rectangle,rounded corners, text width=2.5cm, fill=green!20, node distance=0.5cm, minimum height=2em, text centered]
\tikzstyle{output} = [draw, ellipse,fill=red!20, node distance=0.5cm, minimum height=2em, text centered]
\tikzstyle{block} = [rectangle, draw, fill=blue!20, 
    text width=1.5cm, text centered, minimum height=2em, node distance = 0.5cm]
\tikzstyle{line} = [draw, -latex', shorten >=2pt, shorten <=2pt]
\tikzset{
  invisible/.style={opacity=0},
  visible on/.style={alt={#1{}{invisible}}},
  alt/.code args={<#1>#2#3}{%
    \alt<#1>{\pgfkeysalso{#2}}{\pgfkeysalso{#3}} % \pgfkeysalso doesn't change the path
  },
}

%\usepackage{pgfpages}
%\mode<handout>{
%	%\setbeamercolor{background canvas}{bg=black!20}
%	\pgfpagesuselayout{2 on 1}[border shrink=2mm]
%}

\title{Formation R Perfectionnement}
\date{}

\begin{document}
\frame{\titlepage}

\section{\texorpdfstring{\LARGE Objectifs et
organisation}{Objectifs et organisation}}\label{objectifs-et-organisation}

\subsection*{\LARGE Objectifs et organisation}

\begin{frame}{~}

\large  Apprendre à perfectionner son utilisation de R : acquérir des
points de repères, des réflexes, des méthodes de travail.

\bigskip \pause 

Effectuer un panorama structuré et hiérarchisé de méthodes et outils
largement utilisés.

\bigskip \pause 

Prendre du recul sur le logiciel, comprendre certains modes de
fonctionnement complexes.

\end{frame}

\begin{frame}[fragile]{~}

\large 

\begin{enumerate}
\def\labelenumi{\arabic{enumi}.}
\item
  Travailler sur des données dans R :

  \begin{itemize}
  \tightlist
  \item
    travailler efficacement (+++);
  \item
    optimiser les performances (++);
  \item
    programmer avec R (+);
  \item
    interroger des bases de données (++).
  \end{itemize}
\item
  \pause Présenter des résultats avec R :

  \begin{itemize}
  \tightlist
  \item
    faire des graphiques avec base R et \texttt{ggplot2} (++);
  \item
    faire du reporting (+).
  \end{itemize}
\end{enumerate}

\normalsize \pause

\textbf{Pédagogie} : équilibre entre présentations et cas pratiques.

\textbf{Horaires (proposition !)} : 9h30-12h20, 13h40-16h30

\end{frame}

\begin{frame}[fragile]{~}

\large

Mercredi 21 juin

\begin{itemize}
\tightlist
\item
  Introduction + Travailler sur des données 1 (2h)
\item
  Cas pratiques (2h)
\item
  Faire des graphiques avec \texttt{ggplot2} (2h)
\end{itemize}

\pause Jeudi 22 juin

\begin{itemize}
\tightlist
\item
  Travailler sur des données 2 + R Markdown (2h)
\item
  Cas pratiques (au choix, 4h)
\end{itemize}

\end{frame}

\section{Introduction :
Se~perfectionner~avec~R}\label{introduction-seperfectionneravecr}

\subsection*{Introduction : Se\ perfectionner\ avec\ R}

\begin{frame}{Connaître plus ou connaître mieux ?}

Comme tout langage statistique ou de programmation, R repose sur un
ensemble d'instructions plus ou moins complexes.

\pause \bigskip
Se perfectionner dans la maîtrise de R peut donc signifier deux choses :

\begin{itemize}
\tightlist
\item
  étendre son \og vocabulaire \fg{} d'instructions connues ;
\item
  mieux comprendre les instructions déjà connues.
\end{itemize}

\pause \bigskip
En pratique, les deux \textbf{vont de pair} : en découvrant de nouvelles
fonctions, on est souvent amené à mieux comprendre le fonctionnement de
celles que l'on croyait maîtriser.

\end{frame}

\begin{frame}{Plan de la partie}

\Large 
\tableofcontents[currentsection, sectionstyle = hide, subsectionstyle = show/show/hide]

\end{frame}

\subsection{Chercher (et trouver !) de
l'aide}\label{chercher-et-trouver-de-laide}

\begin{frame}[fragile]{Savoir utiliser l'aide du logiciel}

À tout moment, taper \texttt{help(nomFonction)} ou
\texttt{?\ nomFonction} affiche l'aide de la fonction
\texttt{nomFonction}.

\begin{Shaded}
\begin{Highlighting}[]
\CommentTok{# Aide de la fonction read.csv}
\NormalTok{? read.csv}
\end{Highlighting}
\end{Shaded}

\pause 

\textbf{Remarque} Pour afficher l'aide sur une fonction d'un
\emph{package}, il faut que celui-ci soit au préalable chargé (avec
\texttt{library()} ou \texttt{require()}).

\pause La fonction \texttt{help.search()} ou la commande \texttt{??}
permettent d'effectuer une recherche approximative:

\begin{Shaded}
\begin{Highlighting}[]
\CommentTok{# Recherche à partir du mot-clé csv}
\NormalTok{?? csv}
\end{Highlighting}
\end{Shaded}

\end{frame}

\begin{frame}{Chercher de l'aide en ligne}

Bien souvent, le problème que l'on rencontre a \textbf{déjà été
rencontré par d'autres}.

\bigskip
Pour progresser dans la maîtrise de R, il ne faut donc surtout pas
hésiter à s'appuyer sur les forums de discussion, comme par exemple
\href{http://stackoverflow.com/questions/tagged/r}{\underline{stackoverflow}}.

\pause \bigskip
On gagne ainsi souvent beaucoup de temps en formulant le problème que
l'on rencontre dans un \textbf{moteur de recherche} pour consulter
certaines réponses.

\bigskip  Quand une question semble ne pas avoir été déjà posée, ne pas
hésiter à la poser soi-même, en joignant alors un \textbf{exemple
reproductible} (\emph{minimal working example} ou MWE).

\end{frame}

\begin{frame}[fragile]{Afficher le code d'une fonction}

Quand l'utilisation d'une fonction pose problème (message d'erreur
inattendu), il est souvent utile d'\textbf{afficher son code} pour
comprendre d'où vient le problème.

\pause Pour ce faire, il suffit de saisir son nom sans parenthèses.

\footnotesize

\begin{Shaded}
\begin{Highlighting}[]
\CommentTok{# Code de la fonction read.csv}
\NormalTok{read.csv}
  \NormalTok{## function (file, header = TRUE, sep = ",", quote = "\textbackslash{}"", dec = ".", }
  \NormalTok{##     fill = TRUE, comment.char = "", ...) }
  \NormalTok{## read.table(file = file, header = header, sep = sep, quote = quote, }
  \NormalTok{##     dec = dec, fill = fill, comment.char = comment.char, ...)}
  \NormalTok{## <bytecode: 0x3c13ac0>}
  \NormalTok{## <environment: namespace:utils>}
\end{Highlighting}
\end{Shaded}

\pause \normalsize
Afficher le code d'une fonction est dans certains cas plus difficile,
\emph{cf.}
\href{http://stackoverflow.com/questions/19226816/how-can-i-view-the-source-code-for-a-function}{\underline{stackoverflow}}.

\end{frame}

\subsection{Découvrir de nouvelles
fonctionnalités}\label{decouvrir-de-nouvelles-fonctionnalites}

\begin{frame}{Se repérer dans les CRAN \protect\textit{Task Views}}

Les CRAN \emph{Task Views} recensent les fonctions et \emph{packages} de
façon thématique. Elles sont mises à jour régulièrement et portent sur
des thèmes variés:

\pause

\footnotesize \href{https://cran.r-project.org/web/views/Bayesian.html}{Bayesian},
\href{https://cran.r-project.org/web/views/ChemPhys.html}{ChemPhys},
\href{https://cran.r-project.org/web/views/ClinicalTrials.html}{ClinicalTrials},
\href{https://cran.r-project.org/web/views/Cluster.html}{Cluster},
\href{https://cran.r-project.org/web/views/DifferentialEquations.html}{DifferentialEquations},
\href{https://cran.r-project.org/web/views/Distributions.html}{Distributions},
\href{https://cran.r-project.org/web/views/Econometrics.html}{Econometrics},
\href{https://cran.r-project.org/web/views/Environmetrics.html}{Environmetrics},
\href{https://cran.r-project.org/web/views/ExperimentalDesign.html}{ExperimentalDesign},
\href{https://cran.r-project.org/web/views/ExtremeValue.html}{ExtremeValue},
\href{https://cran.r-project.org/web/views/Finance.html}{Finance},
\href{https://cran.r-project.org/web/views/FunctionalData.html}{FunctionalData},
\href{https://cran.r-project.org/web/views/Genetics.html}{Genetics},
\href{https://cran.r-project.org/web/views/Graphics.html}{Graphics},
\href{https://cran.r-project.org/web/views/HighPerformanceComputing.html}{HighPerformanceComputing},
\href{https://cran.r-project.org/web/views/MachineLearning.html}{MachineLearning},
\href{https://cran.r-project.org/web/views/MedicalImaging.html}{MedicalImaging},
\href{https://cran.r-project.org/web/views/MetaAnalysis.html}{MetaAnalysis},
\href{https://cran.r-project.org/web/views/Multivariate.html}{Multivariate},
\href{https://cran.r-project.org/web/views/NaturalLanguageProcessing.html}{NaturalLanguageProcessing},
\href{https://cran.r-project.org/web/views/NumericalMathematics.html}{NumericalMathematics},
\href{https://cran.r-project.org/web/views/OfficialStatistics.html}{OfficialStatistics},
\href{https://cran.r-project.org/web/views/Optimization.html}{Optimization},
\href{https://cran.r-project.org/web/views/Pharmacokinetics.html}{Pharmacokinetics},
\href{https://cran.r-project.org/web/views/Phylogenetics.html}{Phylogenetics},
\href{https://cran.r-project.org/web/views/Psychometrics.html}{Psychometrics},
\href{https://cran.r-project.org/web/views/ReproducibleResearch.html}{ReproducibleResearch},
\href{https://cran.r-project.org/web/views/Robust.html}{Robust},
\href{https://cran.r-project.org/web/views/SocialSciences.html}{SocialSciences},
\href{https://cran.r-project.org/web/views/Spatial.html}{Spatial},
\href{https://cran.r-project.org/web/views/SpatioTemporal.html}{SpatioTemporal},
\href{https://cran.r-project.org/web/views/Survival.html}{Survival},
\href{https://cran.r-project.org/web/views/TimeSeries.html}{TimeSeries},
\href{https://cran.r-project.org/web/views/WebTechnologies.html}{WebTechnologies},
\href{https://cran.r-project.org/web/views/gR.html}{gR}

\pause \bigskip \normalsize
La liste de toutes les \emph{Task Views} est accessible à la page :
\href{https://cran.r-project.org/web/views}{\underline{https://cran.r-project.org/web/views}}.

\end{frame}

\begin{frame}{Consulter des sites, tutoriels, livres}

De plus en plus de supports sont consacrés à la présentation et à
l'enseignement des fonctionnalités de R, comme par exemple :

\begin{itemize}
\item
  \pause le site
  \href{https://www.r-bloggers.com}{\underline{R-bloggers}}: articles en
  général courts sur des exemples d'applications (de qualité inégale);
\item
  \pause le site \href{https://bookdown.org}{\underline{bookdown.org}}:
  dépôt de livres numériques consacrés à R élaborés avec R Markdown
  (très riches et très complets);
\item
  \pause le site de \href{https://www.rstudio.com}{\underline{RStudio}}:
  nombreux
  \href{https://www.rstudio.com/resources/cheatsheets/}{\underline{aides-mémoires}}
  ou articles présentant les fonctionnalités de l'écosystème RStudio;
\item
  \pause les ouvrages de
  \href{http://hadley.nz}{\underline{Hadley Wickham}}:
  \href{https://github.com/hadley/ggplot2-book}{\underline{ggplot2: elegant graphics for data analysis}}
  (à compiler soi-même),
  \href{http:/:adv-r.had.co.nz}{\underline{Advanced R}}.
\end{itemize}

\end{frame}

\subsection{\texorpdfstring{Utiliser de nouveaux
\protect\textit{packages}}{Utiliser de nouveaux }}\label{utiliser-de-nouveaux}

\begin{frame}{Accéder à la documentation d'un \emph{package}}

Une des principales forces de R est d'être un langage hautement
modulaire comptant \textbf{plusieurs milliers de \emph{packages}} (0 au
21/06/2017).

\pause Toutes les informations sur un \emph{package} sont accessibles
sur sa page du \emph{Comprehensive R Archive Network} (CRAN).

\textbf{Exemple} \url{https://CRAN.R-project.org/package=haven}

\pause \bigskip On trouve en particulier sur cette page:

\begin{itemize}
\tightlist
\item
  les \textbf{dépendances} du \emph{package} (\emph{Depends} et
  \emph{Imports});
\item
  un lien vers sa \textbf{page de développement} (\emph{URL});
\item
  une \textbf{version .pdf de son aide} (\emph{Reference manual})
\item
  éventuellement un ou plusieurs \textbf{documents de démonstration}
  (\emph{Vignettes}).
\end{itemize}

\end{frame}

\begin{frame}[fragile]{Installer un \emph{package} automatiquement}

La fonction \texttt{install.packages("nomPackage")} permet d'installer
automatiquement le \emph{package} \texttt{nomPackage}.

Les données nécessaires sont téléchargées depuis un des dépôts du CRAN
(\emph{repositories} ou en abrégé \texttt{repos}).

C'est la \textbf{méthode à privilégier}: les dépendances nécessaires au
bon fonctionnement du \emph{package} sont détectées et automatiquement
installées.

\pause 

\textbf{Remarque} Cette méthode fonctionne à l'Insee:

\begin{itemize}
\tightlist
\item
  pour les installations locales de R sur les postes de travail;
\item
  sur AUS, \emph{via} un dépôt local spécifique;
\item
  mais PAS sur les sessions des postes de formation.
\end{itemize}

\end{frame}

\begin{frame}[fragile]{Installer un \emph{package} manuellement}

La page d'information d'un \emph{package} comporte également des liens
vers les fichiers qui le composent.

Quand l'installation directe depuis un dépôt du CRAN est indisponible,
il suffit de \textbf{télécharger ces fichiers} et d'\textbf{installer
manuellement le \emph{package}}.

Pour une installation sous Windows, il faut privilégier les
\textbf{fichiers compilés} (\emph{Windows binaries}).

\pause \small

\begin{Shaded}
\begin{Highlighting}[]
\CommentTok{# Note : Le fichier haven._1.0.0.zip est situé }
\CommentTok{# dans le répertoire de travail}
\KeywordTok{install.packages}\NormalTok{(}
  \StringTok{"haven_1.0.0.zip"}\NormalTok{, }\DataTypeTok{repos =} \OtherTok{NULL}\NormalTok{, }\DataTypeTok{type =} \StringTok{"binaries"}
\NormalTok{)}
\end{Highlighting}
\end{Shaded}

\end{frame}

\begin{frame}[fragile]{Installer des \emph{packages} depuis github}

En règle générale, le développement de \emph{packages} s'appuie sur des
plate-formes de \textbf{développement collaboratif} comme
\href{https://github.com}{\underline{Github}}.

\pause La \textbf{page de développement} d'un \emph{package} comporte
plusieurs informations préciseuses :

\begin{itemize}
\tightlist
\item
  la dernière version du \emph{package} et de sa documentation;
\item
  des informations sur son développement;
\item
  une zone pour rapporter d'éventuels \emph{bugs} (\emph{bug reports}).
\end{itemize}

\textbf{Exemple} \url{https://github.com/tidyverse/haven}

\pause La fonction \texttt{install\_github()} du \emph{package}
\texttt{devtools} permet d'installer un \emph{package} directement
depuis GitHub.

\begin{Shaded}
\begin{Highlighting}[]
\KeywordTok{library}\NormalTok{(devtools)}
\KeywordTok{install_github}\NormalTok{(}\StringTok{"tidyverse/haven"}\NormalTok{)}
\end{Highlighting}
\end{Shaded}

\end{frame}

\begin{frame}[fragile]{Utiliser les données d'exemples d'un
\emph{package}}

La plupart des \textbf{packages} contiennent des \textbf{données
d'exemples} utilisées notamment dans son aide ou ses vignettes.

Une fois le \emph{package} installé, il suffit d'utiliser la fonction
\texttt{data(package\ =\ "nomPackage")} pour afficher les données qu'il
contient.

\begin{Shaded}
\begin{Highlighting}[]
\KeywordTok{library}\NormalTok{(ggplot2)}
\KeywordTok{data}\NormalTok{(}\DataTypeTok{package =} \StringTok{"ggplot2"}\NormalTok{)}
\end{Highlighting}
\end{Shaded}

\pause Pour \og rapatrier \fg{} dans l'environnement global les données
d'un \emph{package}, c'est de nouveau la fonction \texttt{data()} qu'il
faut utiliser.

\begin{Shaded}
\begin{Highlighting}[]
\KeywordTok{data}\NormalTok{(mpg)}
\end{Highlighting}
\end{Shaded}

\end{frame}

\section{Travailler~efficacement~sur~des~données~avec~R}\label{travaillerefficacementsurdesdonneesavecr}

\subsection*{Travailler~efficacement~sur~des~données~avec~R}\label{travaillerefficacementsurdesdonneesavecr-1}
\addcontentsline{toc}{subsection}{Travailler~efficacement~sur~des~données~avec~R}

\begin{frame}{Qu'est-ce que travailler efficacement avec R ?}

Appliqué au travail sur des données, l'efficacité peut avoir au moins
deux significations distinctes :

\begin{itemize}
\tightlist
\item
  efficacité \textbf{algorithmique} : minimisation du temps passé par la
  machine pour réaliser une série d'opérations;
\item
  \textbf{productivité} du programmeur : minimisation du temps passé à
  coder une série d'opération.
\end{itemize}

\pause En règle générale, on peut avoir l'idée que plus on souhaite être
efficace algorithmiquement, plus la programmation risque d'être longue
et difficile.

\pause \textbf{Ce n'est pas toujours vrai} : on perd souvent beaucoup de
temps à (ré)inventer une méthode peu efficace quand une beaucoup plus
simple et rapide existe déjà.

\pause \small

\textbf{Référence} \textsc{Gillepsie C., Lovelace R.},
\textit{Efficient R programming} (disponible sur
\href{\%5D(https://bookdown.org/csgillepsie/efficientR)}{\underline{bookdown.org}})

\end{frame}

\begin{frame}[fragile]{Mesure l'efficacité algorithmique}

La fonction \texttt{system.time()} permet de mesurer la durée d'un
traitement.

\footnotesize

\begin{Shaded}
\begin{Highlighting}[]
\KeywordTok{system.time}\NormalTok{(}\KeywordTok{rnorm}\NormalTok{(}\FloatTok{1e6}\NormalTok{))}
  \NormalTok{## utilisateur     système      écoulé }
  \NormalTok{##       0.068       0.000       0.069}
\end{Highlighting}
\end{Shaded}

\pause \normalsize
Néanmoins, elle est inadaptée aux traitements de très courte durée. Dans
ces situations, privilégier la fonction \texttt{microbenchmark()} du
package \texttt{microbenchmark}.

\footnotesize

\begin{Shaded}
\begin{Highlighting}[]
\KeywordTok{library}\NormalTok{(microbenchmark)}
\KeywordTok{microbenchmark}\NormalTok{(}\DataTypeTok{times =} \DecValTok{10}\NormalTok{, }\KeywordTok{rnorm}\NormalTok{(}\FloatTok{1e6}\NormalTok{))}
  \NormalTok{## Unit: milliseconds}
  \NormalTok{##          expr      min       lq    mean   median}
  \NormalTok{##  rnorm(1e+06) 66.47196 66.94083 68.3047 67.37058}
  \NormalTok{##      uq      max neval}
  \NormalTok{##  68.094 76.13308    10}
\end{Highlighting}
\end{Shaded}

\end{frame}

\begin{frame}[fragile]{Mesurer la taille d'un objet en mémoire}

R stocke l'ensemble des fichiers sur lesquels il travaille dans la
mémoire vive.

Afin de loger les objets les plus gros mais aussi d'optimiser les
performances, il est souvent utile de \textbf{limiter la taille des
objets} sur lesquels portent les traitements.

\pause Pour mesurer la taille des objets, utiliser la fonction
\texttt{object\_size()} du \emph{package} \texttt{pryr}.

\begin{Shaded}
\begin{Highlighting}[]
\KeywordTok{library}\NormalTok{(pryr)}
\KeywordTok{object_size}\NormalTok{(}\KeywordTok{rnorm}\NormalTok{(}\FloatTok{1e6}\NormalTok{))}
  \NormalTok{## 8 MB}
\end{Highlighting}
\end{Shaded}

\end{frame}

\begin{frame}[fragile]{Construire un exemple reproductible (MWE)}

Lorsque l'on cherche à améliorer les performances d'un programme, il est
important de pouvoir le tester sur des données \textbf{autonomes et
reproductibles}.

\pause Pour ce faire, les \textbf{fonctions de générations de nombres
aléatoires} de R sont particulièrement utiles.

\footnotesize

\begin{Shaded}
\begin{Highlighting}[]
\CommentTok{# Graine pour pouvoir reproduire l'aléa}
\KeywordTok{set.seed}\NormalTok{(}\DecValTok{2016}\NormalTok{)}

\CommentTok{# Vecteur de nombres de taille 1 000}
\NormalTok{a <-}\StringTok{ }\KeywordTok{rnorm}\NormalTok{(}\DecValTok{1000}\NormalTok{)}

\CommentTok{# Vecteur de lettres de taille 1 000}
\NormalTok{b <-}\StringTok{ }\NormalTok{letters[}\KeywordTok{sample}\NormalTok{(}\DecValTok{1}\NormalTok{:}\DecValTok{26}\NormalTok{, }\DecValTok{1000}\NormalTok{, }\DataTypeTok{replace =} \OtherTok{TRUE}\NormalTok{)]}

\CommentTok{# Matrice logique 1 000 x 100 avec 1 % de TRUE}
\NormalTok{c <-}\StringTok{ }\KeywordTok{matrix}\NormalTok{(}\KeywordTok{runif}\NormalTok{(}\DecValTok{100000}\NormalTok{) >}\StringTok{ }\FloatTok{0.99}\NormalTok{, }\DataTypeTok{ncol =} \DecValTok{100}\NormalTok{)}
\end{Highlighting}
\end{Shaded}

\end{frame}

\begin{frame}{Plan de la partie}

\large  \vfill
\tableofcontents[currentsection, sectionstyle = hide, subsectionstyle = show/show/hide]
\vfill

\end{frame}

\subsection{De l'importance des fonctions dans
R}\label{de-limportance-des-fonctions-dans-r}

\begin{frame}[fragile]{\og Tout ce qui agit est un appel de fonction
\fg{}}

\begin{quote}
\emph{To understand computations in R, two slogans are helpful:}

\begin{itemize}
\item
  \emph{Everything that exists is an object.}
\item
  \emph{Everything that happens is a function call.}
\end{itemize}

\emph{John Chambers}
\end{quote}

\pause \footnotesize

\begin{Shaded}
\begin{Highlighting}[]
\CommentTok{# ... même assigner une valeur}
\KeywordTok{is.function}\NormalTok{(}\StringTok{`}\DataTypeTok{<-}\StringTok{`}\NormalTok{)}
  \NormalTok{## [1] TRUE}
\StringTok{`}\DataTypeTok{<-}\StringTok{`}\NormalTok{(a, }\DecValTok{10}\NormalTok{)}

\CommentTok{# ... même afficher la valeur d'un objet}
\NormalTok{a}
  \NormalTok{## [1] 10}
\KeywordTok{print}\NormalTok{(a)}
  \NormalTok{## [1] 10}
\end{Highlighting}
\end{Shaded}

\end{frame}

\begin{frame}[fragile]{Définir une fonction dans R}

Utilisé avec \texttt{\textless{}-}, \texttt{function()} définit une
nouvelle fonction :

\pause \footnotesize

\begin{Shaded}
\begin{Highlighting}[]
\CommentTok{# Définition de la fonction monCalcul()}
\NormalTok{monCalcul <-}\StringTok{ }\NormalTok{function(a, b)\{}
  \NormalTok{resultat <-}\StringTok{ }\DecValTok{10} \NormalTok{*}\StringTok{ }\NormalTok{a +}\StringTok{ }\NormalTok{b}
  \KeywordTok{return}\NormalTok{(resultat)}
\NormalTok{\}}

\CommentTok{# Code de monCalcul()}
\NormalTok{monCalcul}
  \NormalTok{## function(a, b)\{}
  \NormalTok{##   resultat <- 10 * a + b}
  \NormalTok{##   return(resultat)}
  \NormalTok{## \}}

\CommentTok{# Appel de la fonction monCalcul()}
\KeywordTok{monCalcul}\NormalTok{(}\DecValTok{2}\NormalTok{, }\DecValTok{3}\NormalTok{)}
  \NormalTok{## [1] 23}
\end{Highlighting}
\end{Shaded}

\end{frame}

\begin{frame}[fragile]{Valeurs par défaut des paramètres}

Des valeurs par défaut peuvent être renseignées pour les paramètres.

\begin{Shaded}
\begin{Highlighting}[]
\NormalTok{monCalcul <-}\StringTok{ }\NormalTok{function(a, }\DataTypeTok{b =} \DecValTok{3}\NormalTok{) }\DecValTok{10} \NormalTok{*}\StringTok{ }\NormalTok{a +}\StringTok{ }\NormalTok{b}
\KeywordTok{monCalcul}\NormalTok{(}\DecValTok{8}\NormalTok{)}
  \NormalTok{## [1] 83}
\end{Highlighting}
\end{Shaded}

\pause Les valeurs par défaut peuvent dépendre des autres paramètres.

\begin{Shaded}
\begin{Highlighting}[]
\NormalTok{monCalcul <-}\StringTok{ }\NormalTok{function(a, }\DataTypeTok{b =} \NormalTok{a *}\StringTok{ }\DecValTok{2}\NormalTok{) }\DecValTok{10} \NormalTok{*}\StringTok{ }\NormalTok{a +}\StringTok{ }\NormalTok{b}
\KeywordTok{monCalcul}\NormalTok{(}\DecValTok{2}\NormalTok{)}
  \NormalTok{## [1] 24}
\end{Highlighting}
\end{Shaded}

\pause \textbf{Remarque} Ceci est la conséquence de la \emph{lazy
evaluation} des arguments dans R (\emph{cf.}
\href{http://adv-r.had.co.nz/Functions.html\#function-arguments}{\underline{\textit{Advanced R}}}).

\end{frame}

\begin{frame}[fragile]{Contrôle de la valeur des paramètres}

Des structures conditionnelles \texttt{if()} permettent de contrôler la
valeur des arguments.

\pause \footnotesize

\begin{Shaded}
\begin{Highlighting}[]
\NormalTok{monCalcul <-}\StringTok{ }\NormalTok{function(}\DataTypeTok{a =} \OtherTok{NULL}\NormalTok{, }\DataTypeTok{b =} \OtherTok{NULL}\NormalTok{)\{}
  \NormalTok{if(}\KeywordTok{is.null}\NormalTok{(a)) }\KeywordTok{stop}\NormalTok{(}\StringTok{"a n'est pas renseigné."}\NormalTok{)}
  \NormalTok{if(}\KeywordTok{is.null}\NormalTok{(b))\{}
    \NormalTok{b <-}\StringTok{ }\NormalTok{a *}\StringTok{ }\DecValTok{2}
    \KeywordTok{warning}\NormalTok{(}\StringTok{"b n'est pas renseigné."}\NormalTok{)}
  \NormalTok{\}}
  \KeywordTok{return}\NormalTok{(}\DecValTok{10} \NormalTok{*}\StringTok{ }\NormalTok{a +}\StringTok{ }\NormalTok{b)}
\NormalTok{\}}

\KeywordTok{monCalcul}\NormalTok{(}\DataTypeTok{b =} \DecValTok{3}\NormalTok{)}
  \NormalTok{## Error in monCalcul(b = 3): a n'est pas renseigné.}
\KeywordTok{monCalcul}\NormalTok{(}\DataTypeTok{a =} \DecValTok{1}\NormalTok{)}
  \NormalTok{## Warning in monCalcul(a = 1): b n'est pas renseigné.}
  \NormalTok{## [1] 12}
\end{Highlighting}
\end{Shaded}

\end{frame}

\begin{frame}[fragile]{\large Portée des variables et environnements
(1)}

Dans R \textbf{chaque objet est repéré par son nom et son environnement}
: cela permet d'éviter les conflits de noms.

\pause \footnotesize

\begin{Shaded}
\begin{Highlighting}[]
\CommentTok{# Création d'une fonction sum() un peu absurde}
\NormalTok{sum <-}\StringTok{ }\NormalTok{function(...) }\StringTok{"Ma super somme !"}
\KeywordTok{sum}\NormalTok{(}\DecValTok{2}\NormalTok{, }\DecValTok{3}\NormalTok{)}
  \NormalTok{## [1] "Ma super somme !"}

\CommentTok{# Cette fonction est rattachée à l'environnement global}
\KeywordTok{ls}\NormalTok{()}
  \NormalTok{## [1] "a"         "b"         "c"         "monCalcul"}
  \NormalTok{## [5] "sum"}

\CommentTok{# Mais on peut toujours accéder à la fonction }
\CommentTok{# de base en utilisant ::}
\NormalTok{base::}\KeywordTok{sum}\NormalTok{(}\DecValTok{2}\NormalTok{, }\DecValTok{3}\NormalTok{)}
  \NormalTok{## [1] 5}
\end{Highlighting}
\end{Shaded}

\end{frame}

\begin{frame}[fragile]{\large Portée des variables et environnements
(2)}

\small
À chaque appel d'une fonction, un \textbf{environnement d'exécution}
éphémère est créé.

\footnotesize

\begin{Shaded}
\begin{Highlighting}[]
\NormalTok{maFun <-}\StringTok{ }\NormalTok{function() }\KeywordTok{environment}\NormalTok{()}
\KeywordTok{maFun}\NormalTok{()}
  \NormalTok{## <environment: 0x754fdb0>}
\KeywordTok{maFun}\NormalTok{()}
  \NormalTok{## <environment: 0x87f3708>}
\end{Highlighting}
\end{Shaded}

\pause \small
En conséquence, les instructions exécutées à l'intérieur d'une fonction
\textbf{ne modifient pas l'environnement global}.

\footnotesize

\begin{Shaded}
\begin{Highlighting}[]
\NormalTok{a <-}\StringTok{ }\DecValTok{10}
\NormalTok{maFonction3 <-}\StringTok{ }\NormalTok{function()\{}
  \NormalTok{a <-}\StringTok{ }\DecValTok{5}
\NormalTok{\}}
\KeywordTok{maFonction3}\NormalTok{()}
\NormalTok{a}
  \NormalTok{## [1] 10}
\end{Highlighting}
\end{Shaded}

\end{frame}

\begin{frame}[fragile]{\large Portée des variables et environnements
(3)}

En revanche, les objets définis dans l'environnement global sont
accessibles au sein d'une fonction.

\footnotesize

\begin{Shaded}
\begin{Highlighting}[]
\NormalTok{a <-}\StringTok{ }\DecValTok{10}
\NormalTok{maFonction4 <-}\StringTok{ }\NormalTok{function()\{}
  \NormalTok{a +}\StringTok{ }\DecValTok{5}
\NormalTok{\}}
\KeywordTok{maFonction4}\NormalTok{()}
  \NormalTok{## [1] 15}
\end{Highlighting}
\end{Shaded}

\pause \normalsize
Ceci est dû au fait que les environnements dans lequel R recherche des
objets sont \textbf{emboîtés les uns dans les autres} (\textit{cf.} la
fonction \texttt{search()}).

\textbf{Pour en savoir plus}
\href{http://adv-r.had.co.nz/Environments.html\#function-envs}{\underline{\textit{Advanced R}}},
\href{http://blog.obeautifulcode.com/R/How-R-Searches-And-Finds-Stuff/}{\underline{obeautifulcode.com}}

\end{frame}

\begin{frame}[fragile]{Valeur de retour d'une fonction}

La fonction \texttt{return()} spécifie la valeur à renvoyer. Pour
renvoyer plusieurs valeurs, utiliser une liste.

\pause \footnotesize

\begin{Shaded}
\begin{Highlighting}[]
\NormalTok{maFonction1 <-}\StringTok{ }\NormalTok{function()\{}
  \NormalTok{a <-}\StringTok{ }\DecValTok{1}\NormalTok{:}\DecValTok{5}\NormalTok{; b <-}\StringTok{ }\DecValTok{6}\NormalTok{:}\DecValTok{10}\NormalTok{; }\KeywordTok{return}\NormalTok{(a)}
\NormalTok{\}}
\KeywordTok{maFonction1}\NormalTok{()}
  \NormalTok{## [1] 1 2 3 4 5}

\NormalTok{maFonction2 <-}\StringTok{ }\NormalTok{function()\{}
  \NormalTok{a <-}\StringTok{ }\DecValTok{1}\NormalTok{:}\DecValTok{5}\NormalTok{; b <-}\StringTok{ }\DecValTok{6}\NormalTok{:}\DecValTok{10}\NormalTok{; }\KeywordTok{return}\NormalTok{(}\KeywordTok{list}\NormalTok{(}\DataTypeTok{a =} \NormalTok{a, }\DataTypeTok{b =} \NormalTok{b))}
\NormalTok{\}}
\KeywordTok{maFonction2}\NormalTok{()}
  \NormalTok{## $a}
  \NormalTok{## [1] 1 2 3 4 5}
  \NormalTok{## }
  \NormalTok{## $b}
  \NormalTok{## [1]  6  7  8  9 10}
\end{Highlighting}
\end{Shaded}

\end{frame}

\begin{frame}[fragile]{\large Effets de bord et programmation
fonctionnelle}

Par défaut, les fonctions dans R :

\begin{itemize}
\tightlist
\item
  ne modifient pas l'environnement d'origine (il n'y a \textbf{pas
  d'effets de bord});
\item
  peuvent être utilisées en lieu et place des valeurs qu'elles
  retournent.
\end{itemize}

\begin{Shaded}
\begin{Highlighting}[]
\NormalTok{monCalcul <-}\StringTok{  }\NormalTok{function(a, b) }\DecValTok{10} \NormalTok{*}\StringTok{ }\NormalTok{a +}\StringTok{ }\NormalTok{b}
\KeywordTok{monCalcul}\NormalTok{(}\DecValTok{2}\NormalTok{, }\DecValTok{3}\NormalTok{) +}\StringTok{ }\DecValTok{5}
  \NormalTok{## [1] 28}
\end{Highlighting}
\end{Shaded}

\pause Ces éléments font de R un \textbf{langage particulièrement adapté
à la programmation fonctionnelle}.

\end{frame}

\begin{frame}[fragile]{Quelques principes de la programmation
fonctionnelle}

\vfill

\begin{enumerate}
\def\labelenumi{\arabic{enumi}.}
\tightlist
\item
  \textbf{Ne jamais créer d'effets de bord} Toute modification apportée
  à l'environnement par une fonction passe par sa valeur de sortie.
\end{enumerate}

\vfill

\begin{enumerate}
\def\labelenumi{\arabic{enumi}.}
\setcounter{enumi}{1}
\tightlist
\item
  \pause \textbf{Vectoriser \textit{i.e.} appliquer des fonctions systématiquement à un ensemble d'éléments}
  Fonctions \texttt{*apply()}, \texttt{Reduce()}, \texttt{do.call()}.
\end{enumerate}

\vfill

\begin{enumerate}
\def\labelenumi{\arabic{enumi}.}
\setcounter{enumi}{2}
\tightlist
\item
  \pause \textbf{Structurer les traitements à l'aide de fonctions courtes et explicites}
  Faciliter la relecture, la maintenance et la modularisation.
\end{enumerate}

\vfill

\pause \textbf{Pour en savoir plus}
\href{https://en.wikipedia.org/wiki/FP_(programming_language)}{\underline{Wikipedia}},
\href{https://maryrosecook.com/blog/post/a-practical-introduction-to-functional-programming}{\underline{maryrosecook.com}}.

\vfill

\end{frame}

\subsection{\texorpdfstring{Vectoriser : \texttt{*apply()},
\texttt{Reduce()} et
\texttt{do.call()}}{Vectoriser : *apply(), Reduce() et do.call()}}\label{vectoriser-apply-reduce-et-do.call}

\begin{frame}[fragile]{\large Appliquer sur chaque indépendamment :
\texttt{apply()}}

La fonction \texttt{apply(X,\ MARGIN,\ FUN)} applique la fonction
\texttt{FUN} à la matrice \texttt{X} selon la dimension \texttt{MARGIN}.

\pause \footnotesize

\begin{Shaded}
\begin{Highlighting}[]
\CommentTok{# Définition et affichage de la matrice m}
\NormalTok{m <-}\StringTok{ }\KeywordTok{matrix}\NormalTok{(}\DecValTok{1}\NormalTok{:}\DecValTok{6}\NormalTok{, }\DataTypeTok{ncol =} \DecValTok{3}\NormalTok{)}
\NormalTok{m}
  \NormalTok{##      [,1] [,2] [,3]}
  \NormalTok{## [1,]    1    3    5}
  \NormalTok{## [2,]    2    4    6}

\CommentTok{# Application de la fonction sum() selon les lignes}
\KeywordTok{apply}\NormalTok{(m, }\DecValTok{1}\NormalTok{, sum)}
  \NormalTok{## [1]  9 12}

\CommentTok{# Application de la fonction sum() selon les colonnes}
\KeywordTok{apply}\NormalTok{(m, }\DecValTok{2}\NormalTok{, sum)}
  \NormalTok{## [1]  3  7 11}
\end{Highlighting}
\end{Shaded}

\end{frame}

\begin{frame}[fragile]{\large Appliquer sur chaque indépendamment :
\texttt{lapply()}}

\small
La fonction \texttt{lapply(X,\ FUN)} applique la fonction \texttt{FUN} à
l'objet \texttt{X} (vecteur ou liste).

\pause \footnotesize

\begin{Shaded}
\begin{Highlighting}[]
\NormalTok{l <-}\StringTok{ }\KeywordTok{list}\NormalTok{(}\DecValTok{1}\NormalTok{:}\DecValTok{5}\NormalTok{, }\KeywordTok{c}\NormalTok{(}\DecValTok{6}\NormalTok{:}\DecValTok{9}\NormalTok{, }\OtherTok{NA}\NormalTok{))}
\NormalTok{l}
  \NormalTok{## [[1]]}
  \NormalTok{## [1] 1 2 3 4 5}
  \NormalTok{## }
  \NormalTok{## [[2]]}
  \NormalTok{## [1]  6  7  8  9 NA}
\KeywordTok{lapply}\NormalTok{(l, sum)}
  \NormalTok{## [[1]]}
  \NormalTok{## [1] 15}
  \NormalTok{## }
  \NormalTok{## [[2]]}
  \NormalTok{## [1] NA}
\end{Highlighting}
\end{Shaded}

\pause \small \vspace{-0.2cm}

\textbf{Exemple d'utilisation} Appliquer une fonction à toutes les
variables d'une table.

\end{frame}

\begin{frame}[fragile]{\large Appliquer sur chaque indépendamment :
\texttt{sapply()}}

La fonction \texttt{sapply()} est analogue à la fonction
\texttt{lapply()}, mais simplifie le résultat produit quand c'est
possible.

\begin{Shaded}
\begin{Highlighting}[]
\KeywordTok{sapply}\NormalTok{(l, sum)}
  \NormalTok{## [1] 15 NA}
\end{Highlighting}
\end{Shaded}

\pause Les arguments optionnels de la fonction utilisée peuvent être
ajoutés à la suite dans toutes les fonctions \texttt{*apply()}.

\begin{Shaded}
\begin{Highlighting}[]
\KeywordTok{sapply}\NormalTok{(l, sum, }\DataTypeTok{na.rm =} \OtherTok{TRUE}\NormalTok{)}
  \NormalTok{## [1] 15 30}
\end{Highlighting}
\end{Shaded}

\pause \textbf{Exemple d'utilisation} Calcul de statistiques sur toutes
les variables d'une table.

\end{frame}

\begin{frame}[fragile]{\large Définir une fonction à la volée dans
\texttt{*apply()}}

Il est fréquent que l'opération que l'on souhaite appliquer ne
corresponde pas exactement à une fonction pré-existante.

\pause Dans ce cas, on peut définir une \textbf{fonction à la volée}
dans la fonction \texttt{*apply()}.

\footnotesize

\begin{Shaded}
\begin{Highlighting}[]
\CommentTok{# On souhaite sélectionner le second élément de }
\CommentTok{# de chaque vecteur de la liste l}
\NormalTok{l}
  \NormalTok{## [[1]]}
  \NormalTok{## [1] 1 2 3 4 5}
  \NormalTok{## }
  \NormalTok{## [[2]]}
  \NormalTok{## [1]  6  7  8  9 NA}

\CommentTok{# On définit une fonction dans sapply()}
\KeywordTok{sapply}\NormalTok{(l, function(x) x[}\DecValTok{2}\NormalTok{])}
  \NormalTok{## [1] 2 7}
\end{Highlighting}
\end{Shaded}

\end{frame}

\begin{frame}[fragile]{\large Appliquer sur chaque indépendamment :
\texttt{tapply()}}

La fonction \texttt{tapply(X,\ INDEX,\ FUN)} applique la fonction
\texttt{FUN}, à l'objet \texttt{X} ventilé selon les modalités de
\texttt{INDEX}.

\pause 

\begin{Shaded}
\begin{Highlighting}[]
\CommentTok{# Variables d'âge et de sexe}
\NormalTok{age <-}\StringTok{ }\KeywordTok{c}\NormalTok{(}\DecValTok{45}\NormalTok{, }\DecValTok{50}\NormalTok{, }\DecValTok{35}\NormalTok{, }\DecValTok{20}\NormalTok{)}
\NormalTok{sexe <-}\StringTok{ }\KeywordTok{c}\NormalTok{(}\StringTok{"H"}\NormalTok{, }\StringTok{"F"}\NormalTok{, }\StringTok{"F"}\NormalTok{, }\StringTok{"H"}\NormalTok{)}

\CommentTok{# Âge moyen par sexe}
\KeywordTok{tapply}\NormalTok{(age, sexe, mean)}
  \NormalTok{##    F    H }
  \NormalTok{## 42.5 32.5}
\end{Highlighting}
\end{Shaded}

\normalsize \pause

\textbf{Exemple d'utilisation} Calcul de statistiques agrégées par
catégories.

\end{frame}

\begin{frame}[fragile]{\large Appliquer sur tous : \texttt{do.call()}}

La fonction \texttt{do.call(what,\ args)} permet d'appliquer la fonction
\texttt{what()} à un \textbf{ensemble} d'arguments \texttt{args}
spécifié comme une liste (alors que les fonctions \texttt{*apply()}
appliqueraient \texttt{what()} à \textbf{chaque} élément de
\texttt{args}).

\pause \small

\begin{Shaded}
\begin{Highlighting}[]
\CommentTok{# Concaténation des vecteurs de l}
\KeywordTok{do.call}\NormalTok{(base::c, l)}
  \NormalTok{##  [1]  1  2  3  4  5  6  7  8  9 NA}

\CommentTok{# Equivalent à }
\KeywordTok{c}\NormalTok{(l[[}\DecValTok{1}\NormalTok{]], l[[}\DecValTok{2}\NormalTok{]])}
  \NormalTok{##  [1]  1  2  3  4  5  6  7  8  9 NA}
\end{Highlighting}
\end{Shaded}

\pause \normalsize

\textbf{Exemple d'utilisation} Concaténer de nombreuses tables avec
\texttt{rbind()} ou \texttt{cbind()}.

\end{frame}

\begin{frame}[fragile]{\large Appliquer sur tous successivement :
\texttt{Reduce()}}

La fonction \texttt{Reduce(f,\ x)} permet d'appliquer la fonction
\texttt{f()} \textbf{successivement} à l'ensemble des éléments de
\texttt{x} (alors que \texttt{do.call()} appliquerait \texttt{f}
\textbf{simultanément}).

\pause \small

\begin{Shaded}
\begin{Highlighting}[]
\CommentTok{# Application successive de la division au vecteur 1:4}
\KeywordTok{Reduce}\NormalTok{(}\StringTok{`}\DataTypeTok{/}\StringTok{`}\NormalTok{, }\DecValTok{1}\NormalTok{:}\DecValTok{4}\NormalTok{)}
  \NormalTok{## [1] 0.04166667}

\CommentTok{# Equivalent à }
\NormalTok{((}\DecValTok{1}\NormalTok{/}\DecValTok{2}\NormalTok{)/}\DecValTok{3}\NormalTok{)/}\DecValTok{4}
  \NormalTok{## [1] 0.04166667}
\end{Highlighting}
\end{Shaded}

\pause \normalsize

\textbf{Exemple d'utilisation} Fusionner de nombreuses tables avec
\texttt{merge()} (sur les mêmes identifiants).

\end{frame}

\subsection{Coder efficacement en base
R}\label{coder-efficacement-en-base-r}

\begin{frame}[fragile]{L'idée : En faire faire le moins possible à R}

R est un langage dit \og de haut niveau \fg{} : les objets qui le
composent sont relativement faciles d'utilisation, au prix de
performances limitées.

\vfill

À l'inverse, des langages dits de \og bas niveau \fg{} (par exemple C++)
sont plus difficiles à utiliser mais aussi plus efficaces.

\vfill

\pause La plupart des fonctions fondamentales de R font appel à des
fonctions compilées à partir d'un langage de plus bas niveau.

\vfill

D'où le principe : \textbf{limiter au maximum la surcharge liée à R}
pour retomber au plus vite sur des fonctions pré-compilées.

\pause \vfill

\textbf{Remarque} Il est très facile en pratique d'utiliser R comme une
interface vers des langages de plus bas niveau, \emph{cf.~infra} à
propos de \texttt{Rcpp}.

\end{frame}

\begin{frame}[fragile]{Utiliser les boucles avec parcimonie (1)}

Comme la plupart des langages de programmation, R dispose de
\textbf{structures de contrôles} permettant de réaliser des boucles.

\footnotesize

\begin{Shaded}
\begin{Highlighting}[]
\NormalTok{boucle <-}\StringTok{ }\NormalTok{function(x)\{}
  \NormalTok{cumul <-}\StringTok{ }\KeywordTok{rep}\NormalTok{(}\OtherTok{NA}\NormalTok{, }\KeywordTok{length}\NormalTok{(x))}
  \NormalTok{for(i in }\KeywordTok{seq_along}\NormalTok{(x)) }
    \NormalTok{cumul[i] <-}\StringTok{ }\NormalTok{if(i ==}\StringTok{ }\DecValTok{1}\NormalTok{) x[i] else cumul[i -}\StringTok{ }\DecValTok{1}\NormalTok{] +}\StringTok{ }\NormalTok{x[i]}
  \KeywordTok{return}\NormalTok{(cumul)}
\NormalTok{\}}
\KeywordTok{boucle}\NormalTok{(}\DecValTok{1}\NormalTok{:}\DecValTok{5}\NormalTok{)}
  \NormalTok{## [1]  1  3  6 10 15}
\end{Highlighting}
\end{Shaded}

\pause \normalsize
Ces opérations présentent plusieurs inconvénients :

\begin{enumerate}
\def\labelenumi{\arabic{enumi}.}
\tightlist
\item
  Elles sont longues à écrire et assez peu claires;
\item
  Elles reposent sur des effets de bord;
\item
  Elles sont en général très peu \textbf{efficaces algorithmiquement}.
\end{enumerate}

\end{frame}

\begin{frame}[fragile]{Utiliser les boucles avec parcimonie (2)}

\small
Les méthodes de vectorisation sont en général beaucoup plus efficaces
que les boucles en R :

\begin{itemize}
\tightlist
\item
  vectorisation de haut niveau (\emph{cf.} \emph{supra});
\item
  vectorisation de bas niveau : la vectorisation est opérée par le
  langage de bas niveau auquel fait appel R.
\end{itemize}

\pause \footnotesize

\begin{Shaded}
\begin{Highlighting}[]
\KeywordTok{microbenchmark}\NormalTok{(}\DataTypeTok{times =} \NormalTok{10L}
  \NormalTok{, }\DataTypeTok{boucle =} \KeywordTok{boucle}\NormalTok{(}\DecValTok{1}\NormalTok{:}\DecValTok{10000}\NormalTok{)}
  \NormalTok{, }\DataTypeTok{Reduce =} \KeywordTok{Reduce}\NormalTok{(}\StringTok{`}\DataTypeTok{+}\StringTok{`}\NormalTok{, }\DecValTok{1}\NormalTok{:}\DecValTok{10000}\NormalTok{, }\DataTypeTok{accumulate =} \OtherTok{TRUE}\NormalTok{)}
  \NormalTok{, }\DataTypeTok{cumsum =} \KeywordTok{cumsum}\NormalTok{(}\DecValTok{1}\NormalTok{:}\DecValTok{10000}\NormalTok{)}
\NormalTok{)}
  \NormalTok{## Unit: microseconds}
  \NormalTok{##    expr       min        lq       mean    median}
  \NormalTok{##  boucle 16152.618 16462.201 19031.2555 17284.143}
  \NormalTok{##  Reduce  5595.064  5964.842  6441.1111  6234.577}
  \NormalTok{##  cumsum    35.750    40.257    61.8559    44.762}
  \NormalTok{##         uq       max neval}
  \NormalTok{##  18458.342 34343.917    10}
  \NormalTok{##   6505.539  8956.778    10}
  \NormalTok{##     54.097   186.937    10}
\end{Highlighting}
\end{Shaded}

\end{frame}

\begin{frame}[fragile]{Utiliser l'opérateur \texttt{{[}} au lieu de
\texttt{ifelse()}}

\small
Lorsqu'on crée une variable en faisant intervenir une condition, il est
fréquent d'utiliser la fonction \texttt{ifelse()} :

\footnotesize

\begin{Shaded}
\begin{Highlighting}[]
\NormalTok{notes <-}\StringTok{ }\KeywordTok{runif}\NormalTok{(}\DataTypeTok{n =} \DecValTok{100000}\NormalTok{, }\DataTypeTok{min =} \DecValTok{0}\NormalTok{, }\DataTypeTok{max =} \DecValTok{20}\NormalTok{)}
\NormalTok{mavar <-}\StringTok{ }\KeywordTok{ifelse}\NormalTok{(notes >=}\StringTok{ }\DecValTok{10}\NormalTok{, }\StringTok{"Reçu"}\NormalTok{, }\StringTok{"Recalé"}\NormalTok{)}
\end{Highlighting}
\end{Shaded}

\pause \small
Il est néanmoins beaucoup plus efficace d'utiliser l'opérateur
\texttt{{[}}.

\footnotesize

\begin{Shaded}
\begin{Highlighting}[]
\KeywordTok{microbenchmark}\NormalTok{(}\DataTypeTok{times =} \NormalTok{10L}
  \NormalTok{, }\DataTypeTok{ifelse =} \KeywordTok{ifelse}\NormalTok{(notes >=}\StringTok{ }\DecValTok{10}\NormalTok{, }\StringTok{"Reçu"}\NormalTok{, }\StringTok{"Recalé"}\NormalTok{)}
  \NormalTok{, }\StringTok{"["} \NormalTok{=}\StringTok{ }\NormalTok{\{}
    \NormalTok{mavar <-}\StringTok{ }\KeywordTok{rep}\NormalTok{(}\StringTok{"Recalé"}\NormalTok{, }\KeywordTok{length}\NormalTok{(notes))}
    \NormalTok{mavar[notes >=}\StringTok{ }\DecValTok{10}\NormalTok{] <-}\StringTok{ "Reçu"}
  \NormalTok{\} }
\NormalTok{)}
  \NormalTok{## Unit: milliseconds}
  \NormalTok{##    expr       min        lq      mean    median}
  \NormalTok{##  ifelse 24.766078 24.992902 25.549918 25.163601}
  \NormalTok{##       [  1.363973  1.386601  1.411959  1.407191}
  \NormalTok{##         uq       max neval}
  \NormalTok{##  25.833693 27.527202    10}
  \NormalTok{##   1.442876  1.480608    10}
\end{Highlighting}
\end{Shaded}

\end{frame}

\begin{frame}[fragile]{Simplifier les données : le type \texttt{factor}}

On utilise souvent des chaînes de caractère pour coder une variable de
nature catégorielle.

Le type \texttt{factor} permet de remplacer chaque valeur distincte par
un entier en sauvegardant la table de correspondance. Il est
\textbf{beaucoup plus léger}.

\pause \footnotesize

\begin{Shaded}
\begin{Highlighting}[]
\CommentTok{# Variable à trois modalités codées en caractères}
\NormalTok{acteu <-}\StringTok{ }\KeywordTok{as.character}\NormalTok{(}\KeywordTok{sample}\NormalTok{(}\DecValTok{1}\NormalTok{:}\DecValTok{3}\NormalTok{, }\DecValTok{120000}\NormalTok{, }\DataTypeTok{replace =} \OtherTok{TRUE}\NormalTok{))}
\KeywordTok{object_size}\NormalTok{(acteu)}
  \NormalTok{## 960 kB}

\CommentTok{# Conversion en facteur}
\NormalTok{f.acteu <-}\StringTok{ }\KeywordTok{as.factor}\NormalTok{(acteu)}
\KeywordTok{str}\NormalTok{(f.acteu)}
  \NormalTok{##  Factor w/ 3 levels "1","2","3": 1 3 2 2 2 2 3 1 1 1 ...}
\KeywordTok{object_size}\NormalTok{(f.acteu)}
  \NormalTok{## 481 kB}
\end{Highlighting}
\end{Shaded}

\end{frame}

\begin{frame}[fragile]{Utiliser les noms à bon escient (1)}

La plupart des objets manipulés couramment dans R peuvent être
\textbf{nommés} : vecteurs, matrices, listes, \texttt{data.frame}.

Utiliser des noms est une méthode souvent \textbf{très rapide} pour
\textbf{accéder aux éléments} qui composent ces objets.

~

\pause 

\textbf{Exemple} On cherche à extraire les observations d'une table
\emph{via} leur identifiant \texttt{id}. On compare l'utilisation des
noms à une fusion réalisée avec \texttt{merge()}.

\small 

\begin{Shaded}
\begin{Highlighting}[]
\CommentTok{# Création de la table df}
\NormalTok{id <-}\StringTok{ }\KeywordTok{as.character}\NormalTok{(}\KeywordTok{sample}\NormalTok{(}\FloatTok{1e5}\NormalTok{))}
\NormalTok{sexe <-}\StringTok{ }\KeywordTok{sample}\NormalTok{(}\DecValTok{1}\NormalTok{:}\DecValTok{2}\NormalTok{, }\FloatTok{1e5}\NormalTok{, }\DataTypeTok{replace =} \OtherTok{TRUE}\NormalTok{)}
\NormalTok{df <-}\StringTok{ }\KeywordTok{data.frame}\NormalTok{(id, sexe)}
\end{Highlighting}
\end{Shaded}

\end{frame}

\begin{frame}[fragile]{Utiliser les noms à bon escient (2)}

\footnotesize

\begin{Shaded}
\begin{Highlighting}[]
\CommentTok{# Affectation de noms à df }
\KeywordTok{row.names}\NormalTok{(df) <-}\StringTok{ }\NormalTok{id}

\CommentTok{# Liste des identifiants à extraire}
\NormalTok{extract <-}\StringTok{ }\KeywordTok{c}\NormalTok{(}\StringTok{"234"}\NormalTok{, }\StringTok{"12"}\NormalTok{, }\StringTok{"7890"}\NormalTok{)}

\CommentTok{# Comparaison}
\KeywordTok{microbenchmark}\NormalTok{(}\DataTypeTok{times =} \NormalTok{10L}
  \NormalTok{, }\DataTypeTok{merge =} \KeywordTok{merge}\NormalTok{(}\KeywordTok{data.frame}\NormalTok{(}\DataTypeTok{id =} \NormalTok{extract), df, }\DataTypeTok{sort =} \OtherTok{FALSE}\NormalTok{)}
  \NormalTok{, }\DataTypeTok{names =} \NormalTok{df[extract, ]}
\NormalTok{)}
  \NormalTok{## Unit: milliseconds}
  \NormalTok{##   expr       min        lq      mean    median}
  \NormalTok{##  merge 13.868964 14.261877 14.975551 14.682713}
  \NormalTok{##  names  2.449722  2.483716  2.711015  2.659243}
  \NormalTok{##         uq       max neval}
  \NormalTok{##  15.148270 18.012145    10}
  \NormalTok{##   2.741494  3.314512    10}
\end{Highlighting}
\end{Shaded}

\end{frame}

\begin{frame}[fragile]{À propos des matrices (1)}

Quand c'est possible, \textbf{travailler sur des matrices} (plutôt que
des \texttt{data.frame}) est souvent source d'efficacité :

\vfill

\begin{itemize}
\tightlist
\item
  \pause de nombreuses opérations sont \textbf{vectorisées} pour les
  matrices : sommes en lignes et en colonnes (\texttt{rowSums()} et
  \texttt{colSums()}), etc. ;
\end{itemize}

\vfill 

\begin{itemize}
\tightlist
\item
  \pause l'\textbf{algèbre matricielle} (le produit matriciel notamment)
  est très bien optimisée ;
\end{itemize}

\vfill

\begin{itemize}
\tightlist
\item
  \pause selon la nature du problème, l'utilisation de \textbf{matrices
  lacunaires} (\emph{sparse}) peut faire gagner et en empreinte mémoire
  et en temps de calcul (\emph{cf.} le \emph{package} \texttt{Matrix}).
\end{itemize}

\end{frame}

\begin{frame}[fragile]{À propos des matrices (2)}

\footnotesize 

\begin{Shaded}
\begin{Highlighting}[]
\CommentTok{# Création d'une matrice m avec 99 % de 0}
\NormalTok{v <-}\StringTok{ }\KeywordTok{rep}\NormalTok{(}\DecValTok{0}\NormalTok{, }\FloatTok{1e6}\NormalTok{); v[}\KeywordTok{sample}\NormalTok{(}\FloatTok{1e6}\NormalTok{, }\FloatTok{1e4}\NormalTok{)] <-}\StringTok{ }\KeywordTok{rnorm}\NormalTok{(}\FloatTok{1e4}\NormalTok{)}
\NormalTok{m <-}\StringTok{ }\KeywordTok{matrix}\NormalTok{(v, }\DataTypeTok{ncol =} \DecValTok{100}\NormalTok{)}

\CommentTok{# Transformation en matrice lacunaire}
\KeywordTok{library}\NormalTok{(Matrix)}
\NormalTok{M <-}\StringTok{ }\KeywordTok{Matrix}\NormalTok{(m)}

\CommentTok{# Gain en espace (en ko)}
\KeywordTok{c}\NormalTok{(}\KeywordTok{object_size}\NormalTok{(m), }\KeywordTok{object_size}\NormalTok{(M))}
  \NormalTok{## [1] 8000200  121800}

\CommentTok{# Gain de performances pour la fonction colSums()}
\KeywordTok{microbenchmark}\NormalTok{(}\DataTypeTok{dense =} \KeywordTok{colSums}\NormalTok{(m), }\DataTypeTok{sparse =} \KeywordTok{colSums}\NormalTok{(M))}
  \NormalTok{## Unit: microseconds}
  \NormalTok{##    expr      min        lq       mean    median}
  \NormalTok{##   dense 1258.396 1279.8925 1402.13968 1319.8680}
  \NormalTok{##  sparse   61.805   75.0895   90.45723   86.9045}
  \NormalTok{##        uq      max neval}
  \NormalTok{##  1368.829 4009.479   100}
  \NormalTok{##    95.571  170.348   100}
\end{Highlighting}
\end{Shaded}

\end{frame}

\subsection{dplyr : une grammaire du traitement des
données}\label{dplyr-une-grammaire-du-traitement-des-donnees}

\begin{frame}[fragile]{Philosophie de \texttt{dplyr}}

\texttt{dplyr} est un \emph{package} développé par RStudio et en
particulier par Hadley Wickham. Il constitue un véritable
\textbf{écosystème} visant à faciliter le travail sur des tables
statistiques :

\begin{itemize}
\tightlist
\item
  \pause il fournit un ensemble de \textbf{fonctions élémentaires} (les
  \og verbes \fg{}) pour effectuer les manipulations de données;
\item
  \pause plusieurs verbes peuvent facilement être \textbf{combinés en
  utilisant l'opérateur \texttt{\%\textgreater{}\%}} (\emph{pipe});
\item
  \pause toutes les opérations sont optimisées par du \textbf{code de
  bas niveau}.
\end{itemize}

\begin{Shaded}
\begin{Highlighting}[]
\KeywordTok{library}\NormalTok{(dplyr)}
\end{Highlighting}
\end{Shaded}

\pause

\textbf{Pour en savoir plus} De nombreuses vignettes très pédagogiques
sont disponibles sur la
\href{https://cran.r-project.org/package=dplyr}{\underline{page du \textit{package}}}.
Un
\href{https://www.rstudio.com/wp-content/uploads/2016/01/data-wrangling-french.pdf}{\underline{aide-mémoire}}
est également disponible sur le site de RStudio.

\end{frame}

\begin{frame}[fragile]{\large Données d'exemple : table \texttt{flights}
de \texttt{nycflights13}}

Les exemples relatifs aux \emph{packages} \texttt{dplyr} et
\texttt{data.table} s'appuient sur les données du \emph{package}
\texttt{nycflights13}.

\begin{Shaded}
\begin{Highlighting}[]
\KeywordTok{library}\NormalTok{(nycflights13)}
\end{Highlighting}
\end{Shaded}

\pause Ce \emph{package} contient des données sur tous les vols au
départ de la ville de New-York en 2013.

\small

\begin{Shaded}
\begin{Highlighting}[]
\KeywordTok{data}\NormalTok{(}\DataTypeTok{package =} \StringTok{"nycflights13"}\NormalTok{)}
\KeywordTok{dim}\NormalTok{(flights)}
  \NormalTok{## [1] 336776     19}
\KeywordTok{names}\NormalTok{(flights)[}\DecValTok{1}\NormalTok{:}\DecValTok{9}\NormalTok{]}
  \NormalTok{## [1] "year"           "month"          "day"           }
  \NormalTok{## [4] "dep_time"       "sched_dep_time" "dep_delay"     }
  \NormalTok{## [7] "arr_time"       "sched_arr_time" "arr_delay"}
\end{Highlighting}
\end{Shaded}

\end{frame}

\begin{frame}[fragile]{Simplifier des opérations de base R}

\texttt{dplyr} propose plusieurs verbes pour simplifier certaines
opérations parfois fastidieuses en base R :

\pause \vspace{-1mm} - \texttt{filter()} sélectionne des observations
selon une ou plusieurs conditions;

\vspace{-3mm}

\begin{Shaded}
\begin{Highlighting}[]
\KeywordTok{filter}\NormalTok{(flights, month ==}\StringTok{ }\DecValTok{7}\NormalTok{, day ==}\StringTok{ }\DecValTok{4}\NormalTok{)}
\end{Highlighting}
\end{Shaded}

\pause \vspace{-3mm} - \texttt{arrange()} trie le fichier selon une ou
plusieurs variables;

\vspace{-3mm}

\begin{Shaded}
\begin{Highlighting}[]
\KeywordTok{arrange}\NormalTok{(flights, month, }\KeywordTok{desc}\NormalTok{(distance))}
\end{Highlighting}
\end{Shaded}

\pause \vspace{-3mm} - \texttt{select()} sélectionne des variables par
leur noms;

\vspace{-3mm}

\begin{Shaded}
\begin{Highlighting}[]
\KeywordTok{select}\NormalTok{(flights, year:arr_delay)}
\end{Highlighting}
\end{Shaded}

\pause \vspace{-3mm} - \texttt{rename()} renomme des variables.

\vspace{-3mm}

\begin{Shaded}
\begin{Highlighting}[]
\KeywordTok{rename}\NormalTok{(flights, }\DataTypeTok{annee =} \NormalTok{year)}
\end{Highlighting}
\end{Shaded}

\end{frame}

\begin{frame}[fragile]{Calculer des statistiques avec
\texttt{summarise()}}

La fonction \texttt{summarise()} permet de facilement calculer des
statistiques sur des données.

\pause 

\begin{Shaded}
\begin{Highlighting}[]
\KeywordTok{summarise}\NormalTok{(flights}
  \NormalTok{, }\DataTypeTok{distance_moyenne =} \KeywordTok{mean}\NormalTok{(distance)}
  \NormalTok{, }\DataTypeTok{retard_max =} \KeywordTok{max}\NormalTok{(arr_delay, }\DataTypeTok{na.rm =} \OtherTok{TRUE}\NormalTok{)}
\NormalTok{)}
\end{Highlighting}
\end{Shaded}

\begin{verbatim}
  ##   distance_moyenne retard_max
  ## 1         1039.913       1272
\end{verbatim}

\pause 

\textbf{Remarque} Comme toutes les fonctions de \texttt{dplyr},
\texttt{summarise()} prend un \texttt{data.frame} en entrée et produit
un \texttt{data.frame} en sortie.

\end{frame}

\begin{frame}[fragile]{Ventiler des traitements avec
\texttt{group\_by()}}

Appliqué au préalable à un \texttt{data.frame}, \texttt{group\_by()}
ventile tous les traitements ultérieurs selon les modalités d'une ou
plusieurs variables.

\begin{Shaded}
\begin{Highlighting}[]
\NormalTok{flights_bymonth <-}\StringTok{ }\KeywordTok{group_by}\NormalTok{(flights, month)}
\KeywordTok{summarise}\NormalTok{(flights_bymonth}
  \NormalTok{, }\DataTypeTok{distance_moyenne =} \KeywordTok{mean}\NormalTok{(distance)}
  \NormalTok{, }\DataTypeTok{retard_max =} \KeywordTok{max}\NormalTok{(arr_delay, }\DataTypeTok{na.rm =} \OtherTok{TRUE}\NormalTok{)}
\NormalTok{)[}\DecValTok{1}\NormalTok{:}\DecValTok{3}\NormalTok{, ]}
\end{Highlighting}
\end{Shaded}

\begin{verbatim}
  ##   month distance_moyenne retard_max
  ## 1     1         1006.844       1272
  ## 2     2         1000.982        834
  ## 3     3         1011.987        915
\end{verbatim}

\end{frame}

\begin{frame}[fragile]{Enchaîner des opérations avec
\texttt{\%\textgreater{}\%}}

L'utilisation des verbes de \texttt{dplyr} ne prend tout son intérêt que
quand ils sont enchaînés en utilisant l'opérateur \emph{pipe}
\texttt{\%\textgreater{}\%}.

\texttt{maTable\ \%\textgreater{}\%\ maFonction(param1,\ param2)} est
équivalent à \texttt{maFonction(maTable,\ param1,\ param2)}.

\pause Ainsi, l'\textbf{enchaînement de nombreuses opérations} devient
beaucoup plus facile à mettre en \oe uvre et à comprendre.

\pause \footnotesize

\begin{Shaded}
\begin{Highlighting}[]
\NormalTok{flights %>%}
\StringTok{  }\KeywordTok{group_by}\NormalTok{(year, month, day) %>%}
\StringTok{  }\KeywordTok{summarise}\NormalTok{(}
    \DataTypeTok{arr =} \KeywordTok{mean}\NormalTok{(arr_delay, }\DataTypeTok{na.rm =} \OtherTok{TRUE}\NormalTok{),}
    \DataTypeTok{dep =} \KeywordTok{mean}\NormalTok{(dep_delay, }\DataTypeTok{na.rm =} \OtherTok{TRUE}\NormalTok{)}
  \NormalTok{) %>%}
\StringTok{  }\KeywordTok{filter}\NormalTok{(arr >}\StringTok{ }\DecValTok{30} \NormalTok{|}\StringTok{ }\NormalTok{dep >}\StringTok{ }\DecValTok{30}\NormalTok{)}
\end{Highlighting}
\end{Shaded}

\end{frame}

\begin{frame}[fragile]{Fusionner des tables avec \texttt{*\_join()}}

\texttt{dplyr} dispose de nombreuses fonctions très utiles pour
fusionner une ou plusieurs tables ensemble, qui \textbf{s'inspirent très
fortement de SQL} :

\begin{itemize}
\tightlist
\item
  \texttt{a\ \%\textgreater{}\%\ left\_join(b,\ by\ =\ "id")} : fusionne
  \texttt{a} et \texttt{b} en conservant toutes les observations de
  \texttt{a};
\item
  \texttt{a\ \%\textgreater{}\%\ right\_join(b,\ by\ =\ "id")} :
  fusionne \texttt{a} et \texttt{b} en conservant toutes les
  observations de \texttt{b};
\item
  \texttt{a\ \%\textgreater{}\%\ inner\_join(b,\ by\ =\ "id")} :
  fusionne \texttt{a} et \texttt{b} en ne conservant que les
  observations dans \texttt{a} et \texttt{b};
\item
  \texttt{a\ \%\textgreater{}\%\ full\_join(b,\ by\ =\ "id")} : fusionne
  \texttt{a} et \texttt{b} en conservant toutes les observations.
\end{itemize}

\textbf{Pour en savoir plus} Une
\href{https://cran.r-project.org/web/packages/dplyr/vignettes/two-table.html}{\underline{vignette}}
est consacrée à la présentation des fonctions de \texttt{dplyr} portant
sur deux tables.

\end{frame}

\begin{frame}[fragile]{Comparaison de base R et de \texttt{dplyr}}

\texttt{dplyr} est particulièrement intéressant pour travailler sur des
données par groupe. On compare donc l'utilisation de \texttt{tapply()}
de base R avec \texttt{group\_by()} de \texttt{dplyr}.

\footnotesize

\begin{Shaded}
\begin{Highlighting}[]
\NormalTok{df <-}\StringTok{ }\KeywordTok{data.frame}\NormalTok{(}
  \DataTypeTok{x =} \KeywordTok{rnorm}\NormalTok{(}\FloatTok{1e6}\NormalTok{)}
  \NormalTok{, }\DataTypeTok{by =} \KeywordTok{sample}\NormalTok{(}\FloatTok{1e3}\NormalTok{, }\FloatTok{1e6}\NormalTok{, }\DataTypeTok{replace =} \OtherTok{TRUE}\NormalTok{)}
\NormalTok{)}

\KeywordTok{microbenchmark}\NormalTok{(}\DataTypeTok{times =} \NormalTok{10L}
  \NormalTok{, }\DataTypeTok{base =} \KeywordTok{tapply}\NormalTok{(df$x, df$by, sum)}
  \NormalTok{, }\DataTypeTok{dplyr =} \NormalTok{df %>%}\StringTok{ }\KeywordTok{group_by}\NormalTok{(by) %>%}\StringTok{ }\KeywordTok{summarise}\NormalTok{(}\KeywordTok{sum}\NormalTok{(x))}
\NormalTok{)}
  \NormalTok{## Unit: milliseconds}
  \NormalTok{##   expr      min       lq     mean   median       uq}
  \NormalTok{##   base 37.64261 38.03618 41.82012 41.88673 45.41735}
  \NormalTok{##  dplyr 46.90118 47.41262 47.82028 47.92937 48.27681}
  \NormalTok{##       max neval}
  \NormalTok{##  46.70003    10}
  \NormalTok{##  48.47417    10}
\end{Highlighting}
\end{Shaded}

\end{frame}

\subsection{data.table : un data.frame
optimisé}\label{data.table-un-data.frame-optimise}

\begin{frame}[fragile]{Philosophie de \texttt{data.table}}

Contrairement à \texttt{dplyr}, \texttt{data.table} ne cherche pas à se
substituer à base R mais à le compléter.

Il introduit un nouveau type d'objet, le \texttt{data.table}, qui
\textbf{hérite} du \texttt{data.frame} (tout \texttt{data.table} est un
\texttt{data.frame}).

Appliqué à un \texttt{data.table}, l'opérateur \texttt{{[}} est
\textbf{enrichi et optimisé}.

\begin{Shaded}
\begin{Highlighting}[]
\KeywordTok{library}\NormalTok{(data.table)}
\NormalTok{flights_DT <-}\StringTok{ }\KeywordTok{data.table}\NormalTok{(flights)}
\end{Highlighting}
\end{Shaded}

\textbf{Pour en savoir plus} Là encore des vignettes très pédagogiques
sont disponibles sur la
\href{https://cran.r-project.org/package=data.table}{\underline{page du \textit{package}}}.

\end{frame}

\begin{frame}[fragile]{L'opérateur \texttt{{[}} du \texttt{data.table} :
\texttt{i}, \texttt{j} et \texttt{by}}

La syntaxe de l'opérateur \texttt{{[}} appliqué à un \texttt{data.table}
est la suivante (\texttt{DT} représente le \texttt{data.table}):

\centering \large

\texttt{DT{[}i,\ j,\ by{]}}

\raggedright \normalsize

\begin{itemize}
\tightlist
\item
  \texttt{i} : sélectionner des observations selon une condition;
\item
  \texttt{j} : sélectionner ou \textbf{créer} une ou plusieurs
  variables;
\item
  \texttt{by} : ventiler les traitements selon les modalités d'une ou
  plusieurs variables.
\end{itemize}

\bigskip 

\textbf{Exemple} Retard quotidien maximal au mois de janvier.

\small

\begin{Shaded}
\begin{Highlighting}[]
\NormalTok{flights_DT[}
  \NormalTok{month ==}\StringTok{ }\DecValTok{1}\NormalTok{, }\KeywordTok{max}\NormalTok{(arr_delay, }\DataTypeTok{na.rm =} \OtherTok{TRUE}\NormalTok{), by =}\StringTok{ }\NormalTok{day}
\NormalTok{]}
\end{Highlighting}
\end{Shaded}

\end{frame}

\begin{frame}[fragile]{Sélectionner des observations avec \texttt{i}}

\small
Il est beaucoup plus simple et efficace de sélectionner des observations
dans un \texttt{data.table} que dans un \texttt{data.frame}:

\begin{itemize}
\item
  il n'y a pas à répéter le nom du \texttt{data.frame} dans
  \texttt{{[}};
\item
  il est possible d'indexer un \texttt{data.table} par une ou plusieurs
  \og clés \fg{} permettant une recherche souvent plus rapide.
\end{itemize}

\footnotesize

\begin{Shaded}
\begin{Highlighting}[]
\KeywordTok{setkey}\NormalTok{(flights_DT, origin)}
\KeywordTok{microbenchmark}\NormalTok{(}\DataTypeTok{times =} \NormalTok{100L}
  \NormalTok{, }\DataTypeTok{base =} \NormalTok{flights[flights$origin ==}\StringTok{ "JFK"}\NormalTok{,]}
  \NormalTok{, }\DataTypeTok{dt1 =} \NormalTok{flights_DT[origin ==}\StringTok{ "JFK"}\NormalTok{]}
  \NormalTok{, }\DataTypeTok{dt2 =} \NormalTok{flights_DT[}\KeywordTok{list}\NormalTok{(}\StringTok{"JFK"}\NormalTok{)]}
\NormalTok{)}
  \NormalTok{## Unit: milliseconds}
  \NormalTok{##  expr      min       lq     mean   median       uq}
  \NormalTok{##  base 39.93500 41.02064 47.52981 47.84936 48.94594}
  \NormalTok{##   dt1 11.18588 11.38343 15.11531 11.62197 14.05389}
  \NormalTok{##   dt2 10.61155 10.92853 11.95295 11.03300 11.53576}
  \NormalTok{##        max neval}
  \NormalTok{##  204.32653   100}
  \NormalTok{##  183.28408   100}
  \NormalTok{##   21.06783   100}
\end{Highlighting}
\end{Shaded}

\end{frame}

\begin{frame}[fragile]{Calculer des statistiques avec \texttt{j}}

L'argument \texttt{j} permet de calculer des statistiques agrégées.

\small

\begin{Shaded}
\begin{Highlighting}[]
\NormalTok{flights_DT[, j =}\StringTok{ }\KeywordTok{list}\NormalTok{(}
  \DataTypeTok{distance_moyenne =} \KeywordTok{mean}\NormalTok{(distance)}
  \NormalTok{, }\DataTypeTok{retard_max =} \KeywordTok{max}\NormalTok{(arr_delay, }\DataTypeTok{na.rm =} \OtherTok{TRUE}\NormalTok{)}
\NormalTok{)]}
  \NormalTok{##    distance_moyenne retard_max}
  \NormalTok{## 1:         1039.913       1272}
\end{Highlighting}
\end{Shaded}

\normalsize 

Utilisé avec \texttt{:=} il permet de les refusionner automatiquement
avec les données d'origine.

\small

\begin{Shaded}
\begin{Highlighting}[]
\NormalTok{flights_DT <-}\StringTok{ }\NormalTok{flights_DT[, j :}\ErrorTok{=}\StringTok{ }\KeywordTok{list}\NormalTok{(}
  \DataTypeTok{distance_moyenne =} \KeywordTok{mean}\NormalTok{(distance)}
  \NormalTok{, }\DataTypeTok{retard_max =} \KeywordTok{max}\NormalTok{(arr_delay, }\DataTypeTok{na.rm =} \OtherTok{TRUE}\NormalTok{)}
\NormalTok{)]}
\end{Highlighting}
\end{Shaded}

\end{frame}

\begin{frame}[fragile]{Ventiler des traitements avec \texttt{by} et
\texttt{keyby}}

L'argument \texttt{by} de \texttt{{[}} ventile tous les traitements
renseignés dans \texttt{j} selon les modalités d'une ou plusieurs
variables.

\begin{Shaded}
\begin{Highlighting}[]
\NormalTok{flights_DT[, j =}\StringTok{ }\KeywordTok{list}\NormalTok{(}
  \DataTypeTok{distance_moyenne =} \KeywordTok{mean}\NormalTok{(distance)}
  \NormalTok{, }\DataTypeTok{retard_max =} \KeywordTok{max}\NormalTok{(arr_delay, }\DataTypeTok{na.rm =} \OtherTok{TRUE}\NormalTok{)}
\NormalTok{), by =}\StringTok{ }\NormalTok{month][}\DecValTok{1}\NormalTok{:}\DecValTok{3}\NormalTok{,]}
  \NormalTok{##    month distance_moyenne retard_max}
  \NormalTok{## 1:     1         1006.844       1272}
  \NormalTok{## 2:    10         1038.876        688}
  \NormalTok{## 3:    11         1050.305        796}
\end{Highlighting}
\end{Shaded}

\textbf{Remarque} Par défaut, \texttt{by} ordonne les résultats dans
l'ordre des groupes dans le \texttt{data.table}. \texttt{keyby} trie les
données selon la variable d'agrégation (comme \texttt{group\_by} de
\texttt{dplyr}).

\end{frame}

\begin{frame}[fragile]{Chaîner les opérations dans un
\texttt{data.table}}

Il est très facile de chaîner les opérations sur un \texttt{data.table}
en enchaînant les \texttt{{[}}.

\begin{Shaded}
\begin{Highlighting}[]
\NormalTok{flights_DT[}
  \NormalTok{, j =}\StringTok{ }\KeywordTok{list}\NormalTok{(}
    \DataTypeTok{arr =} \KeywordTok{mean}\NormalTok{(arr_delay, }\DataTypeTok{na.rm =} \OtherTok{TRUE}\NormalTok{)}
    \NormalTok{, }\DataTypeTok{dep =} \KeywordTok{mean}\NormalTok{(dep_delay, }\DataTypeTok{na.rm =} \OtherTok{TRUE}\NormalTok{)}
  \NormalTok{)}
  \NormalTok{, keyby =}\StringTok{ }\KeywordTok{list}\NormalTok{(year, month, day)}
\NormalTok{][arr >}\StringTok{ }\DecValTok{30} \NormalTok{|}\StringTok{ }\NormalTok{dep >}\StringTok{ }\DecValTok{30}\NormalTok{]}
\end{Highlighting}
\end{Shaded}

\textbf{Remarque} Ces chaînages sont possibles avec un
\texttt{data.table} mais pas avec un \texttt{data.frame}.

\end{frame}

\begin{frame}[fragile]{Comparaison de base R, \texttt{dplyr} et
\texttt{data.table}}

\footnotesize

\begin{Shaded}
\begin{Highlighting}[]
\CommentTok{# Conversion de la table de test en data.table}
\NormalTok{dt <-}\StringTok{ }\KeywordTok{data.table}\NormalTok{(df)}

\KeywordTok{microbenchmark}\NormalTok{(}\DataTypeTok{times =} \NormalTok{10L}
  \NormalTok{, }\DataTypeTok{base =} \KeywordTok{tapply}\NormalTok{(df$x, df$by, sum)}
  \NormalTok{, }\DataTypeTok{dplyr =} \NormalTok{df %>%}\StringTok{ }\KeywordTok{group_by}\NormalTok{(by) %>%}\StringTok{ }\KeywordTok{summarise}\NormalTok{(}\KeywordTok{sum}\NormalTok{(x))}
  \NormalTok{, }\DataTypeTok{data.table =} \NormalTok{dt[, }\KeywordTok{sum}\NormalTok{(x), }\DataTypeTok{keyby =} \NormalTok{by]}
\NormalTok{)}
\end{Highlighting}
\end{Shaded}

\vspace{-5mm}

\begin{verbatim}
  ##         expr       lq     mean       uq
  ## 1       base 37.98608 43.61748 45.62521
  ## 2      dplyr 46.59135 47.12917 47.49720
  ## 3 data.table 21.66920 23.70292 23.84025
\end{verbatim}

\normalsize

\textbf{Pour en savoir plus} Cette discussion sur
\href{http://stackoverflow.com/questions/21435339/data-table-vs-dplyr-can-one-do-something-well-the-other-cant-or-does-poorly}{\underline{stackoverflow.com}}
(notamment entre les auteurs des \emph{packages}) aborde les avantages
et les inconvénients de \texttt{dplyr} et \texttt{data.table}.

\end{frame}

\subsection{Aller plus loin avec R}\label{aller-plus-loin-avec-r}

\begin{frame}{Les limites du logiciel}

Les outils présentés jusqu'à présent correspondent à une utilisation
\og classique \fg{} de R : production d'une enquête, redressements,
études.

~

Il arrive néanmoins que certains traitements soient rendus
\textbf{difficiles par les caractéristiques du logiciel} :

\begin{itemize}
\tightlist
\item
  travail sur des volumes de données impossibles à loger en mémoire;
\item
  temps de calcul trop longs et impossibles à réduire.
\end{itemize}

~

Dans ce genre de situations, la solution consiste en général à utiliser
R comme une \textbf{interface} vers des techniques ou langages
susceptibles de répondre au problème posé.

\end{frame}

\begin{frame}[fragile]{\large Travailler sur des données
\emph{out-of-memory}}

Les \emph{packages} \texttt{ff} et \texttt{ffbase} permettent de
travailler sur des objets directement stockés sur le disque dur de
l'ordinateur.

\footnotesize

\begin{Shaded}
\begin{Highlighting}[]
\KeywordTok{library}\NormalTok{(ff)}
\KeywordTok{library}\NormalTok{(ffbase)}

\CommentTok{# Lecture d'un fichier .csv important (RP)}
\NormalTok{ffdf <-}\StringTok{ }\KeywordTok{read.csv2.ffdf}\NormalTok{(}
  \DataTypeTok{file =} \StringTok{"FD_INDREGZA_2013.txt"}
  \NormalTok{, }\DataTypeTok{VERBOSE =} \OtherTok{TRUE}
\NormalTok{)}

\CommentTok{# Calculs simples sur l'objet ffdf}
\KeywordTok{table}\NormalTok{(ffdf$REGION)}
\end{Highlighting}
\end{Shaded}

\normalsize

\textbf{Remarque} Ces \emph{packages} n'ont pas connu d'évolutions
depuis plusieurs années.

\end{frame}

\begin{frame}[fragile]{\large Se connecter à des bases de données}

Une autre solution pour exploiter de grands volumes de données dans R
est de l'utiliser pour \textbf{interroger des bases de données},
\emph{via} par exemple le \emph{package} \texttt{RPostgreSQL}.

\footnotesize

\begin{Shaded}
\begin{Highlighting}[]
\KeywordTok{library}\NormalTok{(RPostgreSQL)}

\CommentTok{# Connexion à la base de données maBdd}
\NormalTok{drv <-}\StringTok{ }\KeywordTok{dbDriver}\NormalTok{(}\StringTok{"PostgreSQL"}\NormalTok{)}
\NormalTok{con <-}\StringTok{ }\KeywordTok{dbConnect}\NormalTok{(drv, }\DataTypeTok{dbname =} \StringTok{"maBdd"}
  \NormalTok{, }\DataTypeTok{host =} \StringTok{"localhost"}\NormalTok{, }\DataTypeTok{port =} \DecValTok{5432}
  \NormalTok{, }\DataTypeTok{user =} \StringTok{"utilisateur"}\NormalTok{, }\DataTypeTok{password =} \StringTok{"motDePasse"}
\NormalTok{)}

\CommentTok{# Requête SQL sur la table maTable}
\KeywordTok{dbGetQuery}\NormalTok{(con, }\StringTok{"SELECT COUNT(*) FROM maTable"}\NormalTok{)}
\end{Highlighting}
\end{Shaded}

\normalsize

\textbf{Remarque} Différents \emph{packages} permettent de se connecter
à différents types de base de données : \texttt{RMySQl} pour MySQL, etc.

\end{frame}

\begin{frame}[fragile]{\large Se connecter à des bases de données avec
\texttt{dplyr}}

\texttt{dplyr} a la particularité de pouvoir fonctionner de façon
totalement transparente sur des bases de données de différents types.

\footnotesize

\begin{Shaded}
\begin{Highlighting}[]
\KeywordTok{library}\NormalTok{(dplyr)}

\CommentTok{# Connexion à la base de données maBdd}
\NormalTok{con <-}\StringTok{ }\KeywordTok{src_postgres}\NormalTok{(}
  \DataTypeTok{dbname =} \StringTok{"maBdd"}\NormalTok{, }\DataTypeTok{host =} \StringTok{"localhost"}\NormalTok{, }\DataTypeTok{port =} \DecValTok{5432}
  \NormalTok{, }\DataTypeTok{user =} \StringTok{"utilisateur"}\NormalTok{, }\DataTypeTok{password =} \StringTok{"motDePasse"}
\NormalTok{)}

\CommentTok{# Requête SQL sur la table maTable...}
\KeywordTok{tbl}\NormalTok{(con, }\StringTok{"SELECT COUNT(*) FROM maTable"}\NormalTok{)}

\CommentTok{# ... ou utilisation des verbes de dplyr}
\KeywordTok{tbl}\NormalTok{(con) %>%}\StringTok{ }\KeywordTok{summarise}\NormalTok{(}\KeywordTok{n}\NormalTok{())}
\end{Highlighting}
\end{Shaded}

\end{frame}

\begin{frame}[fragile]{\large Paralléliser des traitements avec
\texttt{parallel} (1)}

La plupart des ordinateurs possèdent aujourd'hui plusieurs c\oe urs
(\emph{core}) susceptibles de mener des traitements \textbf{en
parallèle} (8 sur chaque serveur d'AUS par exemple).

Par défaut, R n'expoite qu'un seul c\oe ur : le \emph{package}
\texttt{parallel} (mais aussi les \emph{packages} \texttt{snow} ou
\texttt{foreach} par exemple) permettent de \textbf{paralléliser des
structures du type \texttt{*apply}}.

Ce type d'opérations est composé de plusieurs étapes :

\begin{enumerate}
\def\labelenumi{\arabic{enumi}.}
\tightlist
\item
  Création et paramétrage du \og \textit{cluster} \fg{} de c\oe urs à
  utiliser (chargement des fonctions et \emph{packages} nécessaires sur
  chaque c\oe ur);
\item
  Lancement du traitement parallélisé avec \texttt{parLapply()};
\item
  Arrêt des processus du \emph{cluster} avec \texttt{stopCluster()}.
\end{enumerate}

\end{frame}

\begin{frame}[fragile]{\large Paralléliser des traitements avec
\texttt{parallel} (2)}

Dans cet exemple, on cherche à appliquer la fonction \texttt{f} à chaque
matrice de la liste \texttt{l}.

\footnotesize

\begin{Shaded}
\begin{Highlighting}[]
\KeywordTok{library}\NormalTok{(MASS)}
\NormalTok{f <-}\StringTok{ }\NormalTok{function(x) }\KeywordTok{rowSums}\NormalTok{(}\KeywordTok{ginv}\NormalTok{(x))}
\NormalTok{l <-}\StringTok{ }\KeywordTok{lapply}\NormalTok{(}\DecValTok{1}\NormalTok{:}\DecValTok{100}\NormalTok{, function(x) }\KeywordTok{matrix}\NormalTok{(}\KeywordTok{runif}\NormalTok{(}\FloatTok{1e4}\NormalTok{), }\DataTypeTok{ncol =} \FloatTok{1e2}\NormalTok{))}

\CommentTok{# Création et paramétrage du cluster}
\KeywordTok{library}\NormalTok{(parallel)}
\NormalTok{cl <-}\StringTok{ }\KeywordTok{makeCluster}\NormalTok{(}\DecValTok{4}\NormalTok{)}
\KeywordTok{clusterEvalQ}\NormalTok{(cl, }\KeywordTok{library}\NormalTok{(MASS))}
\KeywordTok{clusterExport}\NormalTok{(cl, }\StringTok{"f"}\NormalTok{)}

\CommentTok{# Lancement du calcul parallélisé}
\KeywordTok{parLapply}\NormalTok{(cl, l, f)}

\CommentTok{# Arrêt des processus du cluster}
\KeywordTok{stopCluster}\NormalTok{(cl)}
\end{Highlighting}
\end{Shaded}

\end{frame}

\begin{frame}[fragile]{\large Paralléliser des traitements avec
\texttt{parallel} (3)}

\begin{Shaded}
\begin{Highlighting}[]
\KeywordTok{microbenchmark}\NormalTok{(}\DataTypeTok{times =} \DecValTok{10}
  \NormalTok{, }\KeywordTok{lapply}\NormalTok{(l, f)}
  \NormalTok{, }\KeywordTok{parLapply}\NormalTok{(cl, l, f)}
\NormalTok{)}
  \NormalTok{## Unit: milliseconds}
  \NormalTok{##                 expr      min       lq     mean}
  \NormalTok{##         lapply(l, f) 653.5663 667.8096 678.4173}
  \NormalTok{##  parLapply(cl, l, f) 347.7800 351.0758 411.1644}
  \NormalTok{##    median       uq      max neval}
  \NormalTok{##  672.4361 693.3183 710.1756    10}
  \NormalTok{##  392.6949 454.0194 550.1971    10}
\end{Highlighting}
\end{Shaded}

\end{frame}

\begin{frame}[fragile]{\texttt{Rcpp} : un package R pour utiliser C++
(1)}

Le \emph{package} \texttt{Rcpp} permet d'intégrer facilement des
fonctions codées en C++ dans un programme R.

\footnotesize

\begin{Shaded}
\begin{Highlighting}[]
\KeywordTok{library}\NormalTok{(Rcpp)}
\KeywordTok{cppFunction}\NormalTok{(}\StringTok{'int add(int x, int y) \{}
\StringTok{  int result = x + y;}
\StringTok{  return result;}
\StringTok{\}'}\NormalTok{)}

\KeywordTok{add}\NormalTok{(}\DecValTok{1}\NormalTok{, }\DecValTok{2}\NormalTok{)}
  \NormalTok{## [1] 3}
\end{Highlighting}
\end{Shaded}

\normalsize

\textbf{Remarque} Il est également possible de soumettre un fichier
contenant des fonctions C++ écrit par ailleurs à l'aide de la fonction
\texttt{sourceCpp()}.

\textbf{Pour en savoir plus}
\href{http://adv-r.had.co.nz/Rcpp.html}{\underline{\textit{Advanced R}}}

\end{frame}

\begin{frame}[fragile]{\texttt{Rcpp} : un package R pour utiliser C++
(2)}

Contrairement à R, C++ est un langage de bas niveau : les boucles y sont
en particulier extrêmement rapides.

\textbf{Exemple} Somme cumulée par colonne

\footnotesize

\begin{Shaded}
\begin{Highlighting}[]
\CommentTok{# Fonction C++}
\KeywordTok{cppFunction}\NormalTok{(}\StringTok{'NumericMatrix cumColSumsC(NumericMatrix x) \{}
\StringTok{  int nrow = x.nrow(), ncol = x.ncol();}
\StringTok{  NumericMatrix out(nrow, ncol);}
\StringTok{  for (int j = 0; j < ncol; j++) \{}
\StringTok{    double acc = 0;}
\StringTok{    for(int i = 0; i < nrow; i++)\{}
\StringTok{      acc += x(i, j);}
\StringTok{      out(i, j) = acc;}
\StringTok{    \}}
\StringTok{  \}}
\StringTok{  return out;}
\StringTok{\}'}\NormalTok{)}
\end{Highlighting}
\end{Shaded}

\end{frame}

\begin{frame}[fragile]{\texttt{Rcpp} : un package R pour utiliser C++
(3)}

\footnotesize

\begin{Shaded}
\begin{Highlighting}[]
\CommentTok{# Fonction R}
\NormalTok{cumColSumsR <-}\StringTok{ }\NormalTok{function(x)\{}
  \KeywordTok{apply}\NormalTok{(x, }\DecValTok{2}\NormalTok{, cumsum)}
\NormalTok{\}}

\CommentTok{# Les deux fonctions produisent les mêmes résultats...}
\NormalTok{x <-}\StringTok{ }\KeywordTok{matrix}\NormalTok{(}\KeywordTok{rnorm}\NormalTok{(}\FloatTok{1e6}\NormalTok{), }\DataTypeTok{ncol =} \FloatTok{1e2}\NormalTok{)}
\KeywordTok{all.equal}\NormalTok{(}\KeywordTok{cumColSumsR}\NormalTok{(x), }\KeywordTok{cumColSumsC}\NormalTok{(x))}
  \NormalTok{## [1] TRUE}

\CommentTok{# ... mais cumColSumsC() est beaucoup plus rapide !}
\KeywordTok{summary}\NormalTok{(}\KeywordTok{microbenchmark}\NormalTok{(}\DataTypeTok{times =} \DecValTok{10}
  \NormalTok{, }\KeywordTok{cumColSumsR}\NormalTok{(x)}
  \NormalTok{, }\KeywordTok{cumColSumsC}\NormalTok{(x)}
\NormalTok{))[, }\KeywordTok{c}\NormalTok{(}\StringTok{"expr"}\NormalTok{, }\StringTok{"lq"}\NormalTok{, }\StringTok{"mean"}\NormalTok{, }\StringTok{"uq"}\NormalTok{)]}
  \NormalTok{##             expr        lq      mean        uq}
  \NormalTok{## 1 cumColSumsR(x) 16.420141 19.201384 21.674531}
  \NormalTok{## 2 cumColSumsC(x)  4.177008  4.724542  4.279235}
\end{Highlighting}
\end{Shaded}

\end{frame}

\section{Réaliser des graphiques avec
R}\label{realiser-des-graphiques-avec-r}

\subsection*{Réaliser des graphiques avec R}

\begin{frame}{R et la réalisation de graphiques}

La réalisation de graphiques dans un logiciel statistique est une
opération souvent longue et complexe.

Dans la plupart des cas, l'ajustement fin des paramètres par le biais de
lignes de code relève de la gageure.

\pause R dispose néanmoins de plusieurs caractéristiques qui facilitent
la réalisation de graphiques :

\begin{itemize}
\tightlist
\item
  \textbf{souplesse} : la très grande variété des types d'objets
  simplifie les paramétrages ;
\item
  \textbf{rigueur} : la dimension fonctionnelle du langage aide à
  systématiser l'utilisation des paramètres graphiques ;
\item
  \textbf{adaptabilité} : la liberté de développement de modules
  complémentaires rend possible de profondes innovations dans la
  conception des graphiques.
\end{itemize}

\end{frame}

\begin{frame}[fragile]{Base R ou \texttt{ggplot2} ?}

Il existe aujourd'hui troix principaux paradigmes pour produire des
graphiques avec R :

\begin{itemize}
\item
  les fonctionnalités de base du logiciel du \emph{package}
  \texttt{graphics};
\item
  les fonctionnalités plus élaborées des \emph{packages} \texttt{grid}
  et \texttt{lattice} (non-abordées dans cette formation);
\item
  la \og grammaire des graphiques \fg{} du \emph{package}
  \texttt{ggplot2}.
\end{itemize}

\bigskip 

\underline{Plan de la partie}

\bigskip

\tableofcontents[currentsection, sectionstyle = hide, subsectionstyle = show/show/hide]

\end{frame}

\begin{frame}[fragile]{Données d'exemple : table \texttt{mpg} de
\texttt{ggplot2}}

\small

La plupart des exemples de cette partie sont produits à partir de la
table \texttt{mpg} du \emph{package} \texttt{ggplot2}.

\footnotesize

\begin{Shaded}
\begin{Highlighting}[]
\KeywordTok{library}\NormalTok{(ggplot2)}
\KeywordTok{dim}\NormalTok{(mpg)}
  \NormalTok{## [1] 234  11}
\KeywordTok{names}\NormalTok{(mpg)}
  \NormalTok{##  [1] "manufacturer" "model"        "displ"       }
  \NormalTok{##  [4] "year"         "cyl"          "trans"       }
  \NormalTok{##  [7] "drv"          "cty"          "hwy"         }
  \NormalTok{## [10] "fl"           "class"}
\end{Highlighting}
\end{Shaded}

\pause \vspace{-0.4cm} \small

\begin{itemize}
\tightlist
\item
  \texttt{displ} : cylindrée;
\item
  \texttt{drv} : transmission (\texttt{f} traction, \texttt{r}
  propulsion, \texttt{4} quatre roues motrices);
\item
  \texttt{cty} et \texttt{hwy} : nombre de \emph{miles} parcourus par
  \emph{gallon} d'essence en ville et sur autoroute respectivement.
\end{itemize}

\end{frame}

\subsection{\texorpdfstring{Réaliser des graphiques avec
\texttt{graphics}}{Réaliser des graphiques avec graphics}}\label{realiser-des-graphiques-avec-graphics}

\begin{frame}[fragile]{\large Beaucoup de fonctions, des paramètres
communs}

La création de graphiques avec le \emph{package} de base
\texttt{graphics} s'appuie sur la \textbf{fonction \texttt{plot()}}
ainsi que sur des \textbf{fonctions spécifiques} :

\begin{itemize}
\tightlist
\item
  \texttt{plot(hist(x))}, \texttt{plot(density(x))} : histogrammes et
  densités;
\item
  \texttt{plot(ts)} : représentation de séries chronologiques;
\item
  \texttt{plot(x,\ y)} : nuages de points;
\item
  \texttt{barplot(table(x))} et \texttt{pie(table(x))} : diagrammes en
  bâtons et circulaires.
\end{itemize}

\pause Si ce n'est quelques \textbf{arguments spécifiques}, ces
fonctions partagent un ensemble de \textbf{paramètres graphiques
communs}.

\pause 

\textbf{Pour en savoir plus} Le site
\href{http://www.statmethods.net/graphs/}{\underline{statmethods.net}}
recense et illustre la plupart des fonctions du \emph{package}
\texttt{graphics}.

\end{frame}

\begin{frame}[fragile]{Histogrammes et densités}

Les fonctions \texttt{histogram()} et \texttt{density()} calculent les
statistiques ensuite utilisées par la fonction \texttt{plot()} pour
construire les graphiques.

\pause Arguments spécifiques à \texttt{hist()} : \vspace{-0.3cm}

\begin{itemize}
\tightlist
\item
  \texttt{breaks} : méthode pour déterminer les limites des classes;
\item
  \texttt{labels\ =\ TRUE} : ajoute l'effectif de chaque classe.
\end{itemize}

\pause Arguments spécifiques à \texttt{density()} : \vspace{-0.3cm}

\begin{itemize}
\tightlist
\item
  \texttt{bw} : largeur de la fenêtre utilisée par la fonction de
  lissage;
\item
  \texttt{kernel} : fonction de lissage utilisée.
\end{itemize}

\pause

\textbf{Remarque} L'argument \texttt{plot} de la fonction
\texttt{hist()} (\texttt{TRUE} par défaut) affiche automatiquement un
graphique, sans avoir à appeler explicitement la fonction
\texttt{plot()}.

\end{frame}

\begin{frame}[fragile]{Histogrammes et densités}

\centering \footnotesize

\begin{Shaded}
\begin{Highlighting}[]
\KeywordTok{hist}\NormalTok{(mpg$hwy, }\DataTypeTok{breaks =} \KeywordTok{seq}\NormalTok{(}\DecValTok{10}\NormalTok{, }\DecValTok{44}\NormalTok{, }\DataTypeTok{by =} \DecValTok{2}\NormalTok{), }
     \DataTypeTok{col =} \StringTok{"lightblue"}\NormalTok{, }\DataTypeTok{labels =} \OtherTok{TRUE}\NormalTok{)}
\end{Highlighting}
\end{Shaded}

\includegraphics[height=7cm]{presentation_files/figure-beamer/unnamed-chunk-82-1}

\end{frame}

\begin{frame}[fragile]{Histogrammes et densités}

\centering \footnotesize

\begin{Shaded}
\begin{Highlighting}[]
\KeywordTok{plot}\NormalTok{(}\KeywordTok{density}\NormalTok{(mpg$hwy, }\DataTypeTok{bw =} \FloatTok{0.5}\NormalTok{, }\DataTypeTok{kernel =} \StringTok{"gaussian"}\NormalTok{))}
\end{Highlighting}
\end{Shaded}

\includegraphics[height=7cm]{presentation_files/figure-beamer/unnamed-chunk-83-1}

\end{frame}

\begin{frame}[fragile]{Séries chronologiques avec \texttt{plot(ts)}}

\centering \footnotesize

\begin{Shaded}
\begin{Highlighting}[]
\KeywordTok{class}\NormalTok{(AirPassengers)}
  \NormalTok{## [1] "ts"}
\KeywordTok{plot}\NormalTok{(AirPassengers)}
\end{Highlighting}
\end{Shaded}

\includegraphics[height=6cm]{presentation_files/figure-beamer/unnamed-chunk-84-1}

\end{frame}

\begin{frame}[fragile]{Nuages de points avec \texttt{plot(x,\ y)}}

\centering \footnotesize

\begin{Shaded}
\begin{Highlighting}[]
\KeywordTok{plot}\NormalTok{(mpg$displ, mpg$hwy)}
\end{Highlighting}
\end{Shaded}

\includegraphics[height=6.5cm]{presentation_files/figure-beamer/unnamed-chunk-85-1}

\end{frame}

\begin{frame}[fragile]{Diagrammes en bâtons et circulaires}

La fonction \texttt{table()} permet de calculer les statistiques
utilisées ensuite par \texttt{barplot()} et \texttt{pie()} pour
construire les graphiques.

\pause Arguments spécifiques à \texttt{barplot()} : \vspace{-3mm}

\begin{itemize}
\tightlist
\item
  \texttt{horiz} : construit le graphique horizontalement;
\item
  \texttt{names.arg} : nom à afficher près des barres.
\end{itemize}

\pause Arguments spécifiques à \texttt{pie()} : \vspace{-3mm}

\begin{itemize}
\tightlist
\item
  \texttt{labels} : noms à afficher à côté des portions de disque;
\item
  \texttt{clockwise} : sens dans lequel sont représentées les modalités;
\item
  \texttt{init.angle} : point de départ en degrés.
\end{itemize}

\pause

\textbf{Remarque} Quand \texttt{barplot()} est appliqué à un tri croisé,
la couleur des barres varie et les paramètres deviennent utiles :
\vspace{-3mm}

\begin{itemize}
\tightlist
\item
  \texttt{beside} : position des barres;
\item
  \texttt{legend.text} : ajoute une légende avec le texte indiqué.
\end{itemize}

\end{frame}

\begin{frame}[fragile]{Diagrammes en bâtons et circulaires}

\centering \footnotesize

\begin{Shaded}
\begin{Highlighting}[]
\NormalTok{uni <-}\StringTok{ }\KeywordTok{table}\NormalTok{(mpg$drv)}
\NormalTok{lab <-}\StringTok{ }\KeywordTok{c}\NormalTok{(}\StringTok{"4 roues"}\NormalTok{, }\StringTok{"Traction"}\NormalTok{, }\StringTok{"Propulsion"}\NormalTok{)}
\KeywordTok{barplot}\NormalTok{(uni, }\DataTypeTok{names.arg =} \NormalTok{lab)}
\end{Highlighting}
\end{Shaded}

\includegraphics[height=6cm]{presentation_files/figure-beamer/unnamed-chunk-86-1}

\end{frame}

\begin{frame}[fragile]{Diagrammes en bâtons et circulaires}

\centering \footnotesize

\begin{Shaded}
\begin{Highlighting}[]
\KeywordTok{pie}\NormalTok{(uni, }\DataTypeTok{labels =} \KeywordTok{paste0}\NormalTok{(lab, }\StringTok{"}\CharTok{\textbackslash{}n}\StringTok{"}\NormalTok{, uni)}
    \NormalTok{, }\DataTypeTok{init.angle =} \DecValTok{90}\NormalTok{, }\DataTypeTok{clockwise =} \OtherTok{TRUE}\NormalTok{)}
\end{Highlighting}
\end{Shaded}

\includegraphics[height=6cm]{presentation_files/figure-beamer/unnamed-chunk-87-1}

\end{frame}

\begin{frame}[fragile]{Diagrammes en bâtons et circulaires}

\centering \footnotesize

\begin{Shaded}
\begin{Highlighting}[]
\NormalTok{bi <-}\StringTok{ }\KeywordTok{table}\NormalTok{(mpg$drv, mpg$year)}
\KeywordTok{barplot}\NormalTok{(bi, }\DataTypeTok{horiz =} \OtherTok{TRUE}\NormalTok{, }\DataTypeTok{beside =} \OtherTok{TRUE}\NormalTok{, }\DataTypeTok{legend.text =} \NormalTok{lab)}
\end{Highlighting}
\end{Shaded}

\includegraphics[height=6cm]{presentation_files/figure-beamer/unnamed-chunk-88-1}

\end{frame}

\begin{frame}[fragile]{Couleur, forme et taille des objets}

Plusieurs paramètres permettent de modifier la couleur, la forme ou la
taille des éléments qui composent un graphique:

\begin{itemize}
\tightlist
\item
  \pause \texttt{pch} : entier ou caractère spécial indiquant la forme
  des points à représenter.
\end{itemize}

\includegraphics{presentation_files/figure-beamer/unnamed-chunk-89-1.pdf}

\begin{itemize}
\tightlist
\item
  \pause \texttt{col} : valeur indiquant la couleur du contour des
  formes utilisées. Peut être un entier (recyclé au-delà de 8), un nom
  ou un code RGB hexadécimal (du type \texttt{"\#FF1111"}). \small  
\end{itemize}

\includegraphics{presentation_files/figure-beamer/unnamed-chunk-90-1.pdf}

Pour certaines formes (\texttt{pch} entre 21 et 25), il est également
possible de modifier la couleur de remplissage avec \texttt{bg}.

\end{frame}

\begin{frame}[fragile]{Couleur, forme et taille des objets}

\textbf{Remarque} : la palette de couleurs accessibles en utilisant des
entiers est réduite. Il est possible de l'étendre considérablement
\emph{via} la fonction \texttt{colors()}.

\small

\begin{Shaded}
\begin{Highlighting}[]
\KeywordTok{colors}\NormalTok{()[}\DecValTok{1}\NormalTok{:}\DecValTok{3}\NormalTok{]}
  \NormalTok{## [1] "white"        "aliceblue"    "antiquewhite"}
\KeywordTok{length}\NormalTok{(}\KeywordTok{colors}\NormalTok{())}
  \NormalTok{## [1] 657}
\KeywordTok{grep}\NormalTok{(}\StringTok{"blue"}\NormalTok{, }\KeywordTok{colors}\NormalTok{(), }\DataTypeTok{value =} \OtherTok{TRUE}\NormalTok{)[}\DecValTok{1}\NormalTok{:}\DecValTok{3}\NormalTok{]}
  \NormalTok{## [1] "aliceblue" "blue"      "blue1"}
\end{Highlighting}
\end{Shaded}

\pause \normalsize
- \texttt{cex} : utilisé dans une fonction \texttt{plot()}, \texttt{cex}
permet d'ajuster la taille des points qui le composent.

\includegraphics{presentation_files/figure-beamer/unnamed-chunk-92-1.pdf}

\end{frame}

\begin{frame}[fragile]{Couleur, forme et taille des objets}

La fonction \texttt{legend()} permet d'ajouter une légende.
\footnotesize \center

\pause \vspace{-0.4cm}

\begin{Shaded}
\begin{Highlighting}[]
\NormalTok{t <-}\StringTok{ }\KeywordTok{factor}\NormalTok{(mpg$drv}
  \NormalTok{, }\DataTypeTok{labels =} \KeywordTok{c}\NormalTok{(}\StringTok{"4 roues"}\NormalTok{, }\StringTok{"Traction"}\NormalTok{, }\StringTok{"Propulsion"}\NormalTok{))}
\KeywordTok{plot}\NormalTok{(mpg$displ, mpg$hwy, }\DataTypeTok{pch =} \DecValTok{21}\NormalTok{, }\DataTypeTok{col =} \NormalTok{t, }\DataTypeTok{bg =} \NormalTok{t)}
\KeywordTok{legend}\NormalTok{(}\StringTok{"topright"}\NormalTok{, }\DataTypeTok{legend =} \KeywordTok{unique}\NormalTok{(t), }\DataTypeTok{pch =} \DecValTok{21}
  \NormalTok{, }\DataTypeTok{col =} \KeywordTok{unique}\NormalTok{(t), }\DataTypeTok{pt.bg =} \KeywordTok{unique}\NormalTok{(t))}
\end{Highlighting}
\end{Shaded}

\includegraphics{presentation_files/figure-beamer/unnamed-chunk-93-1.pdf}

\end{frame}

\begin{frame}[fragile]{Titres, texte et axes}

Les titres sont paramétrés à l'aide des fonctions suivantes :

\vspace{-0.3cm}

\begin{itemize}
\tightlist
\item
  \texttt{main} pour ajouter le titre principal;
\item
  \texttt{xlab} et \texttt{ylab} pour ajouter des titres aux axes.
\end{itemize}

\pause La fonction \texttt{text()} permet d'ajouter du texte sur le
graphique en le positionnant par ses coordonnées, éventuellement avec un
décalage (pour nommer des points par exemple).

\pause Il est également possible de paramétrer les axes :

\vspace{-0.3cm}

\begin{itemize}
\tightlist
\item
  \texttt{xlim} et \texttt{ylim} spécifient les valeurs minimales et
  maximales de chaque axe;
\item
  \texttt{axis()} est une fonction qui permet d'ajouter un axe
  personnalisé.
\end{itemize}

\pause 

\textbf{Remarque} Pour produire un graphique sans axe et les rajouter
après, utiliser l'option \texttt{axes\ =\ FALSE} de la fonction
\texttt{plot()}.

\end{frame}

\begin{frame}[fragile]{Combinaison de plusieurs graphiques}

Par défaut l'utilisation de la fonction \texttt{plot()} produit un
nouveau graphique.

\pause Pour superposer différents graphiques, le plus simple est de
commencer par une instruction \texttt{plot()} puis de la compléter :

\begin{itemize}
\tightlist
\item
  avec \texttt{points()} pour ajouter des points;
\item
  avec \texttt{lines()} pour ajouter des lignes;
\item
  avec \texttt{abline()} pour ajouter des lignes d'après une équation;
\item
  avec \texttt{curve()} pour ajouter des courbes d'après une équation.
\end{itemize}

\pause 

\textbf{Exemple} Ajout d'une droite de régression au graphique de
\texttt{hwy} par \texttt{displ}.

\end{frame}

\begin{frame}[fragile]{Combinaison de plusieurs graphiques}

\footnotesize

\begin{Shaded}
\begin{Highlighting}[]
\NormalTok{reg <-}\StringTok{ }\KeywordTok{lm}\NormalTok{(hwy ~}\StringTok{ }\NormalTok{displ, }\DataTypeTok{data =} \NormalTok{mpg)}
\KeywordTok{plot}\NormalTok{(mpg$displ, mpg$hwy)}
\KeywordTok{abline}\NormalTok{(}\DataTypeTok{a =} \NormalTok{reg$coefficients[}\DecValTok{1}\NormalTok{], }\DataTypeTok{b =} \NormalTok{reg$coefficients[}\DecValTok{2}\NormalTok{])}
\end{Highlighting}
\end{Shaded}

\includegraphics{presentation_files/figure-beamer/unnamed-chunk-94-1.pdf}

\end{frame}

\begin{frame}[fragile]{Paramètres généraux et disposition (1)}

Utilisée en dehors de la fonction \texttt{plot()}, la fonction
\texttt{par()} permet de définir l'ensemble des paramètres graphiques
globaux.

\pause Ses mots-clés les plus importants sont :

\begin{itemize}
\item
  \texttt{mfrow} : permet de disposer plusieurs graphiques côte-à-côte.

\begin{Shaded}
\begin{Highlighting}[]
\KeywordTok{par}\NormalTok{(}\DataTypeTok{mfrow =} \KeywordTok{c}\NormalTok{(}\DecValTok{1}\NormalTok{, }\DecValTok{2}\NormalTok{)) }\CommentTok{# 1 ligne et 2 colonnes}
\KeywordTok{par}\NormalTok{(}\DataTypeTok{mfrow =} \KeywordTok{c}\NormalTok{(}\DecValTok{3}\NormalTok{, }\DecValTok{2}\NormalTok{)) }\CommentTok{# 3 lignes et 2 colonnes}
\KeywordTok{par}\NormalTok{(}\DataTypeTok{mfrow =} \KeywordTok{c}\NormalTok{(}\DecValTok{1}\NormalTok{, }\DecValTok{1}\NormalTok{)) }\CommentTok{# 1 ligne et 1 colonne}
\end{Highlighting}
\end{Shaded}
\item
  \texttt{cex} : coefficient multiplicatif pour modifier la taille de
  l'ensemble des textes et symboles utilisés dans les graphiques (1 par
  défaut).
\end{itemize}

\pause 

\textbf{Pour en savoir plus} La
\href{http://stat.ethz.ch/R-manual/R-devel/library/graphics/html/par.html}{page
d'aide} de la fonction \texttt{par()} détaille toutes ces options.

\end{frame}

\begin{frame}[fragile]{Paramètres généraux et disposition (2)}

\begin{Shaded}
\begin{Highlighting}[]
\KeywordTok{par}\NormalTok{(}\DataTypeTok{mfrow =} \KeywordTok{c}\NormalTok{(}\DecValTok{1}\NormalTok{, }\DecValTok{2}\NormalTok{))}
\KeywordTok{plot}\NormalTok{(mpg$displ, mpg$hwy)}
\KeywordTok{plot}\NormalTok{(AirPassengers)}
\end{Highlighting}
\end{Shaded}

\includegraphics{presentation_files/figure-beamer/unnamed-chunk-96-1.pdf}

\end{frame}

\begin{frame}[fragile]{Exportation}

Pour exporter des graphiques depuis R, la démarche consiste à rediriger
le flux de production du graphiques vers un fichier à l'aide d'une
fonction du \emph{package} \texttt{grDevices}. Par exemple :

\pause 

\begin{Shaded}
\begin{Highlighting}[]
\KeywordTok{png}\NormalTok{(}\StringTok{"monGraphique.png"}\NormalTok{, }\DataTypeTok{width =} \DecValTok{10}\NormalTok{, }\DataTypeTok{height =} \DecValTok{8}
    \NormalTok{, }\DataTypeTok{unit =} \StringTok{"cm"}\NormalTok{, }\DataTypeTok{res =} \DecValTok{600}\NormalTok{)}
\KeywordTok{plot}\NormalTok{(mpg$displ, mpg$hwy)}
\KeywordTok{dev.off}\NormalTok{()}
\end{Highlighting}
\end{Shaded}

\pause Dans ce contexte, les fonctions les plus utiles sont :
\texttt{png()}, \texttt{jpeg()} et \texttt{pdf()}. En particulier,
\texttt{pdf()} permet de conserver le caractère vectoriel des graphiques
dans R.

\pause 

\textbf{Remarque} Les graphiques peuvent également facilement être
exportés depuis RStudio en utilisant les menus spécialement conçus à cet
effet.

\end{frame}

\subsection{\texorpdfstring{Réaliser des graphiques avec
\protect\texttt{ggplot2}}{Réaliser des graphiques avec }}\label{realiser-des-graphiques-avec}

\begin{frame}[fragile]{\large L'implémentation d'une grammaire des
graphiques}

Le \emph{package} \texttt{graphics} permet de réaliser une grande
quantité de graphiques mais présente deux limites importantes :

\begin{itemize}
\tightlist
\item
  les fonctions qui le composent forment une casuistique complexe;
\item
  il n'est pas possible d'inventer de nouvelles représentations à partir
  des fonctions existantes.
\end{itemize}

\pause Ce sont ces limites que tente de dépasser le \emph{package}
\texttt{ggplot2} en implémentant une \textbf{grammaire des graphiques}

Comme les éléments du langage, les \textbf{composants élémentaires} d'un
graphique doivent pouvoir être \textbf{réassemblés} pour produire de
\textbf{nouvelles représentations.}

\pause 

\textbf{Pour aller plus loin} \textsc{Wilkinson L.} (2005)
\textit{The Grammar of Graphics}, Springer,
\href{https://github.com/hadley/ggplot2-book}{\underline{ggplot2: elegant graphics for data analysis}}

\end{frame}

\begin{frame}[fragile]{\large Les trois composants essentiels d'un
graphique}

La construction d'un graphique avec \texttt{ggplot2} fait intervenir
trois composants essentiels (d'après Wickham, \emph{ibid.}, 2.3) :

\begin{itemize}
\tightlist
\item
  le \texttt{data.frame} dans lequel sont stockées les données à
  représenter ;
\item
  des correspondances esthétiques (\emph{aesthetic mappings}) entre des
  variables et des propriétés visuelles;
\item
  au moins une couche (\emph{layer}) décrivant comment représenter les
  observations.
\end{itemize}

\pause 

\textbf{Exemple} \emph{Miles per gallon} sur l'autoroute en fonction de
la cylindrée.

\center \small 

\begin{Shaded}
\begin{Highlighting}[]
\KeywordTok{ggplot}\NormalTok{(}\DataTypeTok{data =} \NormalTok{mpg, }\DataTypeTok{mapping =} \KeywordTok{aes}\NormalTok{(}\DataTypeTok{x =} \NormalTok{displ, }\DataTypeTok{y =} \NormalTok{hwy)) +}
\StringTok{  }\KeywordTok{geom_point}\NormalTok{()}
\end{Highlighting}
\end{Shaded}

\end{frame}

\begin{frame}[fragile]{\large Les trois composants essentiels d'un
graphique}

\center \small 

\begin{Shaded}
\begin{Highlighting}[]
\KeywordTok{ggplot}\NormalTok{(}\DataTypeTok{data =} \NormalTok{mpg, }\DataTypeTok{mapping =} \KeywordTok{aes}\NormalTok{(}\DataTypeTok{x =} \NormalTok{displ, }\DataTypeTok{y =} \NormalTok{hwy)) +}
\StringTok{  }\KeywordTok{geom_point}\NormalTok{()}
\end{Highlighting}
\end{Shaded}

\includegraphics[width=0.9\linewidth]{presentation_files/figure-beamer/unnamed-chunk-100-1}

\end{frame}

\begin{frame}{Couleur, forme et taille des objets}

Pour faire varier l'aspect visuel des éléments représentés en fonction
de données, il suffit
d'\textbf{associer une variable à l'attribut de couleur, de taille ou de forme}
dans la fonction \texttt{aes()}.

\textcolor{white}{Selon le type des variables utilisées pour les correspondances esthétiques, \textbf{les échelles sont continues ou discrètes}.}

\textcolor{white}{Quand la même variable est utilisée dans plusieurs correspondances esthétiques, \textbf{les échelles qui lui correspondent sont fusionnées}.}

\textcolor{white}{Au-delà des correspondances esthétiques dans la fonction \texttt{aes()}, \textbf{l'aspect visuel peut être ajusté directement dans la fonction \texttt{geom\_*}}.}

\end{frame}

\begin{frame}[fragile]{Couleur, forme et taille des objets}

\footnotesize \center

\begin{Shaded}
\begin{Highlighting}[]
\KeywordTok{ggplot}\NormalTok{(mpg, }\KeywordTok{aes}\NormalTok{(displ, hwy, }\DataTypeTok{colour =} \NormalTok{cyl, }\DataTypeTok{shape =} \NormalTok{drv)) +}
\StringTok{  }\KeywordTok{geom_point}\NormalTok{()}
\end{Highlighting}
\end{Shaded}

\includegraphics[width=0.9\linewidth]{presentation_files/figure-beamer/unnamed-chunk-101-1}

\end{frame}

\begin{frame}{Couleur, forme et taille des objets}

Pour faire varier l'aspect visuel des éléments représentés en fonction
de données, il suffit
d'\textbf{associer une variable à l'attribut de couleur, de taille ou de forme}
dans la fonction \texttt{aes()}.

Selon le type des variables utilisées pour les correspondances
esthétiques, \textbf{les échelles sont continues ou discrètes}.

\textcolor{white}{Quand la même variable est utilisée dans plusieurs correspondances esthétiques, \textbf{les échelles qui lui correspondent sont fusionnées}.}

\textcolor{white}{Au-delà des correspondances esthétiques dans la fonction \texttt{aes()}, \textbf{l'aspect visuel peut être ajusté directement dans la fonction \texttt{geom\_*}}.}

\end{frame}

\begin{frame}[fragile]{Couleur, forme et taille des objets}

\footnotesize \center

\begin{Shaded}
\begin{Highlighting}[]
\KeywordTok{ggplot}\NormalTok{(mpg, }\KeywordTok{aes}\NormalTok{(displ, hwy, }\DataTypeTok{colour =} \KeywordTok{as.factor}\NormalTok{(cyl)}
  \NormalTok{, }\DataTypeTok{shape =} \NormalTok{drv)) +}
\StringTok{  }\KeywordTok{geom_point}\NormalTok{()}
\end{Highlighting}
\end{Shaded}

\includegraphics[width=0.9\linewidth]{presentation_files/figure-beamer/unnamed-chunk-102-1}

\end{frame}

\begin{frame}{Couleur, forme et taille des objets}

Pour faire varier l'aspect visuel des éléments représentés en fonction
de données, il suffit
d'\textbf{associer une variable à l'attribut de couleur, de taille ou de forme}
dans la fonction \texttt{aes()}.

Selon le type des variables utilisées pour les correspondances
esthétiques, \textbf{les échelles sont continues ou discrètes}.

Quand la même variable est utilisée dans plusieurs correspondances
esthétiques,
\textbf{les échelles qui lui correspondent sont fusionnées}.

\textcolor{white}{Au-delà des correspondances esthétiques dans la fonction \texttt{aes()}, \textbf{l'aspect visuel peut être ajusté directement dans la fonction \texttt{geom\_*}}.}

\end{frame}

\begin{frame}[fragile]{Couleur, forme et taille des objets}

\footnotesize \center

\begin{Shaded}
\begin{Highlighting}[]
\KeywordTok{ggplot}\NormalTok{(mpg, }\KeywordTok{aes}\NormalTok{(displ, hwy, }\DataTypeTok{colour =} \KeywordTok{as.factor}\NormalTok{(cyl)}
  \NormalTok{, }\DataTypeTok{shape =} \KeywordTok{as.factor}\NormalTok{(cyl))) +}
\StringTok{  }\KeywordTok{geom_point}\NormalTok{()}
\end{Highlighting}
\end{Shaded}

\includegraphics[width=0.9\linewidth]{presentation_files/figure-beamer/unnamed-chunk-103-1}

\end{frame}

\begin{frame}{Couleur, forme et taille des objets}

Pour faire varier l'aspect visuel des éléments représentés en fonction
de données, il suffit
d'\textbf{associer une variable à l'attribut de couleur, de taille ou de forme}
dans la fonction \texttt{aes()}.

Selon le type des variables utilisées pour les correspondances
esthétiques, \textbf{les échelles sont continues ou discrètes}.

Quand la même variable est utilisée dans plusieurs correspondances
esthétiques,
\textbf{les échelles qui lui correspondent sont fusionnées}.

Au-delà des correspondances esthétiques dans la fonction \texttt{aes()},
\textbf{l'aspect visuel peut être ajusté directement dans la fonction \texttt{geom\_*}}.

\end{frame}

\begin{frame}[fragile]{Couleur, forme et taille des objets}

\footnotesize \center

\begin{Shaded}
\begin{Highlighting}[]
\KeywordTok{ggplot}\NormalTok{(mpg, }\KeywordTok{aes}\NormalTok{(displ, hwy)) +}
\StringTok{  }\KeywordTok{geom_point}\NormalTok{(}\DataTypeTok{colour =} \StringTok{"red"}\NormalTok{, }\DataTypeTok{size =} \DecValTok{8}\NormalTok{, }\DataTypeTok{alpha =} \FloatTok{0.5}\NormalTok{)}
\end{Highlighting}
\end{Shaded}

\includegraphics[width=0.9\linewidth]{presentation_files/figure-beamer/unnamed-chunk-104-1}

\end{frame}

\begin{frame}[fragile]{Combinaison de plusieurs graphiques}

\footnotesize \center

\begin{Shaded}
\begin{Highlighting}[]
\KeywordTok{ggplot}\NormalTok{(mpg, }\KeywordTok{aes}\NormalTok{(displ, hwy)) +}
\StringTok{  }\KeywordTok{geom_point}\NormalTok{() +}\StringTok{ }\KeywordTok{geom_smooth}\NormalTok{()}
  \NormalTok{## `geom_smooth()` using method = 'loess'}
\end{Highlighting}
\end{Shaded}

\includegraphics[width=0.9\linewidth]{presentation_files/figure-beamer/unnamed-chunk-105-1}

\end{frame}

\begin{frame}[fragile]{Combinaison de plusieurs graphiques}

\footnotesize \center

\begin{Shaded}
\begin{Highlighting}[]
\KeywordTok{ggplot}\NormalTok{(mpg, }\KeywordTok{aes}\NormalTok{(displ, hwy)) +}
\StringTok{  }\KeywordTok{geom_point}\NormalTok{() +}\StringTok{ }\KeywordTok{geom_smooth}\NormalTok{(}\DataTypeTok{method =} \StringTok{"lm"}\NormalTok{, }\DataTypeTok{se =} \OtherTok{FALSE}\NormalTok{)}
\end{Highlighting}
\end{Shaded}

\includegraphics[width=0.9\linewidth]{presentation_files/figure-beamer/unnamed-chunk-106-1}

\end{frame}

\begin{frame}[fragile]{Combinaison de plusieurs graphiques}

\footnotesize \center

\begin{Shaded}
\begin{Highlighting}[]
\KeywordTok{ggplot}\NormalTok{(mpg, }\KeywordTok{aes}\NormalTok{(displ, hwy, }\DataTypeTok{colour =} \NormalTok{drv)) +}
\StringTok{  }\KeywordTok{geom_point}\NormalTok{() +}\StringTok{ }\KeywordTok{geom_smooth}\NormalTok{(}\DataTypeTok{method =} \StringTok{"lm"}\NormalTok{, }\DataTypeTok{se =} \OtherTok{FALSE}\NormalTok{)}
\end{Highlighting}
\end{Shaded}

\includegraphics[width=0.9\linewidth]{presentation_files/figure-beamer/unnamed-chunk-107-1}

\end{frame}

\begin{frame}[fragile]{\large Le fonctionnement en \og couches \fg{} de
\texttt{ggplot2}}

La construction d'un graphique dans \texttt{ggplot2} repose sur la
superposition de couches (\emph{layer}) \textbf{conçues indépendamment}
mais \textbf{réconciliées en fin d'opération}.

\pause Chaque couche est composée de cinq éléments :

\begin{itemize}
\tightlist
\item
  un \texttt{data.frame} (\texttt{data});
\item
  une ou plusieurs correspondances esthétiques (\texttt{mapping});
\item
  une transformation statistique (\texttt{stat});
\item
  un objet géométrique (\texttt{geom});
\item
  un paramètre d'ajustement de la position (\texttt{position}).
\end{itemize}

C'est la \textbf{fonction \texttt{layer()}} qui articule ces cinq
éléments.

\pause 

\textbf{Les fonctions \texttt{geom\_*} vues précédemment sont des appels
pré-paramétrées de \texttt{layer()}}.

\end{frame}

\begin{frame}[fragile]{\large Le fonctionnement en \og couches \fg{} de
\texttt{ggplot2}}

\emph{Un graphique à une couche}

\footnotesize \centering

\begin{Shaded}
\begin{Highlighting}[]
\KeywordTok{ggplot}\NormalTok{() +}\StringTok{ }\KeywordTok{layer}\NormalTok{(}
  \DataTypeTok{data =} \NormalTok{mpg, }\DataTypeTok{mapping =} \KeywordTok{aes}\NormalTok{(displ, hwy), }\DataTypeTok{stat =} \StringTok{"identity"}
  \NormalTok{, }\DataTypeTok{geom =} \StringTok{"point"}\NormalTok{, }\DataTypeTok{position =} \StringTok{"identity"}
\NormalTok{)}
\end{Highlighting}
\end{Shaded}

\includegraphics[width=0.8\linewidth]{presentation_files/figure-beamer/unnamed-chunk-108-1}

\end{frame}

\begin{frame}[fragile]{\large Le fonctionnement en \og couches \fg{} de
\texttt{ggplot2}}

\emph{Un graphique à deux couches}

\footnotesize \centering

\begin{Shaded}
\begin{Highlighting}[]
\KeywordTok{ggplot}\NormalTok{() +}\StringTok{ }\KeywordTok{layer}\NormalTok{(}
  \DataTypeTok{data =} \NormalTok{mpg, }\DataTypeTok{mapping =} \KeywordTok{aes}\NormalTok{(displ, hwy), }\DataTypeTok{stat =} \StringTok{"identity"}
  \NormalTok{, }\DataTypeTok{geom =} \StringTok{"point"}\NormalTok{, }\DataTypeTok{position =} \StringTok{"identity"}
\NormalTok{) +}\StringTok{ }\KeywordTok{layer}\NormalTok{(}
  \DataTypeTok{data =} \NormalTok{mpg, }\DataTypeTok{mapping =} \KeywordTok{aes}\NormalTok{(displ, hwy), }\DataTypeTok{stat =} \StringTok{"smooth"}
  \NormalTok{, }\DataTypeTok{geom =} \StringTok{"line"}\NormalTok{, }\DataTypeTok{position =} \StringTok{"identity"}
  \NormalTok{, }\DataTypeTok{params =} \KeywordTok{list}\NormalTok{(}\DataTypeTok{method =} \StringTok{"lm"}\NormalTok{, }\DataTypeTok{formula =} \NormalTok{y ~}\StringTok{ }\NormalTok{x)}
\NormalTok{)}
\end{Highlighting}
\end{Shaded}

\vfill

\vfill

\end{frame}

\begin{frame}{\large Le fonctionnement en \og couches \fg{} de
\texttt{ggplot2}}

\emph{Un graphique à deux couches}

\centering

\includegraphics[width=1\linewidth]{presentation_files/figure-beamer/unnamed-chunk-110-1}

\end{frame}

\begin{frame}[fragile]{\large Le fonctionnement en \og couches \fg{} de
\texttt{ggplot2}}

\emph{Mise en facteur dans \texttt{ggplot()} de \texttt{data} et
\texttt{mapping}}

\footnotesize \vspace{-1mm}

\begin{Shaded}
\begin{Highlighting}[]
\KeywordTok{ggplot}\NormalTok{(}\DataTypeTok{data =} \NormalTok{mpg, }\DataTypeTok{mapping =} \KeywordTok{aes}\NormalTok{(displ, hwy)) +}\StringTok{ }\KeywordTok{layer}\NormalTok{(}
  \DataTypeTok{stat =} \StringTok{"identity"}\NormalTok{, }\DataTypeTok{geom =} \StringTok{"point"}\NormalTok{, }\DataTypeTok{position =} \StringTok{"identity"}
\NormalTok{) +}\StringTok{ }\KeywordTok{layer}\NormalTok{(}
  \DataTypeTok{stat =} \StringTok{"smooth"}\NormalTok{, }\DataTypeTok{geom =} \StringTok{"line"}\NormalTok{, }\DataTypeTok{position =} \StringTok{"identity"}
  \NormalTok{, }\DataTypeTok{params =} \KeywordTok{list}\NormalTok{(}\DataTypeTok{method =} \StringTok{"lm"}\NormalTok{, }\DataTypeTok{formula =} \NormalTok{y ~}\StringTok{ }\NormalTok{x)}
\NormalTok{)}
\end{Highlighting}
\end{Shaded}

\normalsize \vspace{-3mm}

\emph{Remplacement de \texttt{layer()} par des alias pré-paramétrés}

\footnotesize \vspace{-1mm}

\begin{Shaded}
\begin{Highlighting}[]
\KeywordTok{ggplot}\NormalTok{(}\DataTypeTok{data =} \NormalTok{mpg, }\DataTypeTok{mapping =} \KeywordTok{aes}\NormalTok{(displ, hwy)) +}\StringTok{ }
\StringTok{  }\KeywordTok{geom_point}\NormalTok{() +}\StringTok{ }\KeywordTok{geom_smooth}\NormalTok{(}\DataTypeTok{method =} \StringTok{"lm"}\NormalTok{, }\DataTypeTok{se =} \OtherTok{FALSE}\NormalTok{)}
\end{Highlighting}
\end{Shaded}

\begin{Shaded}
\begin{Highlighting}[]
\KeywordTok{ggplot}\NormalTok{(}\DataTypeTok{data =} \NormalTok{mpg, }\DataTypeTok{mapping =} \KeywordTok{aes}\NormalTok{(displ, hwy)) +}\StringTok{ }
\StringTok{  }\KeywordTok{geom_point}\NormalTok{() +}\StringTok{ }\KeywordTok{stat_smooth}\NormalTok{(}\DataTypeTok{method =} \StringTok{"lm"}\NormalTok{, }\DataTypeTok{se =} \OtherTok{FALSE}\NormalTok{)}
\end{Highlighting}
\end{Shaded}

\end{frame}

\begin{frame}[fragile]{\large Le fonctionnement en \og couches \fg{} de
\texttt{ggplot2}}

À chaque fonction \texttt{geom\_*()} est assocée un paramètre
\texttt{stat} par défaut, et à chaque fonction \texttt{stat\_*()} un
\texttt{geom} par défaut.

\footnotesize \center

\begin{Shaded}
\begin{Highlighting}[]
\KeywordTok{ggplot}\NormalTok{(}\DataTypeTok{data =} \NormalTok{mpg, }\DataTypeTok{mapping =} \KeywordTok{aes}\NormalTok{(displ, hwy)) +}\StringTok{ }
\StringTok{  }\KeywordTok{geom_point}\NormalTok{(}\DataTypeTok{colour =} \StringTok{"red"}\NormalTok{, }\KeywordTok{aes}\NormalTok{(}\DataTypeTok{size =} \NormalTok{cyl)) +}\StringTok{ }
\StringTok{  }\KeywordTok{stat_smooth}\NormalTok{(}\DataTypeTok{geom =} \StringTok{"point"}\NormalTok{, }\DataTypeTok{method =} \StringTok{"lm"}\NormalTok{, }\DataTypeTok{se =} \OtherTok{FALSE}
    \NormalTok{, }\DataTypeTok{colour =} \StringTok{"blue"}\NormalTok{, }\DataTypeTok{shape =} \DecValTok{2}\NormalTok{)}
\end{Highlighting}
\end{Shaded}

\includegraphics[width=0.9\linewidth]{presentation_files/figure-beamer/unnamed-chunk-114-1}

\end{frame}

\begin{frame}[fragile]{\large Le fonctionnement en \og couches \fg{} de
\texttt{ggplot2}}

\footnotesize \center

\begin{Shaded}
\begin{Highlighting}[]
\KeywordTok{ggplot}\NormalTok{(mpg, }\KeywordTok{aes}\NormalTok{(displ, hwy)) +}\StringTok{ }
\StringTok{  }\KeywordTok{geom_point}\NormalTok{(}\KeywordTok{aes}\NormalTok{(}\DataTypeTok{colour =} \NormalTok{drv)) +}\StringTok{ }
\StringTok{  }\KeywordTok{stat_smooth}\NormalTok{(}\DataTypeTok{method =} \StringTok{"lm"}\NormalTok{, }\DataTypeTok{se =} \OtherTok{FALSE}\NormalTok{)}
\end{Highlighting}
\end{Shaded}

\includegraphics[width=0.9\linewidth]{presentation_files/figure-beamer/unnamed-chunk-115-1}

\end{frame}

\begin{frame}[fragile]{\large Le fonctionnement en \og couches \fg{} de
\texttt{ggplot2}}

\footnotesize \center

\begin{Shaded}
\begin{Highlighting}[]
\KeywordTok{ggplot}\NormalTok{(mpg, }\KeywordTok{aes}\NormalTok{(displ, hwy)) +}\StringTok{ }
\StringTok{  }\KeywordTok{geom_point}\NormalTok{(}\KeywordTok{aes}\NormalTok{(}\DataTypeTok{shape =} \NormalTok{drv), }\DataTypeTok{colour =} \StringTok{"red"}\NormalTok{) +}\StringTok{ }
\StringTok{  }\KeywordTok{stat_smooth}\NormalTok{(}\KeywordTok{aes}\NormalTok{(}\DataTypeTok{colour =} \NormalTok{class), }\DataTypeTok{method =} \StringTok{"lm"}\NormalTok{, }\DataTypeTok{se =} \OtherTok{FALSE}\NormalTok{)}
\end{Highlighting}
\end{Shaded}

\includegraphics[width=0.9\linewidth]{presentation_files/figure-beamer/unnamed-chunk-116-1}

\end{frame}

\begin{frame}[fragile]{Histogrammes et densités}

\footnotesize \center

\vspace{-0.3cm}

\begin{Shaded}
\begin{Highlighting}[]
\KeywordTok{ggplot}\NormalTok{(mpg, }\KeywordTok{aes}\NormalTok{(hwy)) +}\StringTok{ }\KeywordTok{geom_histogram}\NormalTok{()}
\end{Highlighting}
\end{Shaded}

\includegraphics[width=0.8\linewidth]{presentation_files/figure-beamer/unnamed-chunk-117-1}

\pause \raggedright \small \vspace{-0.3cm}

\textbf{Remarque} Le positionnement des classes des histogrammes semble
perturbé dans les dernières versions de \texttt{ggplot2} : le paramètre
\texttt{boundary} permet de corriger ce problème (\emph{cf.}
\href{http://stackoverflow.com/questions/37876096/geom-histogram-wrong-bins}{\underline{cette discussion}}).

\end{frame}

\begin{frame}[fragile]{Histogrammes et densités}

\footnotesize \center

\begin{Shaded}
\begin{Highlighting}[]
\KeywordTok{ggplot}\NormalTok{(mpg, }\KeywordTok{aes}\NormalTok{(hwy, }\DataTypeTok{colour =} \NormalTok{drv, }\DataTypeTok{fill =} \NormalTok{drv)) +}\StringTok{ }
\StringTok{  }\KeywordTok{geom_histogram}\NormalTok{()}
\end{Highlighting}
\end{Shaded}

\includegraphics[width=0.9\linewidth]{presentation_files/figure-beamer/unnamed-chunk-118-1}

\end{frame}

\begin{frame}[fragile]{Histogrammes et densités}

\footnotesize \center

\begin{Shaded}
\begin{Highlighting}[]
\KeywordTok{ggplot}\NormalTok{(mpg, }\KeywordTok{aes}\NormalTok{(hwy)) +}\StringTok{ }\KeywordTok{geom_density}\NormalTok{(}\DataTypeTok{bw =} \FloatTok{0.5}\NormalTok{)}
\end{Highlighting}
\end{Shaded}

\includegraphics[width=0.9\linewidth]{presentation_files/figure-beamer/unnamed-chunk-119-1}

\end{frame}

\begin{frame}[fragile]{Histogrammes et densités}

\footnotesize \center

\begin{Shaded}
\begin{Highlighting}[]
\KeywordTok{ggplot}\NormalTok{(mpg, }\KeywordTok{aes}\NormalTok{(hwy, }\DataTypeTok{colour =} \NormalTok{drv, }\DataTypeTok{fill =} \NormalTok{drv)) +}\StringTok{ }
\StringTok{  }\KeywordTok{geom_density}\NormalTok{(}\DataTypeTok{bw =} \FloatTok{0.5}\NormalTok{, }\DataTypeTok{alpha =} \FloatTok{0.5}\NormalTok{)}
\end{Highlighting}
\end{Shaded}

\includegraphics[width=0.9\linewidth]{presentation_files/figure-beamer/unnamed-chunk-120-1}

\end{frame}

\begin{frame}[fragile]{Séries temporelles}

\footnotesize \center

\begin{Shaded}
\begin{Highlighting}[]
\KeywordTok{ggplot}\NormalTok{(economics, }\KeywordTok{aes}\NormalTok{(date, unemploy /}\StringTok{ }\NormalTok{pop)) +}
\StringTok{  }\KeywordTok{geom_line}\NormalTok{()}
\end{Highlighting}
\end{Shaded}

\includegraphics[width=0.9\linewidth]{presentation_files/figure-beamer/unnamed-chunk-121-1}

\end{frame}

\begin{frame}[fragile]{Diagrammes en bâtons et circulaires}

\footnotesize \center

\begin{Shaded}
\begin{Highlighting}[]
\KeywordTok{ggplot}\NormalTok{(mpg, }\KeywordTok{aes}\NormalTok{(drv, }\DataTypeTok{colour =} \NormalTok{drv, }\DataTypeTok{fill =} \NormalTok{drv)) +}\StringTok{ }
\StringTok{  }\KeywordTok{geom_bar}\NormalTok{()}
\end{Highlighting}
\end{Shaded}

\includegraphics[width=0.9\linewidth]{presentation_files/figure-beamer/unnamed-chunk-122-1}

\end{frame}

\begin{frame}[fragile]{Diagrammes en bâtons et circulaires}

\footnotesize \center

\begin{Shaded}
\begin{Highlighting}[]
\KeywordTok{library}\NormalTok{(scales)}
\KeywordTok{ggplot}\NormalTok{(mpg, }\KeywordTok{aes}\NormalTok{(drv, }\DataTypeTok{fill =} \NormalTok{drv)) +}\StringTok{ }
\StringTok{  }\KeywordTok{geom_bar}\NormalTok{(}\KeywordTok{aes}\NormalTok{(}\DataTypeTok{y =} \NormalTok{(..count..)/}\KeywordTok{sum}\NormalTok{(..count..))) +}
\StringTok{  }\KeywordTok{scale_y_continuous}\NormalTok{(}\DataTypeTok{labels=}\NormalTok{percent) +}
\StringTok{  }\KeywordTok{scale_fill_brewer}\NormalTok{(}\DataTypeTok{palette=}\StringTok{"Blues"}\NormalTok{)}
\end{Highlighting}
\end{Shaded}

\includegraphics[width=0.7\linewidth]{presentation_files/figure-beamer/unnamed-chunk-123-1}

\end{frame}

\begin{frame}[fragile]{Diagrammes en bâtons et circulaires}

\footnotesize \center

\begin{Shaded}
\begin{Highlighting}[]
\NormalTok{g <-}\StringTok{ }\KeywordTok{ggplot}\NormalTok{(mpg, }\KeywordTok{aes}\NormalTok{(}\DataTypeTok{x =} \StringTok{""}\NormalTok{, }\DataTypeTok{fill =} \NormalTok{drv, }\DataTypeTok{colour =} \NormalTok{drv)) +}\StringTok{ }
\StringTok{  }\KeywordTok{geom_bar}\NormalTok{(}\DataTypeTok{width =} \DecValTok{1}\NormalTok{)}
\NormalTok{g}
\end{Highlighting}
\end{Shaded}

\includegraphics[width=0.75\linewidth]{presentation_files/figure-beamer/unnamed-chunk-124-1}

\end{frame}

\begin{frame}[fragile]{Diagrammes en bâtons et circulaires}

\footnotesize \center

\begin{Shaded}
\begin{Highlighting}[]
\NormalTok{g +}\StringTok{ }\KeywordTok{coord_polar}\NormalTok{(}\DataTypeTok{theta =} \StringTok{"y"}\NormalTok{) +}\StringTok{ }\KeywordTok{theme_minimal}\NormalTok{() +}
\StringTok{  }\KeywordTok{scale_fill_grey}\NormalTok{() +}\StringTok{ }\KeywordTok{scale_colour_grey}\NormalTok{()}
\end{Highlighting}
\end{Shaded}

\includegraphics[width=0.5\linewidth]{presentation_files/figure-beamer/unnamed-chunk-125-1}

\pause \raggedright \small

\textbf{Pour aller plus loin} Une page du site
\href{http://www.sthda.com/french/wiki/ggplot2-graphique-en-camembert-guide-de-demarrage-rapide-logiciel-r-et-visualisation-de-donnees}{\underline{sthda.com}}
explique (en français) comment produire un diagramme circulaire complet
avec \texttt{ggplot2}.

\end{frame}

\begin{frame}[fragile]{Diagrammes en bâtons et circulaires}

\footnotesize \center

\begin{Shaded}
\begin{Highlighting}[]
\KeywordTok{ggplot}\NormalTok{(mpg, }\KeywordTok{aes}\NormalTok{(drv, }\DataTypeTok{fill =} \KeywordTok{as.factor}\NormalTok{(year))) +}\StringTok{ }
\StringTok{  }\KeywordTok{geom_bar}\NormalTok{()}
\end{Highlighting}
\end{Shaded}

\includegraphics[width=0.9\linewidth]{presentation_files/figure-beamer/unnamed-chunk-126-1}

\end{frame}

\begin{frame}[fragile]{Diagrammes en bâtons et circulaires}

\footnotesize \center

\begin{Shaded}
\begin{Highlighting}[]
\KeywordTok{ggplot}\NormalTok{(mpg, }\KeywordTok{aes}\NormalTok{(drv, }\DataTypeTok{fill =} \KeywordTok{as.factor}\NormalTok{(year))) +}\StringTok{ }
\StringTok{  }\KeywordTok{geom_bar}\NormalTok{(}\DataTypeTok{position =} \StringTok{"fill"}\NormalTok{)}
\end{Highlighting}
\end{Shaded}

\includegraphics[width=0.9\linewidth]{presentation_files/figure-beamer/unnamed-chunk-127-1}

\end{frame}

\begin{frame}[fragile]{Diagrammes en bâtons et circulaires}

\footnotesize \center

\begin{Shaded}
\begin{Highlighting}[]
\KeywordTok{ggplot}\NormalTok{(mpg, }\KeywordTok{aes}\NormalTok{(}\KeywordTok{as.factor}\NormalTok{(year), }\DataTypeTok{fill =} \NormalTok{drv)) +}\StringTok{ }
\StringTok{  }\KeywordTok{geom_bar}\NormalTok{(}\DataTypeTok{position =} \StringTok{"dodge"}\NormalTok{) +}\StringTok{ }
\StringTok{  }\KeywordTok{coord_flip}\NormalTok{()}
\end{Highlighting}
\end{Shaded}

\includegraphics[width=0.9\linewidth]{presentation_files/figure-beamer/unnamed-chunk-128-1}

\end{frame}

\begin{frame}[fragile]{Boîtes à moustaches et assimilés}

\footnotesize \center

\begin{Shaded}
\begin{Highlighting}[]
\KeywordTok{ggplot}\NormalTok{(mpg, }\KeywordTok{aes}\NormalTok{(}\DataTypeTok{x =} \NormalTok{drv, }\DataTypeTok{y =} \NormalTok{hwy)) +}\StringTok{ }
\StringTok{  }\KeywordTok{geom_boxplot}\NormalTok{(}\DataTypeTok{coef =} \FloatTok{1.5}\NormalTok{)}
\end{Highlighting}
\end{Shaded}

\includegraphics[width=0.9\linewidth]{presentation_files/figure-beamer/unnamed-chunk-129-1}

\end{frame}

\begin{frame}[fragile]{Boîtes à moustaches et assimilés}

\footnotesize \center

\begin{Shaded}
\begin{Highlighting}[]
\KeywordTok{ggplot}\NormalTok{(mpg, }\KeywordTok{aes}\NormalTok{(}\DataTypeTok{x =} \NormalTok{drv, }\DataTypeTok{y =} \NormalTok{hwy)) +}\StringTok{ }
\StringTok{  }\KeywordTok{geom_jitter}\NormalTok{()}
\end{Highlighting}
\end{Shaded}

\includegraphics[width=0.9\linewidth]{presentation_files/figure-beamer/unnamed-chunk-130-1}

\end{frame}

\begin{frame}[fragile]{Boîtes à moustaches et assimilés}

\footnotesize \center

\begin{Shaded}
\begin{Highlighting}[]
\KeywordTok{ggplot}\NormalTok{(mpg, }\KeywordTok{aes}\NormalTok{(}\DataTypeTok{x =} \NormalTok{drv, }\DataTypeTok{y =} \NormalTok{hwy)) +}\StringTok{ }
\StringTok{  }\KeywordTok{geom_violin}\NormalTok{()}
\end{Highlighting}
\end{Shaded}

\includegraphics[width=0.9\linewidth]{presentation_files/figure-beamer/unnamed-chunk-131-1}

\end{frame}

\begin{frame}[fragile]{Titres et axes}

\footnotesize \center

\begin{Shaded}
\begin{Highlighting}[]
\KeywordTok{ggplot}\NormalTok{(mpg, }\KeywordTok{aes}\NormalTok{(displ, hwy)) +}\StringTok{ }\KeywordTok{geom_point}\NormalTok{() +}\StringTok{ }
\StringTok{  }\KeywordTok{ggtitle}\NormalTok{(}\StringTok{"Mon titre avec un retour }\CharTok{\textbackslash{}n}\StringTok{à la ligne"}\NormalTok{) +}
\StringTok{  }\KeywordTok{xlab}\NormalTok{(}\StringTok{"Cylindrée"}\NormalTok{) +}\StringTok{ }\KeywordTok{ylab}\NormalTok{(}\StringTok{"Miles per gallon"}\NormalTok{) +}
\StringTok{  }\KeywordTok{coord_cartesian}\NormalTok{(}\DataTypeTok{xlim =} \KeywordTok{c}\NormalTok{(}\DecValTok{0}\NormalTok{,}\DecValTok{10}\NormalTok{), }\DataTypeTok{ylim =} \KeywordTok{c}\NormalTok{(}\DecValTok{0}\NormalTok{, }\DecValTok{100}\NormalTok{))}
\end{Highlighting}
\end{Shaded}

\includegraphics[width=0.9\linewidth]{presentation_files/figure-beamer/unnamed-chunk-132-1}

\end{frame}

\begin{frame}[fragile]{Disposition : le \emph{facetting}}

\footnotesize \center

\begin{Shaded}
\begin{Highlighting}[]
\KeywordTok{ggplot}\NormalTok{(mpg, }\KeywordTok{aes}\NormalTok{(displ, hwy)) +}
\StringTok{  }\KeywordTok{geom_point}\NormalTok{() +}\StringTok{ }\KeywordTok{geom_smooth}\NormalTok{(}\DataTypeTok{method =} \StringTok{"lm"}\NormalTok{, }\DataTypeTok{se =} \OtherTok{FALSE}\NormalTok{) +}\StringTok{ }
\StringTok{  }\KeywordTok{facet_wrap}\NormalTok{(~manufacturer, }\DataTypeTok{nrow =} \DecValTok{3}\NormalTok{)}
\end{Highlighting}
\end{Shaded}

\includegraphics[width=0.9\linewidth]{presentation_files/figure-beamer/unnamed-chunk-133-1}

\end{frame}

\begin{frame}[fragile]{Disposition : le \emph{facetting}}

\footnotesize \center

\begin{Shaded}
\begin{Highlighting}[]
\KeywordTok{ggplot}\NormalTok{(mpg, }\KeywordTok{aes}\NormalTok{(displ, hwy)) +}
\StringTok{  }\KeywordTok{geom_point}\NormalTok{() +}\StringTok{ }\KeywordTok{geom_smooth}\NormalTok{(}\DataTypeTok{method =} \StringTok{"lm"}\NormalTok{, }\DataTypeTok{se =} \OtherTok{FALSE}\NormalTok{) +}\StringTok{ }
\StringTok{  }\KeywordTok{facet_grid}\NormalTok{(drv~class)}
\end{Highlighting}
\end{Shaded}

\includegraphics[width=1\linewidth]{presentation_files/figure-beamer/unnamed-chunk-134-1}

\end{frame}

\begin{frame}[fragile]{Sauvegarde et exportation}

Le résultat de la fonction \texttt{ggplot()} pouvant être stocké dans un
objet R, il est possible de le sauvegarder tel quel avec \texttt{save()}
ou \texttt{saveRDS()} et de le réutiliser par la suite dans R.

\begin{Shaded}
\begin{Highlighting}[]
\NormalTok{g <-}\StringTok{ }\KeywordTok{ggplot}\NormalTok{(mpg, }\KeywordTok{aes}\NormalTok{(displ, hwy)) +}\StringTok{ }\KeywordTok{geom_point}\NormalTok{()}
\KeywordTok{saveRDS}\NormalTok{(g, }\DataTypeTok{file =} \StringTok{"g.rds"}\NormalTok{)}
\end{Highlighting}
\end{Shaded}

\pause La fonction \texttt{ggsave()} simplifie l'export de graphiques en
dehors de R. Par défaut, elle sauvegarde le dernier graphique produit.

\begin{Shaded}
\begin{Highlighting}[]
\NormalTok{g +}\StringTok{ }\KeywordTok{geom_smooth}\NormalTok{(}\DataTypeTok{method =} \StringTok{"lm"}\NormalTok{, }\DataTypeTok{se =} \OtherTok{FALSE}\NormalTok{)}
\KeywordTok{ggsave}\NormalTok{(}\StringTok{"monGraphique.pdf"}\NormalTok{)}
\KeywordTok{ggsave}\NormalTok{(}\StringTok{"monGraphique.png"}\NormalTok{)}
\end{Highlighting}
\end{Shaded}

\end{frame}

\section{Générer automatiquement des documents depuis
R}\label{generer-automatiquement-des-documents-depuis-r}

\subsection*{Générer automatiquement des documents depuis R}

\begin{frame}{\large Pourquoi générer automatiquement des documents ?}

\begin{itemize}
\item
  Exporter et documenter des \textbf{traitements} en vue d'une
  réutilisation future : statistiques pour une étude, traitements
  réalisés lors d'une réunion de travail, etc.

  \vspace{0.2cm}

  \small  \textbf{Remarque} Utilisation analogue à celle permise par les
  instructions \textcolor{blue}{\texttt{ODS RTF}} ou
  \textcolor{blue}{\texttt{ODS PDF}} de SAS.
\end{itemize}

\pause \normalsize 

\begin{itemize}
\tightlist
\item
  Construire des \textbf{rapports complets et automatisés} pour des
  tâches répétitives : rapports d'utilisation, tests de la cohérence ou
  de la qualité de nouvelles données, etc.
\end{itemize}

\pause \vspace{0.2cm}

\begin{itemize}
\tightlist
\item
  Produire des publications \textbf{reproductibles} sur différents
  supports : notes, documentation, articles de revues, etc.
\end{itemize}

\end{frame}

\begin{frame}[fragile]{\large Principe de la génération automatique de
documents}

La génération automatique de documents complets repose sur deux éléments
:

\begin{enumerate}
\def\labelenumi{\arabic{enumi}.}
\item
  Articuler le code, les résultats et le commentaire dans un
  \textbf{même document} : garantir la cohérence et faciliter les mises
  à jour;
\item
  Formater de façon standardisée le document vers \textbf{plusieurs
  sorties} : \texttt{.html}, \texttt{.pdf}, \texttt{.docx},
  \texttt{.odt}.
\end{enumerate}

\pause \small \centering

\begin{tikzpicture}[auto]
    \node [input] (code) {Code R};
    \node [input, right= of code] (tableaux) {Tableaux, graphiques};
    \node [input, right= of tableaux] (commentaire) {Commentaire};
    \node [output, below of=tableaux, node distance = 2cm] (output) {\texttt{.pdf} \texttt{.html} \texttt{.docx} \texttt{.odt}};
    \path [line] (code) -- (output);
    \path [line] (tableaux) -- (output);
    \path [line] (commentaire) -- (output);
\end{tikzpicture}

\end{frame}

\begin{frame}[fragile]{\large Etapes de la génération automatique de
documents}

\small
\centering

\begin{tikzpicture}[auto]
    \draw[draw, red, very thick, dotted, visible on =<8->] ($(code.north west)+(-0.2,0.9)$)  rectangle ($(commentaire.south east) +(0.2,-3.3)$) node[] at (3.3, 0.9) {\href{https://blog.rstudio.org/2014/06/18/r-markdown-v2/}{\underline{R Markdown v2}} (\texttt{.Rmd})}; 
    \node [input] (code) {Code R};
    \node [input, right= of code] (tableaux) {Tableaux, graphiques};
    \node [input, right= of tableaux] (commentaire) {Commentaire};
    \node [block, below of=tableaux, node distance = 1.5cm, visible on =<2->] (knitr) {\href{http://yihui.name/knitr/}{\underline{knitr}}};
    \node [block, below of=knitr, node distance = 1.5cm, visible on =<3->] (pandoc) {\href{http://rmarkdown.rstudio.com/authoring_pandoc_markdown.html}{\underline{pandoc}}};
    \node [output, below of=pandoc, node distance = 1.5cm, visible on =<5->] (docx) {\texttt{.docx} \texttt{.odt}};
    \node [output, left of=docx, node distance = 3cm, visible on =<4->] (html) {\texttt{.html}};
    \node [output, right of=docx, node distance = 3cm, visible on =<6->] (pdf) {\texttt{.pdf}};
    \node [block, right of=pandoc, node distance = 3cm, visible on =<7->] (latex) {\href{https://www.latex-project.org/}{\underline{LaTeX}}};
    \path [line, visible on =<2->] (code) -- (knitr);
    \path [line, visible on =<2->] (tableaux) -- (knitr);
    \path [line, visible on =<2->] (commentaire) -- (knitr);
    \path [line, visible on =<3->] (knitr) -- (pandoc) node[pos=0.50] {\texttt{.md}};
    \path [line, visible on =<4->] (pandoc) -- (html);
    \path [line, visible on =<5->] (pandoc) -- (docx);
    \path [line, visible on =<7->] (pandoc) -- (latex) node[pos=0.50] {\texttt{.tex}};
    \path [line, visible on =<7->] (latex) -- (pdf);
    \mode<beamer>{\path [visible on =<6>] (pandoc) -- (pdf)  node[pos=0.50] {?};}
\end{tikzpicture}

\raggedright
\pause[9] \textbf{Note} \texttt{rmarkdown} et \texttt{knitr} sont des
\emph{packages} R (avec plusieurs dépendances); pandoc et LaTeX sont des
programmes autonomes.

\end{frame}

\begin{frame}[fragile]{Préparer et tester l'environnement de travail}

\begin{enumerate}
\def\labelenumi{\arabic{enumi}.}
\tightlist
\item
  Travailler sous RStudio

  \begin{itemize}
  \tightlist
  \item
    RStudio facilite l'édition et la compilation de fichier
    \texttt{.Rmd};
  \item
    pandoc est embarqué par défaut dans RStudio.
  \end{itemize}
\end{enumerate}

\pause \bigskip 

\begin{enumerate}
\def\labelenumi{\arabic{enumi}.}
\setcounter{enumi}{1}
\tightlist
\item
  Installer les \emph{packages} nécessaires

  \begin{itemize}
  \tightlist
  \item
    installer le \emph{package} \texttt{rmarkdown} et ses dépendances;
  \item
    installer le \emph{package} \texttt{knitr} et ses dépendances.
  \end{itemize}
\end{enumerate}

\pause \bigskip 

\begin{enumerate}
\def\labelenumi{\arabic{enumi}.}
\setcounter{enumi}{2}
\tightlist
\item
  Pour produire des fichiers \texttt{.pdf}, installer LaTeX
  (\href{https://miktex.org/}{\underline{MiKTeX}} sous Windows) et
  \href{http://superuser.com/questions/341192/how-can-i-display-the-contents-of-an-environment-variable-from-the-command-promp}{\underline{s'assurer}}
  que ses programmes figurent dans le \emph{path} de Windows.
\end{enumerate}

\pause \bigskip 

\begin{enumerate}
\def\labelenumi{\arabic{enumi}.}
\setcounter{enumi}{3}
\tightlist
\item
  Créer un nouveau fichier R Markdown (\texttt{.Rmd}), installer les
  \emph{packages} complémentaires demandés, choisir le type de document
  et compiler le fichier d'exemple (\texttt{Ctrl\ +\ K}).
\end{enumerate}

\end{frame}

\begin{frame}[fragile]{Ecrire du texte dans R Markdown}

Pour écrire du texte dans un document R Markdown, il suffit de le
\textbf{taper dans le fichier \texttt{.Rmd}} (sans le commenter ni
l'échapper d'aucune manière).

\pause Des \textbf{balises} spéciales permettent de mettre en forme le
document :

\begin{itemize}
\tightlist
\item
  les signes \texttt{*} et \texttt{\_} permettent de mettre des mots en
  \texttt{*italique*} ou en \texttt{**gras**};
\item
  les six niveaux de titres sont préfixés par les signes \texttt{\#}
  (premier niveau), \texttt{\#\#} (deuxième niveau), etc.
\item
  des listes sont automatiquement créées à partir de successions de
  \texttt{-} ou de séquences de nombres ou de lettres séparées par un
  retour à la ligne.
\end{itemize}

\pause 

\textbf{Note} Pour une présentation synthétique de R Markdown, se
référer à
l'\href{https://www.rstudio.com/wp-content/uploads/2015/02/rmarkdown-cheatsheet.pdf}{\underline{aide-mémoire}}
(\emph{cheat sheet}) sur le site de RStudio.

\end{frame}

\begin{frame}[fragile]{Ecrire du code dans R Markdown}

Les blocs de code R sont intégrés dans R Markdown de la façon suivante :

\begin{verbatim}
```{r}
2 + 2
```
\end{verbatim}

\pause Par défaut \textbf{le code est évalué}, et \textbf{lui-même ainsi
que ses résultats sont affichés} dans le document en sortie :

\begin{Shaded}
\begin{Highlighting}[]
\DecValTok{2} \NormalTok{+}\StringTok{ }\DecValTok{2}
\end{Highlighting}
\end{Shaded}

\begin{verbatim}
  ## [1] 4
\end{verbatim}

\end{frame}

\begin{frame}[fragile]{Ecrire du code dans R Markdown}

Les \textbf{options} saisies en début de bloc permettent de préciser à
\texttt{knitr} la manière de le prendre en compte, par exemple:

\begin{itemize}
\tightlist
\item
  \texttt{eval=FALSE} : le bloc n'est pas évalué;
\item
  \texttt{echo=FALSE} : le bloc n'est pas affiché;
\item
  \texttt{collapse=TRUE} : code et résultats sont affichés à la suite.
\end{itemize}

\pause 

\begin{verbatim}
```{r, echo=FALSE}
2 + 2
```
\end{verbatim}

\begin{verbatim}
  ## [1] 4
\end{verbatim}

\pause 

\textbf{Note} Toutes les options de \texttt{knitr} relatives aux blocs
de code (\emph{chunk options}) sont présentées sur la
\href{http://yihui.name/knitr/options/}{\underline{page}} du créateur du
\emph{package}, Yihui Xie.

\end{frame}

\begin{frame}[fragile]{Ecrire du code dans R Markdown}

Il est également possible d'intégrer le résultat d'un traitement R dans
le corps d'un paragraphe avec la syntaxe :

\begin{verbatim}
`r   `
\end{verbatim}

\pause 

\textbf{Exemple} Pour intégrer dans le texte la date de compilation du
document, utiliser

\begin{verbatim}
Document compilé le `r Sys.Date()`.
\end{verbatim}

\pause Document compilé le 2017-06-21.

\end{frame}

\begin{frame}[fragile]{Intégrer des graphiques dans R Markdown}

Tous les graphiques produits par les blocs de code sont
\textbf{automatiquement intégrés au fichier final}.

\pause Un \textbf{grand nombre d'options} sont consacrées au paramétrage
des graphiques, notamment :

\begin{itemize}
\tightlist
\item
  \texttt{fig.width}, \texttt{fig.height} : largeur et hauteur utilisées
  pour produire le graphique, en pouces;
\item
  \texttt{fig.asp} : rapport hauteur/largeur (\texttt{fig.height} est
  neutralisé quand \texttt{fig.asp} est renseigné);
\item
  \texttt{out.width}, \texttt{out.height} : largeur et hauteur du
  graphique dans la sortie finale;
\item
  \texttt{fig.align} : alignement du grahique (\texttt{"left"},
  \texttt{"right"} ou \texttt{"center"});
\item
  \texttt{dpi} (72 par défaut) : résolution (utile uniquement pour
  HTML).
\end{itemize}

\end{frame}

\begin{frame}[fragile]{Intégrer des graphiques dans R Markdown}

\footnotesize \center

\begin{verbatim}
```{r, fig.asp = 3/4, fig.width = 4}
plot(mpg$displ, mpg$hwy)
```
\end{verbatim}

\includegraphics{presentation_files/figure-beamer/unnamed-chunk-145-1.pdf}

\end{frame}

\begin{frame}[fragile]{Intégrer des graphiques dans R Markdown}

\footnotesize \center

\begin{verbatim}
```{r, fig.asp = 3/4, fig.width = 6, out.width = "4in"}
plot(mpg$displ, mpg$hwy)
```
\end{verbatim}

\includegraphics[width=4in]{presentation_files/figure-beamer/unnamed-chunk-147-1}

\end{frame}

\begin{frame}[fragile]{Intégrer des tableaux dans R Markdown}

Pour construire un tableau dans R Markdown, il suffit de le \og dessiner
\fg{} avec les signes \texttt{-} et \texttt{\textbar{}} :

\begin{verbatim}
Colonne 1 | Colonne 2 | Colonne 3
--------: | :-------- | :-------:
1         | a         | `TRUE`
2         | b         | `FALSE`
\end{verbatim}

\pause 

\begin{longtable}[]{@{}rlc@{}}
\toprule
Colonne 1 & Colonne 2 & Colonne 3\tabularnewline
\midrule
\endhead
1 & a & \texttt{TRUE}\tabularnewline
2 & b & \texttt{FALSE}\tabularnewline
\bottomrule
\end{longtable}

Les \texttt{:} permettent de spécifier l'alignement des colonnes.

\end{frame}

\begin{frame}[fragile]{Intégrer des tableaux dans R Markdown}

En règle générale cependant, les tableaux à intégrer sont générés
automatiquement à partir des données.

\footnotesize

\begin{verbatim}
```{r}
resultat <- data.table(mpg)[
    , list(hwy=mean(hwy), cty=mean(cty)), by = drv
]
resultat
```
\end{verbatim}

\vspace{-5mm}

\begin{verbatim}
  ##    drv      hwy     cty
  ## 1:   f 28.16038 19.9717
  ## 2:   4 19.17476 14.3301
  ## 3:   r 21.00000 14.0800
\end{verbatim}

\pause \normalsize

La fonction \texttt{knitr::kable()} permet de \textbf{transformer un
objet R en tableau formaté pour R Markdown}.

\end{frame}

\begin{frame}[fragile]{Intégrer des tableaux dans R Markdown}

\footnotesize 

\vspace{-2mm}

\begin{verbatim}
```{r, results = "asis"}
knitr::kable(resultat)
```
\end{verbatim}

\pause \vspace{-10mm}

\begin{verbatim}
|drv |      hwy|     cty|
|:---|--------:|-------:|
|f   | 28.16038| 19.9717|
|4   | 19.17476| 14.3301|
|r   | 21.00000| 14.0800|
\end{verbatim}

\pause \vspace{-1mm}

Ce qui donne une fois formaté par R Markdown:

\begin{longtable}[]{@{}lrr@{}}
\toprule
drv & hwy & cty\tabularnewline
\midrule
\endhead
f & 28.16038 & 19.9717\tabularnewline
4 & 19.17476 & 14.3301\tabularnewline
r & 21.00000 & 14.0800\tabularnewline
\bottomrule
\end{longtable}

\end{frame}

\begin{frame}[fragile]{Paramétrer un document R Markdown}

La plupart des paramètres généraux du documents sont à indiquer dans son
en-tête (désigné par l'acronyme YAML) :

\footnotesize

\begin{verbatim}
---
title: "Formation R Perfectionnement"
author: "Martin Chevalier (Insee)"
output:
  html_document:
    highlight: haddock
    toc: yes
    toc_depth: 2
    toc_float: yes
---
\end{verbatim}

\pause \normalsize

\textbf{Pour en savoir plus} Le site de RStudio documente le paramétrage
de l'en-tête YAML selon les formats de sortie souhaités
(\href{http://rmarkdown.rstudio.com/html_document_format.html}{\underline{\texttt{html}}},
\href{http://rmarkdown.rstudio.com/pdf_document_format.html}{\underline{\texttt{pdf}}}).

\end{frame}

\end{document}
