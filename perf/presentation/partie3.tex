\documentclass[12pt,ignorenonframetext,]{beamer}
\setbeamertemplate{caption}[numbered]
\setbeamertemplate{caption label separator}{: }
\setbeamercolor{caption name}{fg=normal text.fg}
\beamertemplatenavigationsymbolsempty
\usepackage{lmodern}
\usepackage{amssymb,amsmath}
\usepackage{ifxetex,ifluatex}
\usepackage{fixltx2e} % provides \textsubscript
\ifnum 0\ifxetex 1\fi\ifluatex 1\fi=0 % if pdftex
  \usepackage[T1]{fontenc}
  \usepackage[utf8]{inputenc}
\else % if luatex or xelatex
  \ifxetex
    \usepackage{mathspec}
  \else
    \usepackage{fontspec}
  \fi
  \defaultfontfeatures{Ligatures=TeX,Scale=MatchLowercase}
\fi
% use upquote if available, for straight quotes in verbatim environments
\IfFileExists{upquote.sty}{\usepackage{upquote}}{}
% use microtype if available
\IfFileExists{microtype.sty}{%
\usepackage{microtype}
\UseMicrotypeSet[protrusion]{basicmath} % disable protrusion for tt fonts
}{}
\newif\ifbibliography
\hypersetup{
            pdftitle={Formation R Perfectionnement},
            pdfborder={0 0 0},
            breaklinks=true}
\urlstyle{same}  % don't use monospace font for urls
\usepackage{color}
\usepackage{fancyvrb}
\newcommand{\VerbBar}{|}
\newcommand{\VERB}{\Verb[commandchars=\\\{\}]}
\DefineVerbatimEnvironment{Highlighting}{Verbatim}{commandchars=\\\{\}}
% Add ',fontsize=\small' for more characters per line
\newenvironment{Shaded}{}{}
\newcommand{\KeywordTok}[1]{\textcolor[rgb]{0.00,0.00,1.00}{#1}}
\newcommand{\DataTypeTok}[1]{#1}
\newcommand{\DecValTok}[1]{#1}
\newcommand{\BaseNTok}[1]{#1}
\newcommand{\FloatTok}[1]{#1}
\newcommand{\ConstantTok}[1]{#1}
\newcommand{\CharTok}[1]{\textcolor[rgb]{0.00,0.50,0.50}{#1}}
\newcommand{\SpecialCharTok}[1]{\textcolor[rgb]{0.00,0.50,0.50}{#1}}
\newcommand{\StringTok}[1]{\textcolor[rgb]{0.00,0.50,0.50}{#1}}
\newcommand{\VerbatimStringTok}[1]{\textcolor[rgb]{0.00,0.50,0.50}{#1}}
\newcommand{\SpecialStringTok}[1]{\textcolor[rgb]{0.00,0.50,0.50}{#1}}
\newcommand{\ImportTok}[1]{#1}
\newcommand{\CommentTok}[1]{\textcolor[rgb]{0.00,0.50,0.00}{#1}}
\newcommand{\DocumentationTok}[1]{\textcolor[rgb]{0.00,0.50,0.00}{#1}}
\newcommand{\AnnotationTok}[1]{\textcolor[rgb]{0.00,0.50,0.00}{#1}}
\newcommand{\CommentVarTok}[1]{\textcolor[rgb]{0.00,0.50,0.00}{#1}}
\newcommand{\OtherTok}[1]{\textcolor[rgb]{1.00,0.25,0.00}{#1}}
\newcommand{\FunctionTok}[1]{#1}
\newcommand{\VariableTok}[1]{#1}
\newcommand{\ControlFlowTok}[1]{\textcolor[rgb]{0.00,0.00,1.00}{#1}}
\newcommand{\OperatorTok}[1]{#1}
\newcommand{\BuiltInTok}[1]{#1}
\newcommand{\ExtensionTok}[1]{#1}
\newcommand{\PreprocessorTok}[1]{\textcolor[rgb]{1.00,0.25,0.00}{#1}}
\newcommand{\AttributeTok}[1]{#1}
\newcommand{\RegionMarkerTok}[1]{#1}
\newcommand{\InformationTok}[1]{\textcolor[rgb]{0.00,0.50,0.00}{#1}}
\newcommand{\WarningTok}[1]{\textcolor[rgb]{0.00,0.50,0.00}{\textbf{#1}}}
\newcommand{\AlertTok}[1]{\textcolor[rgb]{1.00,0.00,0.00}{#1}}
\newcommand{\ErrorTok}[1]{\textcolor[rgb]{1.00,0.00,0.00}{\textbf{#1}}}
\newcommand{\NormalTok}[1]{#1}
\usepackage{longtable,booktabs}
\usepackage{caption}
% These lines are needed to make table captions work with longtable:
\makeatletter
\def\fnum@table{\tablename~\thetable}
\makeatother
\usepackage{graphicx,grffile}
\makeatletter
\def\maxwidth{\ifdim\Gin@nat@width>\linewidth\linewidth\else\Gin@nat@width\fi}
\def\maxheight{\ifdim\Gin@nat@height>\textheight0.8\textheight\else\Gin@nat@height\fi}
\makeatother
% Scale images if necessary, so that they will not overflow the page
% margins by default, and it is still possible to overwrite the defaults
% using explicit options in \includegraphics[width, height, ...]{}
\setkeys{Gin}{width=\maxwidth,height=\maxheight,keepaspectratio}

% Prevent slide breaks in the middle of a paragraph:
\widowpenalties 1 10000
\raggedbottom

\AtBeginPart{
  \let\insertpartnumber\relax
  \let\partname\relax
  \frame{\partpage}
}
\AtBeginSection{
  \ifbibliography
  \else
    \let\insertsectionnumber\relax
    \let\sectionname\relax
    \frame{\sectionpage}
  \fi
}
\AtBeginSubsection{
  \let\insertsubsectionnumber\relax
  \let\subsectionname\relax
  \frame{\subsectionpage}
}

\setlength{\parindent}{0pt}
\setlength{\parskip}{6pt plus 2pt minus 1pt}
\setlength{\emergencystretch}{3em}  % prevent overfull lines
\providecommand{\tightlist}{%
  \setlength{\itemsep}{0pt}\setlength{\parskip}{0pt}}
\setcounter{secnumdepth}{0}

\usepackage[french]{babel}
\usepackage{lmodern}
\usepackage{graphicx}
\usepackage{xcolor}
\usepackage{textcomp} 
\usepackage{amsmath, amsfonts, amssymb, amsthm}
\usepackage{booktabs,multirow}
\usepackage{setspace}
\usepackage{float}
\usepackage{pgfpages}
\usepackage{colortbl}
\usepackage{epstopdf}
\usepackage{framed}



\definecolor{shadecolor}{RGB}{248,248,248}
\definecolor{grayInsee}{HTML}{5a5758}
\definecolor{redInsee}{HTML}{ed1443}

%Everything about the notes and formatting of the notepage
\setbeamertemplate{note page}{%
	Notes personnelles
	\insertnote%
}


\setbeamertemplate{navigation symbols}{}
\usetheme{default} %Malmoe not bad
\setbeamertemplate{footline}[frame number]


%\setbeamerfont{frametitle}{size=\normalsize}
%\setbeamerfont{framesubtitle}{size=\Large}
%\setbeamercolor{frametitle}{fg=grayInsee}
%\setbeamercolor{framesubtitle}{fg=redInsee}
\setbeamercolor{title}{fg=grayInsee}
\setbeamercolor{subsection in toc}{fg=grayInsee}
\setbeamertemplate{frametitle}{%
	\large \textcolor{grayInsee}{\subsecname}
	\\ \vspace{0.1cm} \Large \textcolor{redInsee}{\insertframetitle}
}
%\setbeamertemplate{frametitle}{%
%	\large \textcolor{grayInsee}{
%		\ifx\intertsubsection\emptyset
%			\secname \\ \vspace{0.1cm} 
%		\else 
%			\subsecname \\ \vspace{0.1cm}
%		\fi
%	}
%	\Large \textcolor{redInsee}{\insertframetitle}
%}
\setbeamercolor{local structure}{fg=redInsee}

\AtBeginSection[]
{\ifnum \thesection>1
  \begin{frame}
  \vfill
  \begin{center}
  \LARGE
  \textcolor{grayInsee}{\insertsectionhead}
  \end{center}
  \vfill
  \end{frame}
\else
\fi
}

\AtBeginSubsection[]{}

\title{\Large Formation \textbf{R} Perfectionnement}

\institute{ \includegraphics[height = 2.5cm]{../figures/Logo_Insee.png}\\ ~ \\ \normalsize Martin \textsc{Chevalier} (Insee)}

\author{Drees -- 16-17 avril 2018}

\date{}

\renewenvironment{Shaded}{\begin{snugshade}}{\end{snugshade}}

\newcommand{\aparte}[2]{
	{\small\textsf{\textbf{#1} #2}}
}

\newcommand{\intertitre}[1]{\textcolor{redInsee}{\textbf{#1}}}

%\usepackage{enumitem}
%\setlist{nolistsep}

\usepackage{tikz}
\usetikzlibrary{shapes,arrows,calc, positioning}
\tikzstyle{input} = [draw, rectangle,rounded corners, text width=2.5cm, fill=green!20, node distance=0.5cm, minimum height=2em, text centered]
\tikzstyle{output} = [draw, ellipse,fill=red!20, node distance=0.5cm, minimum height=2em, text centered]
\tikzstyle{block} = [rectangle, draw, fill=blue!20, 
    text width=1.5cm, text centered, minimum height=2em, node distance = 0.5cm]
\tikzstyle{line} = [draw, -latex', shorten >=2pt, shorten <=2pt]
\tikzset{
  invisible/.style={opacity=0},
  visible on/.style={alt={#1{}{invisible}}},
  alt/.code args={<#1>#2#3}{%
    \alt<#1>{\pgfkeysalso{#2}}{\pgfkeysalso{#3}} % \pgfkeysalso doesn't change the path
  },
}

%\usepackage{pgfpages}
%\mode<handout>{
%	%\setbeamercolor{background canvas}{bg=black!20}
%	\pgfpagesuselayout{2 on 1}[border shrink=2mm]
%}

\title{Formation R Perfectionnement}
\date{}

\begin{document}
\frame{\titlepage}

\section{Générer automatiquement des documents depuis
R}\label{generer-automatiquement-des-documents-depuis-r}

\subsection*{Générer automatiquement des documents depuis R}

\begin{frame}{\large Pourquoi générer automatiquement des documents ?}

\begin{itemize}
\item
  Exporter et documenter des \textbf{traitements} en vue d'une
  réutilisation future : statistiques pour une étude, traitements
  réalisés lors d'une réunion de travail, etc.

  \vspace{0.2cm} \small  \textbf{Remarque} Utilisation analogue à celle
  permise par les instructions \textcolor{blue}{\texttt{ODS RTF}} ou
  \textcolor{blue}{\texttt{ODS PDF}} de SAS.
\end{itemize}

\pause \normalsize 

\begin{itemize}
\tightlist
\item
  Construire des \textbf{rapports complets et automatisés} pour des
  tâches répétitives : rapports d'utilisation, tests de la cohérence ou
  de la qualité de nouvelles données, etc.
\end{itemize}

\pause \vspace{0.2cm}

\begin{itemize}
\tightlist
\item
  Produire des publications \textbf{reproductibles} sur différents
  supports : notes, documentation, articles de revues, etc.
\end{itemize}

\end{frame}

\begin{frame}[fragile]{\large Principe de la génération automatique de
documents}

La génération automatique de documents complets repose sur deux éléments
:

\begin{enumerate}
\def\labelenumi{\arabic{enumi}.}
\item
  Articuler le code, les résultats et le commentaire dans un
  \textbf{même document} : garantir la cohérence et faciliter les mises
  à jour;
\item
  Formater de façon standardisée le document vers \textbf{plusieurs
  sorties} : \texttt{.html}, \texttt{.pdf}, \texttt{.docx},
  \texttt{.odt}.
\end{enumerate}

\pause \small \centering

\begin{tikzpicture}[auto]
    \node [input] (code) {Code R};
    \node [input, right= of code] (tableaux) {Tableaux, graphiques};
    \node [input, right= of tableaux] (commentaire) {Commentaire};
    \node [output, below of=tableaux, node distance = 2cm] (output) {\texttt{.pdf} \texttt{.html} \texttt{.docx} \texttt{.odt}};
    \path [line] (code) -- (output);
    \path [line] (tableaux) -- (output);
    \path [line] (commentaire) -- (output);
\end{tikzpicture}

\end{frame}

\begin{frame}[fragile]{\large Etapes de la génération automatique de
documents}

\small
\centering

\begin{tikzpicture}[auto]
    \draw[draw, red, very thick, dotted, visible on =<8->] ($(code.north west)+(-0.2,0.9)$)  rectangle ($(commentaire.south east) +(0.2,-3.3)$) node[] at (3.3, 0.9) {\href{https://blog.rstudio.org/2014/06/18/r-markdown-v2/}{\underline{R Markdown v2}} (\texttt{.Rmd})}; 
    \node [input] (code) {Code R};
    \node [input, right= of code] (tableaux) {Tableaux, graphiques};
    \node [input, right= of tableaux] (commentaire) {Commentaire};
    \node [block, below of=tableaux, node distance = 1.5cm, visible on =<2->] (knitr) {\href{http://yihui.name/knitr/}{\underline{knitr}}};
    \node [block, below of=knitr, node distance = 1.5cm, visible on =<3->] (pandoc) {\href{http://rmarkdown.rstudio.com/authoring_pandoc_markdown.html}{\underline{pandoc}}};
    \node [output, below of=pandoc, node distance = 1.5cm, visible on =<5->] (docx) {\texttt{.docx} \texttt{.odt}};
    \node [output, left of=docx, node distance = 3cm, visible on =<4->] (html) {\texttt{.html}};
    \node [output, right of=docx, node distance = 3cm, visible on =<6->] (pdf) {\texttt{.pdf}};
    \node [block, right of=pandoc, node distance = 3cm, visible on =<7->] (latex) {\href{https://www.latex-project.org/}{\underline{LaTeX}}};
    \path [line, visible on =<2->] (code) -- (knitr);
    \path [line, visible on =<2->] (tableaux) -- (knitr);
    \path [line, visible on =<2->] (commentaire) -- (knitr);
    \path [line, visible on =<3->] (knitr) -- (pandoc) node[pos=0.50] {\texttt{.md}};
    \path [line, visible on =<4->] (pandoc) -- (html);
    \path [line, visible on =<5->] (pandoc) -- (docx);
    \path [line, visible on =<7->] (pandoc) -- (latex) node[pos=0.50] {\texttt{.tex}};
    \path [line, visible on =<7->] (latex) -- (pdf);
    \mode<beamer>{\path [visible on =<6>] (pandoc) -- (pdf)  node[pos=0.50] {?};}
\end{tikzpicture}

\raggedright
\pause[9] \intertitre{Note} \texttt{rmarkdown} et \texttt{knitr} sont
des \emph{packages} R (avec plusieurs dépendances); pandoc et LaTeX sont
des programmes autonomes.

\end{frame}

\begin{frame}[fragile]{Préparer et tester l'environnement de travail}

\begin{enumerate}
\def\labelenumi{\arabic{enumi}.}
\tightlist
\item
  Travailler sous RStudio

  \begin{itemize}
  \tightlist
  \item
    RStudio facilite l'édition et la compilation de fichier
    \texttt{.Rmd};
  \item
    pandoc est embarqué par défaut dans RStudio.
  \end{itemize}
\end{enumerate}

\pause \bigskip 

\begin{enumerate}
\def\labelenumi{\arabic{enumi}.}
\setcounter{enumi}{1}
\tightlist
\item
  Installer les \emph{packages} nécessaires

  \begin{itemize}
  \tightlist
  \item
    installer le \emph{package} \texttt{rmarkdown} et ses dépendances;
  \item
    installer le \emph{package} \texttt{knitr} et ses dépendances.
  \end{itemize}
\end{enumerate}

\pause \bigskip 

\begin{enumerate}
\def\labelenumi{\arabic{enumi}.}
\setcounter{enumi}{2}
\tightlist
\item
  Pour produire des fichiers \texttt{.pdf}, installer LaTeX
  (\href{https://miktex.org/}{\underline{MiKTeX}} sous Windows) et
  \href{http://superuser.com/questions/341192/how-can-i-display-the-contents-of-an-environment-variable-from-the-command-promp}{\underline{s'assurer}}
  que ses programmes figurent dans le \emph{path} de Windows.
\end{enumerate}

\pause \bigskip 

\begin{enumerate}
\def\labelenumi{\arabic{enumi}.}
\setcounter{enumi}{3}
\tightlist
\item
  Créer un nouveau fichier R Markdown (\texttt{.Rmd}), installer les
  \emph{packages} complémentaires demandés, choisir le type de document
  et compiler le fichier d'exemple (\texttt{Ctrl\ +\ K}).
\end{enumerate}

\end{frame}

\begin{frame}[fragile]{Ecrire du texte dans R Markdown}

Pour écrire du texte dans un document R Markdown, il suffit de le
\textbf{taper dans le fichier \texttt{.Rmd}} (sans le commenter ni
l'échapper d'aucune manière).

\pause Des \textbf{balises} spéciales permettent de mettre en forme le
document :

\begin{itemize}
\tightlist
\item
  les signes \texttt{*} et \texttt{\_} permettent de mettre des mots en
  \texttt{*italique*} ou en \texttt{**gras**};
\item
  les six niveaux de titres sont préfixés par les signes \texttt{\#}
  (premier niveau), \texttt{\#\#} (deuxième niveau), etc.
\item
  des listes sont automatiquement créées à partir de successions de
  \texttt{-} ou de séquences de nombres ou de lettres séparées par un
  retour à la ligne.
\end{itemize}

\pause 

\intertitre{Note} Pour une présentation synthétique de R Markdown, se
référer à
l'\href{https://www.rstudio.com/wp-content/uploads/2015/02/rmarkdown-cheatsheet.pdf}{\underline{aide-mémoire}}
(\emph{cheat sheet}) sur le site de RStudio.

\end{frame}

\begin{frame}[fragile]{Ecrire du code dans R Markdown}

Les blocs de code R sont intégrés dans R Markdown de la façon suivante :

\begin{verbatim}
```{r}
2 + 2
```
\end{verbatim}

\pause Par défaut \textbf{le code est évalué}, et \textbf{lui-même ainsi
que ses résultats sont affichés} dans le document en sortie :

\begin{Shaded}
\begin{Highlighting}[]
\DecValTok{2} \OperatorTok{+}\StringTok{ }\DecValTok{2}
\end{Highlighting}
\end{Shaded}

\begin{verbatim}
  ## [1] 4
\end{verbatim}

\end{frame}

\begin{frame}[fragile]{Ecrire du code dans R Markdown}

Les \textbf{options} saisies en début de bloc permettent de préciser à
\texttt{knitr} la manière de le prendre en compte, par exemple:

\begin{itemize}
\tightlist
\item
  \texttt{eval=FALSE} : le bloc n'est pas évalué;
\item
  \texttt{echo=FALSE} : le bloc n'est pas affiché;
\item
  \texttt{collapse=TRUE} : code et résultats sont affichés à la suite.
\end{itemize}

\pause 

\begin{verbatim}
```{r, echo=FALSE}
2 + 2
```
\end{verbatim}

\begin{verbatim}
  ## [1] 4
\end{verbatim}

\pause 

\intertitre{Note} Toutes les options de \texttt{knitr} relatives aux
blocs de code (\emph{chunk options}) sont présentées sur la
\href{http://yihui.name/knitr/options/}{\underline{page}} du créateur du
\emph{package}, Yihui Xie.

\end{frame}

\begin{frame}[fragile]{Ecrire du code dans R Markdown}

Il est également possible d'intégrer le résultat d'un traitement R dans
le corps d'un paragraphe avec la syntaxe :

\begin{verbatim}
`r   `
\end{verbatim}

\pause 

\intertitre{Exemple} Pour intégrer dans le texte la date de compilation
du document, utiliser

\begin{verbatim}
Document compilé le `r Sys.Date()`.
\end{verbatim}

\pause Document compilé le 2018-04-10.

\end{frame}

\begin{frame}[fragile]{Intégrer des graphiques dans R Markdown}

Tous les graphiques produits par les blocs de code sont
\textbf{automatiquement intégrés au fichier final}.

\pause Un \textbf{grand nombre d'options} sont consacrées au paramétrage
des graphiques, notamment :

\begin{itemize}
\tightlist
\item
  \texttt{fig.width}, \texttt{fig.height} : largeur et hauteur utilisées
  pour produire le graphique, en pouces;
\item
  \texttt{fig.asp} : rapport hauteur/largeur (\texttt{fig.height} est
  neutralisé quand \texttt{fig.asp} est renseigné);
\item
  \texttt{out.width}, \texttt{out.height} : largeur et hauteur du
  graphique dans la sortie finale;
\item
  \texttt{fig.align} : alignement du grahique (\texttt{"left"},
  \texttt{"right"} ou \texttt{"center"});
\item
  \texttt{dpi} (72 par défaut) : résolution (utile uniquement pour
  HTML).
\end{itemize}

\end{frame}

\begin{frame}[fragile]{Intégrer des graphiques dans R Markdown}

\footnotesize \center

\begin{verbatim}
```{r, fig.asp = 3/4, fig.width = 4}
plot(mpg$displ, mpg$hwy)
```
\end{verbatim}

\includegraphics{partie3_files/figure-beamer/unnamed-chunk-9-1.pdf}

\end{frame}

\begin{frame}[fragile]{Intégrer des graphiques dans R Markdown}

\footnotesize \center

\begin{verbatim}
```{r, fig.asp = 3/4, fig.width = 6, out.width = "4in"}
plot(mpg$displ, mpg$hwy)
```
\end{verbatim}

\includegraphics[width=4in]{partie3_files/figure-beamer/unnamed-chunk-11-1}

\end{frame}

\begin{frame}[fragile]{Intégrer des tableaux dans R Markdown}

Pour construire un tableau dans R Markdown, il suffit de le \og dessiner
\fg{} avec les signes \texttt{-} et \texttt{\textbar{}} :

\begin{verbatim}
Colonne 1 | Colonne 2 | Colonne 3
--------: | :-------- | :-------:
1         | a         | `TRUE`
2         | b         | `FALSE`
\end{verbatim}

\pause 

\begin{longtable}[]{@{}rlc@{}}
\toprule
Colonne 1 & Colonne 2 & Colonne 3\tabularnewline
\midrule
\endhead
1 & a & \texttt{TRUE}\tabularnewline
2 & b & \texttt{FALSE}\tabularnewline
\bottomrule
\end{longtable}

Les \texttt{:} permettent de spécifier l'alignement des colonnes.

\end{frame}

\begin{frame}[fragile]{Intégrer des tableaux dans R Markdown}

En règle générale cependant, les tableaux à intégrer sont générés
automatiquement à partir des données.

\footnotesize

\begin{verbatim}
```{r}
resultat <- data.table(mpg)[
    , list(hwy=mean(hwy), cty=mean(cty)), by = drv
]
resultat
```
\end{verbatim}

\vspace{-5mm}

\begin{verbatim}
  ##    drv      hwy     cty
  ## 1:   f 28.16038 19.9717
  ## 2:   4 19.17476 14.3301
  ## 3:   r 21.00000 14.0800
\end{verbatim}

\pause \normalsize

La fonction \texttt{knitr::kable()} permet de \textbf{transformer un
objet R en tableau formaté pour R Markdown}.

\end{frame}

\begin{frame}[fragile]{Intégrer des tableaux dans R Markdown}

\footnotesize 

\vspace{-2mm}

\begin{verbatim}
```{r, results = "asis"}
knitr::kable(resultat)
```
\end{verbatim}

\pause \vspace{-10mm}

\begin{verbatim}
|drv |      hwy|     cty|
|:---|--------:|-------:|
|f   | 28.16038| 19.9717|
|4   | 19.17476| 14.3301|
|r   | 21.00000| 14.0800|
\end{verbatim}

\pause \vspace{-1mm}

Ce qui donne une fois formaté par R Markdown:

\begin{longtable}[]{@{}lrr@{}}
\toprule
drv & hwy & cty\tabularnewline
\midrule
\endhead
f & 28.16038 & 19.9717\tabularnewline
4 & 19.17476 & 14.3301\tabularnewline
r & 21.00000 & 14.0800\tabularnewline
\bottomrule
\end{longtable}

\end{frame}

\begin{frame}[fragile]{Paramétrer un document R Markdown}

La plupart des paramètres généraux du documents sont à indiquer dans son
en-tête (désigné par l'acronyme YAML) :

\footnotesize

\begin{verbatim}
---
title: "Formation R Perfectionnement"
author: "Martin Chevalier (Insee)"
output:
  html_document:
    highlight: haddock
    toc: yes
    toc_depth: 2
    toc_float: yes
---
\end{verbatim}

\pause \normalsize

\intertitre{Pour en savoir plus} Le site de RStudio documente le
paramétrage de l'en-tête YAML selon les formats de sortie souhaités
(\href{http://rmarkdown.rstudio.com/html_document_format.html}{\underline{\texttt{html}}},
\href{http://rmarkdown.rstudio.com/pdf_document_format.html}{\underline{\texttt{pdf}}}).

\end{frame}

\end{document}
