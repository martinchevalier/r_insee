\documentclass[12pt,ignorenonframetext,]{beamer}
\setbeamertemplate{caption}[numbered]
\setbeamertemplate{caption label separator}{: }
\setbeamercolor{caption name}{fg=normal text.fg}
\beamertemplatenavigationsymbolsempty
\usepackage{lmodern}
\usepackage{amssymb,amsmath}
\usepackage{ifxetex,ifluatex}
\usepackage{fixltx2e} % provides \textsubscript
\ifnum 0\ifxetex 1\fi\ifluatex 1\fi=0 % if pdftex
  \usepackage[T1]{fontenc}
  \usepackage[utf8]{inputenc}
\else % if luatex or xelatex
  \ifxetex
    \usepackage{mathspec}
  \else
    \usepackage{fontspec}
  \fi
  \defaultfontfeatures{Ligatures=TeX,Scale=MatchLowercase}
\fi
% use upquote if available, for straight quotes in verbatim environments
\IfFileExists{upquote.sty}{\usepackage{upquote}}{}
% use microtype if available
\IfFileExists{microtype.sty}{%
\usepackage{microtype}
\UseMicrotypeSet[protrusion]{basicmath} % disable protrusion for tt fonts
}{}
\newif\ifbibliography
\hypersetup{
            pdftitle={Formation R Perfectionnement},
            pdfborder={0 0 0},
            breaklinks=true}
\urlstyle{same}  % don't use monospace font for urls
\usepackage{color}
\usepackage{fancyvrb}
\newcommand{\VerbBar}{|}
\newcommand{\VERB}{\Verb[commandchars=\\\{\}]}
\DefineVerbatimEnvironment{Highlighting}{Verbatim}{commandchars=\\\{\}}
% Add ',fontsize=\small' for more characters per line
\newenvironment{Shaded}{}{}
\newcommand{\KeywordTok}[1]{\textcolor[rgb]{0.00,0.00,1.00}{#1}}
\newcommand{\DataTypeTok}[1]{#1}
\newcommand{\DecValTok}[1]{#1}
\newcommand{\BaseNTok}[1]{#1}
\newcommand{\FloatTok}[1]{#1}
\newcommand{\ConstantTok}[1]{#1}
\newcommand{\CharTok}[1]{\textcolor[rgb]{0.00,0.50,0.50}{#1}}
\newcommand{\SpecialCharTok}[1]{\textcolor[rgb]{0.00,0.50,0.50}{#1}}
\newcommand{\StringTok}[1]{\textcolor[rgb]{0.00,0.50,0.50}{#1}}
\newcommand{\VerbatimStringTok}[1]{\textcolor[rgb]{0.00,0.50,0.50}{#1}}
\newcommand{\SpecialStringTok}[1]{\textcolor[rgb]{0.00,0.50,0.50}{#1}}
\newcommand{\ImportTok}[1]{#1}
\newcommand{\CommentTok}[1]{\textcolor[rgb]{0.00,0.50,0.00}{#1}}
\newcommand{\DocumentationTok}[1]{\textcolor[rgb]{0.00,0.50,0.00}{#1}}
\newcommand{\AnnotationTok}[1]{\textcolor[rgb]{0.00,0.50,0.00}{#1}}
\newcommand{\CommentVarTok}[1]{\textcolor[rgb]{0.00,0.50,0.00}{#1}}
\newcommand{\OtherTok}[1]{\textcolor[rgb]{1.00,0.25,0.00}{#1}}
\newcommand{\FunctionTok}[1]{#1}
\newcommand{\VariableTok}[1]{#1}
\newcommand{\ControlFlowTok}[1]{\textcolor[rgb]{0.00,0.00,1.00}{#1}}
\newcommand{\OperatorTok}[1]{#1}
\newcommand{\BuiltInTok}[1]{#1}
\newcommand{\ExtensionTok}[1]{#1}
\newcommand{\PreprocessorTok}[1]{\textcolor[rgb]{1.00,0.25,0.00}{#1}}
\newcommand{\AttributeTok}[1]{#1}
\newcommand{\RegionMarkerTok}[1]{#1}
\newcommand{\InformationTok}[1]{\textcolor[rgb]{0.00,0.50,0.00}{#1}}
\newcommand{\WarningTok}[1]{\textcolor[rgb]{0.00,0.50,0.00}{\textbf{#1}}}
\newcommand{\AlertTok}[1]{\textcolor[rgb]{1.00,0.00,0.00}{#1}}
\newcommand{\ErrorTok}[1]{\textcolor[rgb]{1.00,0.00,0.00}{\textbf{#1}}}
\newcommand{\NormalTok}[1]{#1}

% Prevent slide breaks in the middle of a paragraph:
\widowpenalties 1 10000
\raggedbottom

\AtBeginPart{
  \let\insertpartnumber\relax
  \let\partname\relax
  \frame{\partpage}
}
\AtBeginSection{
  \ifbibliography
  \else
    \let\insertsectionnumber\relax
    \let\sectionname\relax
    \frame{\sectionpage}
  \fi
}
\AtBeginSubsection{
  \let\insertsubsectionnumber\relax
  \let\subsectionname\relax
  \frame{\subsectionpage}
}

\setlength{\parindent}{0pt}
\setlength{\parskip}{6pt plus 2pt minus 1pt}
\setlength{\emergencystretch}{3em}  % prevent overfull lines
\providecommand{\tightlist}{%
  \setlength{\itemsep}{0pt}\setlength{\parskip}{0pt}}
\setcounter{secnumdepth}{0}

\usepackage[french]{babel}
\usepackage{lmodern}
\usepackage{graphicx}
\usepackage{xcolor}
\usepackage{textcomp} 
\usepackage{amsmath, amsfonts, amssymb, amsthm}
\usepackage{booktabs,multirow}
\usepackage{setspace}
\usepackage{float}
\usepackage{pgfpages}
\usepackage{colortbl}
\usepackage{epstopdf}
\usepackage{framed}



\definecolor{shadecolor}{RGB}{248,248,248}
\definecolor{grayInsee}{HTML}{5a5758}
\definecolor{redInsee}{HTML}{ed1443}

%Everything about the notes and formatting of the notepage
\setbeamertemplate{note page}{%
	Notes personnelles
	\insertnote%
}


\setbeamertemplate{navigation symbols}{}
\usetheme{default} %Malmoe not bad
\setbeamertemplate{footline}[frame number]


%\setbeamerfont{frametitle}{size=\normalsize}
%\setbeamerfont{framesubtitle}{size=\Large}
%\setbeamercolor{frametitle}{fg=grayInsee}
%\setbeamercolor{framesubtitle}{fg=redInsee}
\setbeamercolor{title}{fg=grayInsee}
\setbeamercolor{subsection in toc}{fg=grayInsee}
\setbeamertemplate{frametitle}{%
	\large \textcolor{grayInsee}{\subsecname}
	\\ \vspace{0.1cm} \Large \textcolor{redInsee}{\insertframetitle}
}
%\setbeamertemplate{frametitle}{%
%	\large \textcolor{grayInsee}{
%		\ifx\intertsubsection\emptyset
%			\secname \\ \vspace{0.1cm} 
%		\else 
%			\subsecname \\ \vspace{0.1cm}
%		\fi
%	}
%	\Large \textcolor{redInsee}{\insertframetitle}
%}
\setbeamercolor{local structure}{fg=redInsee}

\AtBeginSection[]
{\ifnum \thesection>1
  \begin{frame}
  \vfill
  \begin{center}
  \LARGE
  \textcolor{grayInsee}{\insertsectionhead}
  \end{center}
  \vfill
  \end{frame}
\else
\fi
}

\AtBeginSubsection[]{}

\title{\Large Formation \textbf{R} Perfectionnement}

\institute{ \includegraphics[height = 2.5cm]{../figures/Logo_Insee.png}\\ ~ \\ \normalsize Martin \textsc{Chevalier} (Insee)}

\author{15-16 janvier 2018}

\date{}

\renewenvironment{Shaded}{\begin{snugshade}}{\end{snugshade}}

\newcommand{\aparte}[2]{
	{\small\textsf{\textbf{#1} #2}}
}

%\usepackage{enumitem}
%\setlist{nolistsep}

\usepackage{tikz}
\usetikzlibrary{shapes,arrows,calc, positioning}
\tikzstyle{input} = [draw, rectangle,rounded corners, text width=2.5cm, fill=green!20, node distance=0.5cm, minimum height=2em, text centered]
\tikzstyle{output} = [draw, ellipse,fill=red!20, node distance=0.5cm, minimum height=2em, text centered]
\tikzstyle{block} = [rectangle, draw, fill=blue!20, 
    text width=1.5cm, text centered, minimum height=2em, node distance = 0.5cm]
\tikzstyle{line} = [draw, -latex', shorten >=2pt, shorten <=2pt]
\tikzset{
  invisible/.style={opacity=0},
  visible on/.style={alt={#1{}{invisible}}},
  alt/.code args={<#1>#2#3}{%
    \alt<#1>{\pgfkeysalso{#2}}{\pgfkeysalso{#3}} % \pgfkeysalso doesn't change the path
  },
}

%\usepackage{pgfpages}
%\mode<handout>{
%	%\setbeamercolor{background canvas}{bg=black!20}
%	\pgfpagesuselayout{2 on 1}[border shrink=2mm]
%}

\title{Formation R Perfectionnement}
\date{}

\begin{document}
\frame{\titlepage}

\section{Travailler~efficacement~sur~des~données~avec~R}\label{travaillerefficacementsurdesdonneesavecr}

\subsection*{Travailler~efficacement~sur~des~données~avec~R}\label{travaillerefficacementsurdesdonneesavecr-1}
\addcontentsline{toc}{subsection}{Travailler~efficacement~sur~des~données~avec~R}

\begin{frame}{Qu'est-ce que travailler efficacement avec R ?}

Appliqué au travail sur des données, l'efficacité peut avoir au moins
deux significations distinctes :

\begin{itemize}
\tightlist
\item
  efficacité \textbf{algorithmique} : minimisation du temps passé par la
  machine pour réaliser une série d'opérations;
\item
  \textbf{productivité} du programmeur : minimisation du temps passé à
  coder une série d'opération.
\end{itemize}

\pause En règle générale, on peut avoir l'idée que plus on souhaite être
efficace algorithmiquement, plus la programmation risque d'être longue
et difficile.

\pause \textbf{Ce n'est pas toujours vrai} : on perd souvent beaucoup de
temps à (ré)inventer une méthode peu efficace quand une beaucoup plus
simple et rapide existe déjà.

\pause \small

\textbf{Référence} \textsc{Gillepsie C., Lovelace R.},
\textit{Efficient R programming} (disponible sur
\href{\%5D(https://bookdown.org/csgillepsie/efficientR)}{\underline{bookdown.org}})

\end{frame}

\begin{frame}[fragile]{Mesure l'efficacité algorithmique}

La fonction \texttt{system.time()} permet de mesurer la durée d'un
traitement.

\footnotesize

\begin{Shaded}
\begin{Highlighting}[]
\KeywordTok{system.time}\NormalTok{(}\KeywordTok{rnorm}\NormalTok{(}\FloatTok{1e6}\NormalTok{))}
\NormalTok{  ##    user  system elapsed }
\NormalTok{  ##   0.141   0.000   0.141}
\end{Highlighting}
\end{Shaded}

\pause \normalsize
Néanmoins, elle est inadaptée aux traitements de très courte durée. Dans
ces situations, privilégier la fonction \texttt{microbenchmark()} du
package \texttt{microbenchmark}.

\footnotesize

\begin{Shaded}
\begin{Highlighting}[]
\KeywordTok{library}\NormalTok{(microbenchmark)}
\KeywordTok{microbenchmark}\NormalTok{(}\DataTypeTok{times =} \DecValTok{10}\NormalTok{, }\KeywordTok{rnorm}\NormalTok{(}\FloatTok{1e6}\NormalTok{))}
\NormalTok{  ## Unit: milliseconds}
\NormalTok{  ##          expr      min       lq     mean   median}
\NormalTok{  ##  rnorm(1e+06) 69.56526 69.77588 70.48042 70.52055}
\NormalTok{  ##        uq      max neval}
\NormalTok{  ##  71.06151 71.49039    10}
\end{Highlighting}
\end{Shaded}

\end{frame}

\begin{frame}[fragile]{Mesurer la taille d'un objet en mémoire}

R stocke l'ensemble des fichiers sur lesquels il travaille dans la
mémoire vive.

Afin de loger les objets les plus gros mais aussi d'optimiser les
performances, il est souvent utile de \textbf{limiter la taille des
objets} sur lesquels portent les traitements.

\pause Pour mesurer la taille des objets, utiliser la fonction
\texttt{object\_size()} du \emph{package} \texttt{pryr}.

\begin{Shaded}
\begin{Highlighting}[]
\KeywordTok{library}\NormalTok{(pryr)}
\KeywordTok{object_size}\NormalTok{(}\KeywordTok{rnorm}\NormalTok{(}\FloatTok{1e6}\NormalTok{))}
\NormalTok{  ## 8 MB}
\end{Highlighting}
\end{Shaded}

\end{frame}

\begin{frame}[fragile]{Construire un exemple reproductible (MWE)}

Lorsque l'on cherche à améliorer les performances d'un programme, il est
important de pouvoir le tester sur des données \textbf{autonomes et
reproductibles}.

\pause Pour ce faire, les \textbf{fonctions de générations de nombres
aléatoires} de R sont particulièrement utiles.

\footnotesize

\begin{Shaded}
\begin{Highlighting}[]
\CommentTok{# Graine pour pouvoir reproduire l'aléa}
\KeywordTok{set.seed}\NormalTok{(}\DecValTok{2018}\NormalTok{)}

\CommentTok{# Vecteur de nombres de taille 1 000}
\NormalTok{a <-}\StringTok{ }\KeywordTok{rnorm}\NormalTok{(}\DecValTok{1000}\NormalTok{)}

\CommentTok{# Vecteur de lettres de taille 1 000}
\NormalTok{b <-}\StringTok{ }\NormalTok{letters[}\KeywordTok{sample}\NormalTok{(}\DecValTok{1}\OperatorTok{:}\DecValTok{26}\NormalTok{, }\DecValTok{1000}\NormalTok{, }\DataTypeTok{replace =} \OtherTok{TRUE}\NormalTok{)]}

\CommentTok{# Matrice logique 1 000 x 100 avec 1 % de TRUE}
\NormalTok{c <-}\StringTok{ }\KeywordTok{matrix}\NormalTok{(}\KeywordTok{runif}\NormalTok{(}\DecValTok{100000}\NormalTok{) }\OperatorTok{>}\StringTok{ }\FloatTok{0.99}\NormalTok{, }\DataTypeTok{ncol =} \DecValTok{100}\NormalTok{)}
\end{Highlighting}
\end{Shaded}

\end{frame}

\begin{frame}{Plan de la partie}

\large  \vfill
\tableofcontents[currentsection, sectionstyle = hide, subsectionstyle = show/show/hide]
\vfill

\end{frame}

\subsection{De l'importance des fonctions dans
R}\label{de-limportance-des-fonctions-dans-r}

\begin{frame}[fragile]{\og Tout ce qui agit est un appel de fonction
\fg{}}

\begin{quote}
\emph{To understand computations in R, two slogans are helpful:}

\begin{itemize}
\item
  \emph{Everything that exists is an object.}
\item
  \emph{Everything that happens is a function call.}
\end{itemize}

\emph{John Chambers}
\end{quote}

\pause \footnotesize

\begin{Shaded}
\begin{Highlighting}[]
\CommentTok{# ... même assigner une valeur}
\KeywordTok{is.function}\NormalTok{(}\StringTok{`}\DataTypeTok{<-}\StringTok{`}\NormalTok{)}
\NormalTok{  ## [1] TRUE}
\StringTok{`}\DataTypeTok{<-}\StringTok{`}\NormalTok{(a, }\DecValTok{10}\NormalTok{)}

\CommentTok{# ... même afficher la valeur d'un objet}
\NormalTok{a}
\NormalTok{  ## [1] 10}
\KeywordTok{print}\NormalTok{(a)}
\NormalTok{  ## [1] 10}
\end{Highlighting}
\end{Shaded}

\end{frame}

\begin{frame}[fragile]{Définir une fonction dans R}

Utilisé avec \texttt{\textless{}-}, \texttt{function()} définit une
nouvelle fonction :

\pause \footnotesize

\begin{Shaded}
\begin{Highlighting}[]
\CommentTok{# Définition de la fonction monCalcul()}
\NormalTok{monCalcul <-}\StringTok{ }\ControlFlowTok{function}\NormalTok{(a, b)\{}
\NormalTok{  resultat <-}\StringTok{ }\DecValTok{10} \OperatorTok{*}\StringTok{ }\NormalTok{a }\OperatorTok{+}\StringTok{ }\NormalTok{b}
  \KeywordTok{return}\NormalTok{(resultat)}
\NormalTok{\}}

\CommentTok{# Code de monCalcul()}
\NormalTok{monCalcul}
\NormalTok{  ## function(a, b)\{}
\NormalTok{  ##   resultat <- 10 * a + b}
\NormalTok{  ##   return(resultat)}
\NormalTok{  ## \}}

\CommentTok{# Appel de la fonction monCalcul()}
\KeywordTok{monCalcul}\NormalTok{(}\DecValTok{2}\NormalTok{, }\DecValTok{3}\NormalTok{)}
\NormalTok{  ## [1] 23}
\end{Highlighting}
\end{Shaded}

\end{frame}

\begin{frame}[fragile]{Valeurs par défaut des paramètres}

Des valeurs par défaut peuvent être renseignées pour les paramètres.

\begin{Shaded}
\begin{Highlighting}[]
\NormalTok{monCalcul <-}\StringTok{ }\ControlFlowTok{function}\NormalTok{(a, }\DataTypeTok{b =} \DecValTok{3}\NormalTok{) }\DecValTok{10} \OperatorTok{*}\StringTok{ }\NormalTok{a }\OperatorTok{+}\StringTok{ }\NormalTok{b}
\KeywordTok{monCalcul}\NormalTok{(}\DecValTok{8}\NormalTok{)}
\NormalTok{  ## [1] 83}
\end{Highlighting}
\end{Shaded}

\pause Les valeurs par défaut peuvent dépendre des autres paramètres.

\begin{Shaded}
\begin{Highlighting}[]
\NormalTok{monCalcul <-}\StringTok{ }\ControlFlowTok{function}\NormalTok{(a, }\DataTypeTok{b =}\NormalTok{ a }\OperatorTok{*}\StringTok{ }\DecValTok{2}\NormalTok{) }\DecValTok{10} \OperatorTok{*}\StringTok{ }\NormalTok{a }\OperatorTok{+}\StringTok{ }\NormalTok{b}
\KeywordTok{monCalcul}\NormalTok{(}\DecValTok{2}\NormalTok{)}
\NormalTok{  ## [1] 24}
\end{Highlighting}
\end{Shaded}

\pause \textbf{Remarque} Ceci est la conséquence de la \emph{lazy
evaluation} des arguments dans R (\emph{cf.}
\href{http://adv-r.had.co.nz/Functions.html\#function-arguments}{\underline{\textit{Advanced R}}}).

\end{frame}

\begin{frame}[fragile]{Contrôle de la valeur des paramètres}

Des structures conditionnelles \texttt{if()} permettent de contrôler la
valeur des arguments.

\pause \footnotesize

\begin{Shaded}
\begin{Highlighting}[]
\NormalTok{monCalcul <-}\StringTok{ }\ControlFlowTok{function}\NormalTok{(}\DataTypeTok{a =} \OtherTok{NULL}\NormalTok{, }\DataTypeTok{b =} \OtherTok{NULL}\NormalTok{)\{}
  \ControlFlowTok{if}\NormalTok{(}\KeywordTok{is.null}\NormalTok{(a)) }\KeywordTok{stop}\NormalTok{(}\StringTok{"a n'est pas renseigné."}\NormalTok{)}
  \ControlFlowTok{if}\NormalTok{(}\KeywordTok{is.null}\NormalTok{(b))\{}
\NormalTok{    b <-}\StringTok{ }\NormalTok{a }\OperatorTok{*}\StringTok{ }\DecValTok{2}
    \KeywordTok{warning}\NormalTok{(}\StringTok{"b n'est pas renseigné."}\NormalTok{)}
\NormalTok{  \}}
  \KeywordTok{return}\NormalTok{(}\DecValTok{10} \OperatorTok{*}\StringTok{ }\NormalTok{a }\OperatorTok{+}\StringTok{ }\NormalTok{b)}
\NormalTok{\}}

\KeywordTok{monCalcul}\NormalTok{(}\DataTypeTok{b =} \DecValTok{3}\NormalTok{)}
\NormalTok{  ## Error in monCalcul(b = 3): a n'est pas renseigné.}
\KeywordTok{monCalcul}\NormalTok{(}\DataTypeTok{a =} \DecValTok{1}\NormalTok{)}
\NormalTok{  ## Warning in monCalcul(a = 1): b n'est pas renseigné.}
\NormalTok{  ## [1] 12}
\end{Highlighting}
\end{Shaded}

\end{frame}

\begin{frame}[fragile]{\large Portée des variables et environnements
(1)}

Dans R \textbf{chaque objet est repéré par son nom et son environnement}
: cela permet d'éviter les conflits de noms.

\pause \footnotesize

\begin{Shaded}
\begin{Highlighting}[]
\CommentTok{# Création d'une fonction sum() un peu absurde}
\NormalTok{sum <-}\StringTok{ }\ControlFlowTok{function}\NormalTok{(...) }\StringTok{"Ma super somme !"}
\KeywordTok{sum}\NormalTok{(}\DecValTok{2}\NormalTok{, }\DecValTok{3}\NormalTok{)}
\NormalTok{  ## [1] "Ma super somme !"}

\CommentTok{# Cette fonction est rattachée à l'environnement global}
\KeywordTok{ls}\NormalTok{()}
\NormalTok{  ## [1] "a"         "b"         "c"         "monCalcul"}
\NormalTok{  ## [5] "sum"}

\CommentTok{# Mais on peut toujours accéder à la fonction }
\CommentTok{# de base en utilisant ::}
\NormalTok{base}\OperatorTok{::}\KeywordTok{sum}\NormalTok{(}\DecValTok{2}\NormalTok{, }\DecValTok{3}\NormalTok{)}
\NormalTok{  ## [1] 5}
\end{Highlighting}
\end{Shaded}

\end{frame}

\begin{frame}[fragile]{\large Portée des variables et environnements
(2)}

\small
À chaque appel d'une fonction, un \textbf{environnement d'exécution} est
créé.

\footnotesize

\begin{Shaded}
\begin{Highlighting}[]
\NormalTok{maFun <-}\StringTok{ }\ControlFlowTok{function}\NormalTok{() }\KeywordTok{environment}\NormalTok{()}
\KeywordTok{maFun}\NormalTok{()}
\NormalTok{  ## <environment: 0x562bb6437f70>}
\KeywordTok{maFun}\NormalTok{()}
\NormalTok{  ## <environment: 0x562bb658b100>}
\end{Highlighting}
\end{Shaded}

\pause \small
En conséquence, les instructions exécutées à l'intérieur d'une fonction
\textbf{ne modifient pas l'environnement global}.

\footnotesize

\begin{Shaded}
\begin{Highlighting}[]
\NormalTok{a <-}\StringTok{ }\DecValTok{10}
\NormalTok{maFonction3 <-}\StringTok{ }\ControlFlowTok{function}\NormalTok{()\{}
\NormalTok{  a <-}\StringTok{ }\DecValTok{5}
\NormalTok{\}}
\KeywordTok{maFonction3}\NormalTok{()}
\NormalTok{a}
\NormalTok{  ## [1] 10}
\end{Highlighting}
\end{Shaded}

\end{frame}

\begin{frame}[fragile]{\large Portée des variables et environnements
(3)}

En revanche, les objets définis dans l'environnement global sont
accessibles au sein d'une fonction.

\footnotesize

\begin{Shaded}
\begin{Highlighting}[]
\NormalTok{a <-}\StringTok{ }\DecValTok{10}
\NormalTok{maFonction4 <-}\StringTok{ }\ControlFlowTok{function}\NormalTok{()\{}
\NormalTok{  a }\OperatorTok{+}\StringTok{ }\DecValTok{5}
\NormalTok{\}}
\KeywordTok{maFonction4}\NormalTok{()}
\NormalTok{  ## [1] 15}
\end{Highlighting}
\end{Shaded}

\pause \normalsize
Ceci est dû au fait que les environnements dans lequel R recherche des
objets sont \textbf{emboîtés les uns dans les autres} (\textit{cf.} la
fonction \texttt{search()}).

\textbf{Pour en savoir plus}
\href{http://adv-r.had.co.nz/Environments.html\#function-envs}{\underline{\textit{Advanced R}}},
\href{http://blog.obeautifulcode.com/R/How-R-Searches-And-Finds-Stuff/}{\underline{obeautifulcode.com}}

\end{frame}

\begin{frame}[fragile]{Valeur de retour d'une fonction}

La fonction \texttt{return()} spécifie la valeur à renvoyer. Pour
renvoyer plusieurs valeurs, utiliser une liste.

\pause \footnotesize

\begin{Shaded}
\begin{Highlighting}[]
\NormalTok{maFonction1 <-}\StringTok{ }\ControlFlowTok{function}\NormalTok{()\{}
\NormalTok{  a <-}\StringTok{ }\DecValTok{1}\OperatorTok{:}\DecValTok{5}\NormalTok{; b <-}\StringTok{ }\DecValTok{6}\OperatorTok{:}\DecValTok{10}\NormalTok{; }\KeywordTok{return}\NormalTok{(a)}
\NormalTok{\}}
\KeywordTok{maFonction1}\NormalTok{()}
\NormalTok{  ## [1] 1 2 3 4 5}

\NormalTok{maFonction2 <-}\StringTok{ }\ControlFlowTok{function}\NormalTok{()\{}
\NormalTok{  a <-}\StringTok{ }\DecValTok{1}\OperatorTok{:}\DecValTok{5}\NormalTok{; b <-}\StringTok{ }\DecValTok{6}\OperatorTok{:}\DecValTok{10}\NormalTok{; }\KeywordTok{return}\NormalTok{(}\KeywordTok{list}\NormalTok{(}\DataTypeTok{a =}\NormalTok{ a, }\DataTypeTok{b =}\NormalTok{ b))}
\NormalTok{\}}
\KeywordTok{maFonction2}\NormalTok{()}
\NormalTok{  ## $a}
\NormalTok{  ## [1] 1 2 3 4 5}
\NormalTok{  ## }
\NormalTok{  ## $b}
\NormalTok{  ## [1]  6  7  8  9 10}
\end{Highlighting}
\end{Shaded}

\end{frame}

\begin{frame}[fragile]{\large Effets de bord et programmation
fonctionnelle}

Par défaut, les fonctions dans R :

\begin{itemize}
\tightlist
\item
  ne modifient pas l'environnement d'origine (il n'y a \textbf{pas
  d'effets de bord});
\item
  peuvent être utilisées en lieu et place des valeurs qu'elles
  retournent.
\end{itemize}

\begin{Shaded}
\begin{Highlighting}[]
\NormalTok{monCalcul <-}\StringTok{  }\ControlFlowTok{function}\NormalTok{(a, b) }\DecValTok{10} \OperatorTok{*}\StringTok{ }\NormalTok{a }\OperatorTok{+}\StringTok{ }\NormalTok{b}
\KeywordTok{monCalcul}\NormalTok{(}\DecValTok{2}\NormalTok{, }\DecValTok{3}\NormalTok{) }\OperatorTok{+}\StringTok{ }\DecValTok{5}
\NormalTok{  ## [1] 28}
\end{Highlighting}
\end{Shaded}

\pause Ces éléments font de R un \textbf{langage particulièrement adapté
à la programmation fonctionnelle}.

\end{frame}

\begin{frame}[fragile]{Quelques principes de la programmation
fonctionnelle}

\vfill

\begin{enumerate}
\def\labelenumi{\arabic{enumi}.}
\tightlist
\item
  \textbf{Ne jamais créer d'effets de bord} Toute modification apportée
  à l'environnement par une fonction passe par sa valeur de sortie.
\end{enumerate}

\vfill

\begin{enumerate}
\def\labelenumi{\arabic{enumi}.}
\setcounter{enumi}{1}
\tightlist
\item
  \pause \textbf{Vectoriser \textit{i.e.} appliquer des fonctions systématiquement à un ensemble d'éléments}
  Fonctions \texttt{*apply()}, \texttt{Reduce()}, \texttt{do.call()}.
\end{enumerate}

\vfill

\begin{enumerate}
\def\labelenumi{\arabic{enumi}.}
\setcounter{enumi}{2}
\tightlist
\item
  \pause \textbf{Structurer les traitements à l'aide de fonctions courtes et explicites}
  Faciliter la relecture, la maintenance et la modularisation.
\end{enumerate}

\vfill

\pause \textbf{Pour en savoir plus}
\href{https://en.wikipedia.org/wiki/FP_(programming_language)}{\underline{Wikipedia}},
\href{https://maryrosecook.com/blog/post/a-practical-introduction-to-functional-programming}{\underline{maryrosecook.com}}.

\vfill

\end{frame}

\subsection{\texorpdfstring{Vectoriser : \texttt{*apply()},
\texttt{Reduce()} et
\texttt{do.call()}}{Vectoriser : *apply(), Reduce() et do.call()}}\label{vectoriser-apply-reduce-et-do.call}

\begin{frame}[fragile]{\large Appliquer sur chaque indépendamment :
\texttt{apply()}}

La fonction \texttt{apply(X,\ MARGIN,\ FUN)} applique la fonction
\texttt{FUN} à la \textbf{matrice} \texttt{X} selon la dimension
\texttt{MARGIN}.

\pause \footnotesize

\begin{Shaded}
\begin{Highlighting}[]
\CommentTok{# Définition et affichage de la matrice m}
\NormalTok{m <-}\StringTok{ }\KeywordTok{matrix}\NormalTok{(}\DecValTok{1}\OperatorTok{:}\DecValTok{6}\NormalTok{, }\DataTypeTok{ncol =} \DecValTok{3}\NormalTok{)}
\NormalTok{m}
\NormalTok{  ##      [,1] [,2] [,3]}
\NormalTok{  ## [1,]    1    3    5}
\NormalTok{  ## [2,]    2    4    6}

\CommentTok{# Application de la fonction sum() selon les lignes}
\KeywordTok{apply}\NormalTok{(m, }\DecValTok{1}\NormalTok{, sum)}
\NormalTok{  ## [1]  9 12}

\CommentTok{# Application de la fonction sum() selon les colonnes}
\KeywordTok{apply}\NormalTok{(m, }\DecValTok{2}\NormalTok{, sum)}
\NormalTok{  ## [1]  3  7 11}
\end{Highlighting}
\end{Shaded}

\end{frame}

\begin{frame}[fragile]{\large Appliquer sur chaque indépendamment :
\texttt{lapply()}}

\small
La fonction \texttt{lapply(X,\ FUN)} applique la fonction \texttt{FUN}
au \textbf{vecteur} ou à la \textbf{liste} \texttt{X}.

\pause \footnotesize

\begin{Shaded}
\begin{Highlighting}[]
\NormalTok{l <-}\StringTok{ }\KeywordTok{list}\NormalTok{(}\DecValTok{1}\OperatorTok{:}\DecValTok{5}\NormalTok{, }\KeywordTok{c}\NormalTok{(}\DecValTok{6}\OperatorTok{:}\DecValTok{9}\NormalTok{, }\OtherTok{NA}\NormalTok{))}
\NormalTok{l}
\NormalTok{  ## [[1]]}
\NormalTok{  ## [1] 1 2 3 4 5}
\NormalTok{  ## }
\NormalTok{  ## [[2]]}
\NormalTok{  ## [1]  6  7  8  9 NA}
\KeywordTok{lapply}\NormalTok{(l, sum)}
\NormalTok{  ## [[1]]}
\NormalTok{  ## [1] 15}
\NormalTok{  ## }
\NormalTok{  ## [[2]]}
\NormalTok{  ## [1] NA}
\end{Highlighting}
\end{Shaded}

\pause \small \vspace{-0.2cm}

\textbf{Exemple d'utilisation} Appliquer une fonction à toutes les
variables d'une table.

\end{frame}

\begin{frame}[fragile]{\large Appliquer sur chaque indépendamment :
\texttt{sapply()}}

La fonction \texttt{sapply()} est analogue à la fonction
\texttt{lapply()}, mais simplifie le résultat produit quand c'est
possible.

\begin{Shaded}
\begin{Highlighting}[]
\KeywordTok{sapply}\NormalTok{(l, sum)}
\NormalTok{  ## [1] 15 NA}
\end{Highlighting}
\end{Shaded}

\pause Les arguments optionnels de la fonction utilisée peuvent être
ajoutés à la suite dans toutes les fonctions \texttt{*apply()}.

\begin{Shaded}
\begin{Highlighting}[]
\KeywordTok{sapply}\NormalTok{(l, sum, }\DataTypeTok{na.rm =} \OtherTok{TRUE}\NormalTok{)}
\NormalTok{  ## [1] 15 30}
\end{Highlighting}
\end{Shaded}

\pause \textbf{Exemple d'utilisation} Calcul de statistiques sur toutes
les variables d'une table.

\end{frame}

\begin{frame}[fragile]{\large Définir une fonction à la volée dans
\texttt{*apply()}}

Il est fréquent que l'opération que l'on souhaite appliquer ne
corresponde pas exactement à une fonction pré-existante.

\pause Dans ce cas, on peut définir une \textbf{fonction à la volée}
dans la fonction \texttt{*apply()}.

\footnotesize

\begin{Shaded}
\begin{Highlighting}[]
\CommentTok{# On souhaite sélectionner le second élément de }
\CommentTok{# de chaque vecteur de la liste l}
\NormalTok{l}
\NormalTok{  ## [[1]]}
\NormalTok{  ## [1] 1 2 3 4 5}
\NormalTok{  ## }
\NormalTok{  ## [[2]]}
\NormalTok{  ## [1]  6  7  8  9 NA}

\CommentTok{# On définit une fonction dans sapply()}
\KeywordTok{sapply}\NormalTok{(l, }\ControlFlowTok{function}\NormalTok{(x) x[}\DecValTok{2}\NormalTok{])}
\NormalTok{  ## [1] 2 7}
\end{Highlighting}
\end{Shaded}

\end{frame}

\begin{frame}[fragile]{\large Appliquer sur chaque par groupe :
\texttt{tapply()}}

La fonction \texttt{tapply(X,\ INDEX,\ FUN)} applique la fonction
\texttt{FUN}, à l'objet \texttt{X} ventilé selon les modalités de
\texttt{INDEX}.

\footnotesize \pause

\begin{Shaded}
\begin{Highlighting}[]
\CommentTok{# Variables d'âge et de sexe}
\NormalTok{age <-}\StringTok{ }\KeywordTok{c}\NormalTok{(}\DecValTok{45}\NormalTok{, }\DecValTok{50}\NormalTok{, }\DecValTok{35}\NormalTok{, }\DecValTok{20}\NormalTok{)}
\NormalTok{sexe <-}\StringTok{ }\KeywordTok{c}\NormalTok{(}\StringTok{"H"}\NormalTok{, }\StringTok{"F"}\NormalTok{, }\StringTok{"F"}\NormalTok{, }\StringTok{"H"}\NormalTok{)}

\CommentTok{# Âge moyen par sexe}
\KeywordTok{tapply}\NormalTok{(age, sexe, mean)}
\NormalTok{  ##    F    H }
\NormalTok{  ## 42.5 32.5}

\CommentTok{# Même résultat avec une combinaison de sapply() et de split()}
\KeywordTok{sapply}\NormalTok{(}\KeywordTok{split}\NormalTok{(age, sexe), mean)}
\NormalTok{  ##    F    H }
\NormalTok{  ## 42.5 32.5}
\end{Highlighting}
\end{Shaded}

\normalsize \pause \vspace{-0.2cm}

\textbf{Exemple d'utilisation} Calcul de statistiques agrégées par
catégories.

\end{frame}

\begin{frame}[fragile]{\large Appliquer sur tous : \texttt{do.call()}}

La fonction \texttt{do.call(what,\ args)} permet d'appliquer la fonction
\texttt{what()} à un \textbf{ensemble} d'arguments \texttt{args}
spécifié comme une liste (alors que les fonctions \texttt{*apply()}
appliqueraient \texttt{what()} à \textbf{chaque} élément de
\texttt{args}).

\pause \small

\begin{Shaded}
\begin{Highlighting}[]
\CommentTok{# Concaténation des vecteurs de l}
\KeywordTok{do.call}\NormalTok{(base}\OperatorTok{::}\NormalTok{c, l)}
\NormalTok{  ##  [1]  1  2  3  4  5  6  7  8  9 NA}

\CommentTok{# Equivalent à }
\NormalTok{base}\OperatorTok{::}\KeywordTok{c}\NormalTok{(l[[}\DecValTok{1}\NormalTok{]], l[[}\DecValTok{2}\NormalTok{]])}
\NormalTok{  ##  [1]  1  2  3  4  5  6  7  8  9 NA}
\end{Highlighting}
\end{Shaded}

\pause \normalsize

\textbf{Exemple d'utilisation} Concaténer de nombreuses tables avec
\texttt{rbind()} ou \texttt{cbind()}.

\end{frame}

\begin{frame}[fragile]{\large Appliquer sur tous successivement :
\texttt{Reduce()}}

La fonction \texttt{Reduce(f,\ x)} permet d'appliquer la fonction
\texttt{f()} \textbf{successivement} à l'ensemble des éléments de
\texttt{x} (alors que \texttt{do.call()} applique \texttt{f}
\textbf{simultanément}).

\pause \small

\begin{Shaded}
\begin{Highlighting}[]
\CommentTok{# Application successive de la division au vecteur 1:4}
\KeywordTok{Reduce}\NormalTok{(}\StringTok{`}\DataTypeTok{/}\StringTok{`}\NormalTok{, }\DecValTok{1}\OperatorTok{:}\DecValTok{4}\NormalTok{)}
\NormalTok{  ## [1] 0.04166667}

\CommentTok{# Equivalent à }
\NormalTok{((}\DecValTok{1}\OperatorTok{/}\DecValTok{2}\NormalTok{)}\OperatorTok{/}\DecValTok{3}\NormalTok{)}\OperatorTok{/}\DecValTok{4}
\NormalTok{  ## [1] 0.04166667}
\end{Highlighting}
\end{Shaded}

\pause \normalsize

\textbf{Exemple d'utilisation} Fusionner de nombreuses tables avec
\texttt{merge()} (sur les mêmes identifiants).

\end{frame}

\subsection{Coder efficacement en base
R}\label{coder-efficacement-en-base-r}

\begin{frame}[fragile]{L'idée : En faire faire le moins possible à R}

R est un langage dit \og de haut niveau \fg{} : les objets qui le
composent sont relativement faciles d'utilisation, au prix de
performances limitées.

\vfill

À l'inverse, des langages dits de \og bas niveau \fg{} (par exemple C++)
sont plus difficiles à utiliser mais aussi plus efficaces.

\vfill

\pause La plupart des fonctions fondamentales de R font appel à des
fonctions compilées à partir d'un langage de plus bas niveau.

\vfill

D'où le principe : \textbf{limiter au maximum la surcharge liée à R}
pour retomber au plus vite sur des fonctions pré-compilées.

\pause \vfill

\textbf{Remarque} Il est très facile en pratique d'utiliser R comme une
interface vers des langages de plus bas niveau, \emph{cf.~infra} à
propos de \texttt{Rcpp}.

\end{frame}

\begin{frame}[fragile]{Utiliser les boucles avec parcimonie (1)}

Comme la plupart des langages de programmation, R dispose de
\textbf{structures de contrôles} permettant de réaliser des boucles.

\footnotesize

\begin{Shaded}
\begin{Highlighting}[]
\NormalTok{boucle <-}\StringTok{ }\ControlFlowTok{function}\NormalTok{(x)\{}
\NormalTok{  cumul <-}\StringTok{ }\KeywordTok{rep}\NormalTok{(}\OtherTok{NA}\NormalTok{, }\KeywordTok{length}\NormalTok{(x))}
  \ControlFlowTok{for}\NormalTok{(i }\ControlFlowTok{in} \KeywordTok{seq_along}\NormalTok{(x)) }
\NormalTok{    cumul[i] <-}\StringTok{ }\ControlFlowTok{if}\NormalTok{(i }\OperatorTok{==}\StringTok{ }\DecValTok{1}\NormalTok{) x[i] }\ControlFlowTok{else}\NormalTok{ cumul[i }\OperatorTok{-}\StringTok{ }\DecValTok{1}\NormalTok{] }\OperatorTok{+}\StringTok{ }\NormalTok{x[i]}
  \KeywordTok{return}\NormalTok{(cumul)}
\NormalTok{\}}
\KeywordTok{boucle}\NormalTok{(}\DecValTok{1}\OperatorTok{:}\DecValTok{5}\NormalTok{)}
\NormalTok{  ## [1]  1  3  6 10 15}
\end{Highlighting}
\end{Shaded}

\pause \normalsize
Ces opérations présentent plusieurs inconvénients :

\begin{enumerate}
\def\labelenumi{\arabic{enumi}.}
\tightlist
\item
  Elles sont longues à écrire et assez peu claires;
\item
  Elles reposent sur des effets de bord;
\item
  Elles sont en général très peu \textbf{efficaces algorithmiquement}.
\end{enumerate}

\end{frame}

\begin{frame}[fragile]{Utiliser les boucles avec parcimonie (2)}

\small
Les méthodes de vectorisation sont en général beaucoup plus efficaces
que les boucles en R :

\begin{itemize}
\tightlist
\item
  vectorisation de haut niveau (\emph{cf.} \emph{supra});
\item
  vectorisation de bas niveau : la vectorisation est opérée par le
  langage de bas niveau auquel fait appel R.
\end{itemize}

\pause \footnotesize

\begin{Shaded}
\begin{Highlighting}[]
\KeywordTok{summary}\NormalTok{(}\KeywordTok{microbenchmark}\NormalTok{(}\DataTypeTok{times =}\NormalTok{ 10L}
\NormalTok{  , }\DataTypeTok{boucle =} \KeywordTok{boucle}\NormalTok{(}\DecValTok{1}\OperatorTok{:}\FloatTok{1e4}\NormalTok{)}
\NormalTok{  , }\DataTypeTok{Reduce =} \KeywordTok{Reduce}\NormalTok{(}\StringTok{`}\DataTypeTok{+}\StringTok{`}\NormalTok{, }\DecValTok{1}\OperatorTok{:}\FloatTok{1e4}\NormalTok{, }\DataTypeTok{accumulate =} \OtherTok{TRUE}\NormalTok{)}
\NormalTok{  , }\DataTypeTok{cumsum =} \KeywordTok{cumsum}\NormalTok{(}\DecValTok{1}\OperatorTok{:}\FloatTok{1e4}\NormalTok{)}
\NormalTok{))[, }\DecValTok{1}\OperatorTok{:}\DecValTok{4}\NormalTok{]}
\NormalTok{  ##     expr       min        lq       mean}
\NormalTok{  ## 1 boucle 16705.371 22047.502 23381.2887}
\NormalTok{  ## 2 Reduce  6106.949  6723.186  9166.0918}
\NormalTok{  ## 3 cumsum    36.226    38.036    55.9669}
\end{Highlighting}
\end{Shaded}

\end{frame}

\begin{frame}[fragile]{Tirer le meilleur parti de la compilation (1)}

On distingue souvent deux familles de langages informatiques :

\begin{itemize}
\tightlist
\item
  \pause \vspace{-0.2cm} les langages \textbf{compilés} (C, C++) :
  l'ensemble du code est transformé en langage machine par un
  \emph{compilateur} puis soumis par le système d'exploitation;
\item
  \pause les langages \textbf{interprétés} (R, Python) : les
  instructions du code sont soumises les unes après les autres par un
  \emph{interpréteur}, ce qui est moins efficace (\emph{cf.} boucles en
  R).
\end{itemize}

\vspace{0.1cm} \pause La fonction \texttt{compiler::cmpfun()} permet
néanmoins de \textbf{compiler} des fonctions R avant utilisation.

\pause \footnotesize \vspace{-0.2cm}

\begin{Shaded}
\begin{Highlighting}[]
\CommentTok{# Compilation de la fonction boucle()}
\NormalTok{boucle_compil <-}\StringTok{ }\NormalTok{compiler}\OperatorTok{::}\KeywordTok{cmpfun}\NormalTok{(boucle)}
\KeywordTok{microbenchmark}\NormalTok{(}\KeywordTok{boucle}\NormalTok{(}\DecValTok{1}\OperatorTok{:}\FloatTok{1e4}\NormalTok{), }\KeywordTok{boucle_compil}\NormalTok{(}\DecValTok{1}\OperatorTok{:}\FloatTok{1e4}\NormalTok{))}
\NormalTok{  ## Unit: milliseconds}
\NormalTok{  ##                    expr       min        lq      mean}
\NormalTok{  ##         boucle(1:10000) 15.082436 18.375149 21.189216}
\NormalTok{  ##  boucle_compil(1:10000)  1.525206  1.652735  1.939735}
\NormalTok{  ##     median        uq       max neval}
\NormalTok{  ##  20.956547 24.001520 27.866753   100}
\NormalTok{  ##   1.707153  2.126435  3.806843   100}
\end{Highlighting}
\end{Shaded}

\end{frame}

\begin{frame}[fragile]{Tirer le meilleur parti de la compilation (2)}

Une autre fonctionnalité du \emph{package} \texttt{compiler} est la
compilation \og juste-à-temps \fg{} (ou \textit{just-in-time}, JIT): le
code n'est plus interprété mais \textbf{compilé au fur et à mesure}.

\bigskip \pause Dans R, on active le mode JIT pour une session grâce à
la fonction \texttt{compiler::enableJIT()} en spécifiant le
\href{https://stat.ethz.ch/R-manual/R-devel/library/compiler/html/compile.html}{\underline{niveau de compilation JIT}}
(de 0 à 3).

\pause \footnotesize

\begin{Shaded}
\begin{Highlighting}[]
\CommentTok{# Passage au niveau maximal de compilation JIT}
\NormalTok{compiler}\OperatorTok{::}\KeywordTok{enableJIT}\NormalTok{(}\DecValTok{3}\NormalTok{)}
\NormalTok{  ## [1] 0}
\KeywordTok{summary}\NormalTok{(}\KeywordTok{microbenchmark}\NormalTok{(}\KeywordTok{boucle}\NormalTok{(}\DecValTok{1}\OperatorTok{:}\FloatTok{1e4}\NormalTok{), }\KeywordTok{boucle_compil}\NormalTok{(}\DecValTok{1}\OperatorTok{:}\FloatTok{1e4}\NormalTok{)))[, }\DecValTok{1}\OperatorTok{:}\DecValTok{4}\NormalTok{]}
\NormalTok{  ##                     expr      min       lq     mean}
\NormalTok{  ## 1        boucle(1:10000) 1.494327 1.574196 1.928021}
\NormalTok{  ## 2 boucle_compil(1:10000) 1.512143 1.564685 1.713391}
\end{Highlighting}
\end{Shaded}

\pause \small \textbf{Remarque} Depuis R 3.4.0, \texttt{enableJIT()}
vaut 3 par défaut.

\end{frame}

\begin{frame}[fragile]{Utiliser l'opérateur \texttt{{[}} au lieu de
\texttt{ifelse()}}

\small
Lorsqu'on crée une variable en faisant intervenir une condition, il est
fréquent d'utiliser la fonction \texttt{ifelse()} :

\footnotesize

\begin{Shaded}
\begin{Highlighting}[]
\NormalTok{notes <-}\StringTok{ }\KeywordTok{runif}\NormalTok{(}\DataTypeTok{n =} \DecValTok{100000}\NormalTok{, }\DataTypeTok{min =} \DecValTok{0}\NormalTok{, }\DataTypeTok{max =} \DecValTok{20}\NormalTok{)}
\NormalTok{mavar <-}\StringTok{ }\KeywordTok{ifelse}\NormalTok{(notes }\OperatorTok{>=}\StringTok{ }\DecValTok{10}\NormalTok{, }\StringTok{"Reçu"}\NormalTok{, }\StringTok{"Recalé"}\NormalTok{)}
\end{Highlighting}
\end{Shaded}

\pause \small
Il est néanmoins beaucoup plus efficace d'utiliser l'opérateur
\texttt{{[}}.

\footnotesize

\begin{Shaded}
\begin{Highlighting}[]
\KeywordTok{microbenchmark}\NormalTok{(}\DataTypeTok{times =}\NormalTok{ 10L}
\NormalTok{  , }\DataTypeTok{ifelse =} \KeywordTok{ifelse}\NormalTok{(notes }\OperatorTok{>=}\StringTok{ }\DecValTok{10}\NormalTok{, }\StringTok{"Reçu"}\NormalTok{, }\StringTok{"Recalé"}\NormalTok{)}
\NormalTok{  , }\StringTok{"["}\NormalTok{ =}\StringTok{ }\NormalTok{\{}
\NormalTok{    mavar <-}\StringTok{ }\KeywordTok{rep}\NormalTok{(}\StringTok{"Recalé"}\NormalTok{, }\KeywordTok{length}\NormalTok{(notes))}
\NormalTok{    mavar[notes }\OperatorTok{>=}\StringTok{ }\DecValTok{10}\NormalTok{] <-}\StringTok{ "Reçu"}
\NormalTok{  \} }
\NormalTok{)}
\NormalTok{  ## Unit: milliseconds}
\NormalTok{  ##    expr       min        lq      mean   median}
\NormalTok{  ##  ifelse 27.815496 29.111395 42.367241 31.15054}
\NormalTok{  ##       [  1.376459  1.407397  1.875707  1.50845}
\NormalTok{  ##         uq        max neval}
\NormalTok{  ##  33.501269 147.813919    10}
\NormalTok{  ##   1.711601   3.406202    10}
\end{Highlighting}
\end{Shaded}

\end{frame}

\begin{frame}[fragile]{Simplifier les données : le type \texttt{factor}}

On utilise souvent des chaînes de caractère pour coder une variable de
nature catégorielle.

Le type \texttt{factor} permet de remplacer chaque valeur distincte par
un entier en sauvegardant la table de correspondance. Il est
\textbf{beaucoup plus léger}.

\pause \footnotesize

\begin{Shaded}
\begin{Highlighting}[]
\CommentTok{# Variable à deux modalités codées en caractères}
\NormalTok{sexe <-}\StringTok{ }\KeywordTok{sample}\NormalTok{(}\KeywordTok{c}\NormalTok{(}\StringTok{"H"}\NormalTok{, }\StringTok{"F"}\NormalTok{), }\DecValTok{120000}\NormalTok{, }\DataTypeTok{replace =} \OtherTok{TRUE}\NormalTok{)}
\KeywordTok{object_size}\NormalTok{(sexe)}
\NormalTok{  ## 960 kB}

\CommentTok{# Conversion en facteur}
\NormalTok{f.sexe <-}\StringTok{ }\KeywordTok{factor}\NormalTok{(sexe)}
\KeywordTok{str}\NormalTok{(f.sexe)}
\NormalTok{  ##  Factor w/ 2 levels "F","H": 1 2 1 2 1 2 2 2 1 1 ...}
\KeywordTok{object_size}\NormalTok{(f.sexe)}
\NormalTok{  ## 481 kB}
\end{Highlighting}
\end{Shaded}

\end{frame}

\begin{frame}[fragile]{Utiliser les noms à bon escient (1)}

La plupart des objets manipulés couramment dans R peuvent être
\textbf{nommés} : vecteurs, matrices, listes, \texttt{data.frame}.

Utiliser des noms est une méthode souvent \textbf{très rapide} pour
\textbf{accéder aux éléments} qui composent ces objets.

~

\pause 

\textbf{Exemple} On cherche à extraire les observations d'une table
\emph{via} leur identifiant \texttt{id}. On compare l'utilisation des
noms à une fusion réalisée avec \texttt{merge()}.

\small 

\begin{Shaded}
\begin{Highlighting}[]
\CommentTok{# Création de la table df}
\NormalTok{id <-}\StringTok{ }\KeywordTok{as.character}\NormalTok{(}\KeywordTok{sample}\NormalTok{(}\FloatTok{1e5}\NormalTok{))}
\NormalTok{sexe <-}\StringTok{ }\KeywordTok{sample}\NormalTok{(}\DecValTok{1}\OperatorTok{:}\DecValTok{2}\NormalTok{, }\FloatTok{1e5}\NormalTok{, }\DataTypeTok{replace =} \OtherTok{TRUE}\NormalTok{)}
\NormalTok{df <-}\StringTok{ }\KeywordTok{data.frame}\NormalTok{(id, sexe)}
\end{Highlighting}
\end{Shaded}

\end{frame}

\begin{frame}[fragile]{Utiliser les noms à bon escient (2)}

\footnotesize

\begin{Shaded}
\begin{Highlighting}[]
\CommentTok{# Affectation de noms à df }
\KeywordTok{row.names}\NormalTok{(df) <-}\StringTok{ }\NormalTok{id}

\CommentTok{# Liste des identifiants à extraire}
\NormalTok{extract <-}\StringTok{ }\KeywordTok{c}\NormalTok{(}\StringTok{"234"}\NormalTok{, }\StringTok{"12"}\NormalTok{, }\StringTok{"7890"}\NormalTok{)}

\CommentTok{# Comparaison}
\KeywordTok{microbenchmark}\NormalTok{(}\DataTypeTok{times =}\NormalTok{ 10L}
\NormalTok{  , }\DataTypeTok{merge =} \KeywordTok{merge}\NormalTok{(}\KeywordTok{data.frame}\NormalTok{(}\DataTypeTok{id =}\NormalTok{ extract), df, }\DataTypeTok{sort =} \OtherTok{FALSE}\NormalTok{)}
\NormalTok{  , }\DataTypeTok{names =}\NormalTok{ df[extract, ]}
\NormalTok{)}
\NormalTok{  ## Unit: milliseconds}
\NormalTok{  ##   expr       min        lq      mean    median}
\NormalTok{  ##  merge 15.657587 17.971193 22.305240 22.837128}
\NormalTok{  ##  names  2.725046  2.746538  3.782196  3.233351}
\NormalTok{  ##         uq       max neval}
\NormalTok{  ##  27.666208 30.020111    10}
\NormalTok{  ##   5.349018  5.757353    10}
\end{Highlighting}
\end{Shaded}

\end{frame}

\begin{frame}[fragile]{À propos des matrices (1)}

Quand c'est possible, \textbf{travailler sur des matrices} (plutôt que
des \texttt{data.frame}) est souvent source d'efficacité :

\vfill

\begin{itemize}
\tightlist
\item
  \pause de nombreuses opérations sont \textbf{vectorisées} pour les
  matrices : sommes en lignes et en colonnes (\texttt{rowSums()} et
  \texttt{colSums()}), etc. ;
\end{itemize}

\vfill 

\begin{itemize}
\tightlist
\item
  \pause l'\textbf{algèbre matricielle} (le produit matriciel notamment)
  est très bien optimisée ;
\end{itemize}

\vfill

\begin{itemize}
\tightlist
\item
  \pause selon la nature du problème, l'utilisation de \textbf{matrices
  lacunaires} (\emph{sparse}) peut faire gagner et en empreinte mémoire
  et en temps de calcul (\emph{cf.} le \emph{package} \texttt{Matrix}).
\end{itemize}

\end{frame}

\begin{frame}[fragile]{À propos des matrices (2)}

\footnotesize 

\begin{Shaded}
\begin{Highlighting}[]
\CommentTok{# Création d'une matrice m avec 99 % de 0}
\NormalTok{v <-}\StringTok{ }\KeywordTok{rep}\NormalTok{(}\DecValTok{0}\NormalTok{, }\FloatTok{1e6}\NormalTok{); v[}\KeywordTok{sample}\NormalTok{(}\FloatTok{1e6}\NormalTok{, }\FloatTok{1e4}\NormalTok{)] <-}\StringTok{ }\KeywordTok{rnorm}\NormalTok{(}\FloatTok{1e4}\NormalTok{)}
\NormalTok{m <-}\StringTok{ }\KeywordTok{matrix}\NormalTok{(v, }\DataTypeTok{ncol =} \DecValTok{100}\NormalTok{)}

\CommentTok{# Transformation en matrice lacunaire}
\KeywordTok{library}\NormalTok{(Matrix)}
\NormalTok{M <-}\StringTok{ }\KeywordTok{Matrix}\NormalTok{(m)}

\CommentTok{# Gain en espace (en ko)}
\KeywordTok{c}\NormalTok{(}\KeywordTok{object_size}\NormalTok{(m), }\KeywordTok{object_size}\NormalTok{(M))}
\NormalTok{  ## [1] 8000200  121824}

\CommentTok{# Gain de performances pour la fonction colSums()}
\KeywordTok{microbenchmark}\NormalTok{(}\DataTypeTok{dense =} \KeywordTok{colSums}\NormalTok{(m), }\DataTypeTok{sparse =} \KeywordTok{colSums}\NormalTok{(M))}
\NormalTok{  ## Unit: microseconds}
\NormalTok{  ##    expr      min       lq       mean   median       uq}
\NormalTok{  ##   dense 1271.903 1290.240 1465.75316 1344.764 1408.385}
\NormalTok{  ##  sparse   54.282   66.571   92.75276   87.245   93.173}
\NormalTok{  ##       max neval}
\NormalTok{  ##  4048.731   100}
\NormalTok{  ##   595.279   100}
\end{Highlighting}
\end{Shaded}

\end{frame}

\subsection{dplyr : une grammaire du traitement des
données}\label{dplyr-une-grammaire-du-traitement-des-donnees}

\begin{frame}[fragile]{Philosophie de \texttt{dplyr}}

\texttt{dplyr} est un \emph{package} développé par RStudio et en
particulier par Hadley Wickham. Il constitue un véritable
\textbf{écosystème} visant à faciliter le travail sur des tables
statistiques :

\begin{itemize}
\tightlist
\item
  \pause il fournit un ensemble de \textbf{fonctions élémentaires} (les
  \og verbes \fg{}) pour effectuer les manipulations de données;
\item
  \pause plusieurs verbes peuvent facilement être \textbf{combinés en
  utilisant l'opérateur \texttt{\%\textgreater{}\%}} (\emph{pipe});
\item
  \pause toutes les opérations sont optimisées par du \textbf{code de
  bas niveau}.
\end{itemize}

\begin{Shaded}
\begin{Highlighting}[]
\KeywordTok{library}\NormalTok{(dplyr)}
\end{Highlighting}
\end{Shaded}

\pause

\textbf{Pour en savoir plus} De nombreuses vignettes très pédagogiques
sont disponibles sur la
\href{https://cran.r-project.org/package=dplyr}{\underline{page du \textit{package}}}.
Un
\href{https://www.rstudio.com/wp-content/uploads/2016/01/data-wrangling-french.pdf}{\underline{aide-mémoire}}
est également disponible sur le site de RStudio.

\end{frame}

\begin{frame}[fragile]{\large Données d'exemple : table \texttt{flights}
de \texttt{nycflights13}}

Les exemples relatifs aux \emph{packages} \texttt{dplyr} et
\texttt{data.table} s'appuient sur les données du \emph{package}
\texttt{nycflights13}.

\begin{Shaded}
\begin{Highlighting}[]
\KeywordTok{library}\NormalTok{(nycflights13)}
\end{Highlighting}
\end{Shaded}

\pause Ce \emph{package} contient des données sur tous les vols au
départ de la ville de New-York en 2013.

\small

\begin{Shaded}
\begin{Highlighting}[]
\KeywordTok{data}\NormalTok{(}\DataTypeTok{package =} \StringTok{"nycflights13"}\NormalTok{)}
\KeywordTok{dim}\NormalTok{(flights)}
\NormalTok{  ## [1] 336776     19}
\KeywordTok{names}\NormalTok{(flights)[}\DecValTok{1}\OperatorTok{:}\DecValTok{9}\NormalTok{]}
\NormalTok{  ## [1] "year"           "month"          "day"           }
\NormalTok{  ## [4] "dep_time"       "sched_dep_time" "dep_delay"     }
\NormalTok{  ## [7] "arr_time"       "sched_arr_time" "arr_delay"}
\end{Highlighting}
\end{Shaded}

\end{frame}

\begin{frame}[fragile]{Simplifier des opérations de base R}

\texttt{dplyr} propose plusieurs verbes pour simplifier certaines
opérations parfois fastidieuses en base R :

\pause \vspace{-1mm} - \texttt{filter()} sélectionne des observations
selon une ou plusieurs conditions;

\vspace{-3mm}

\begin{Shaded}
\begin{Highlighting}[]
\KeywordTok{filter}\NormalTok{(flights, month }\OperatorTok{==}\StringTok{ }\DecValTok{7}\NormalTok{, day }\OperatorTok{==}\StringTok{ }\DecValTok{4}\NormalTok{)}
\end{Highlighting}
\end{Shaded}

\pause \vspace{-3mm} - \texttt{arrange()} trie le fichier selon une ou
plusieurs variables;

\vspace{-3mm}

\begin{Shaded}
\begin{Highlighting}[]
\KeywordTok{arrange}\NormalTok{(flights, month, }\KeywordTok{desc}\NormalTok{(distance))}
\end{Highlighting}
\end{Shaded}

\pause \vspace{-3mm} - \texttt{select()} sélectionne des variables par
leur noms;

\vspace{-3mm}

\begin{Shaded}
\begin{Highlighting}[]
\KeywordTok{select}\NormalTok{(flights, year}\OperatorTok{:}\NormalTok{arr_delay)}
\end{Highlighting}
\end{Shaded}

\pause \vspace{-3mm} - \texttt{rename()} renomme des variables.

\vspace{-3mm}

\begin{Shaded}
\begin{Highlighting}[]
\KeywordTok{rename}\NormalTok{(flights, }\DataTypeTok{annee =}\NormalTok{ year)}
\end{Highlighting}
\end{Shaded}

\end{frame}

\begin{frame}[fragile]{Calculer des statistiques avec
\texttt{summarise()}}

La fonction \texttt{summarise()} permet de facilement calculer des
statistiques sur des données.

\pause 

\begin{Shaded}
\begin{Highlighting}[]
\KeywordTok{summarise}\NormalTok{(flights}
\NormalTok{  , }\DataTypeTok{distance_moyenne =} \KeywordTok{mean}\NormalTok{(distance)}
\NormalTok{  , }\DataTypeTok{retard_max =} \KeywordTok{max}\NormalTok{(arr_delay, }\DataTypeTok{na.rm =} \OtherTok{TRUE}\NormalTok{)}
\NormalTok{)}
\end{Highlighting}
\end{Shaded}

\begin{verbatim}
  ##   distance_moyenne retard_max
  ## 1         1039.913       1272
\end{verbatim}

\pause 

\textbf{Remarque} Comme toutes les fonctions de \texttt{dplyr},
\texttt{summarise()} prend un \texttt{data.frame} en entrée et produit
un \texttt{data.frame} en sortie.

\end{frame}

\begin{frame}[fragile]{Ventiler des traitements avec
\texttt{group\_by()}}

Appliqué au préalable à un \texttt{data.frame}, \texttt{group\_by()}
ventile tous les traitements ultérieurs selon les modalités d'une ou
plusieurs variables.

\begin{Shaded}
\begin{Highlighting}[]
\NormalTok{flights_bymonth <-}\StringTok{ }\KeywordTok{group_by}\NormalTok{(flights, month)}
\KeywordTok{summarise}\NormalTok{(flights_bymonth}
\NormalTok{  , }\DataTypeTok{distance_moyenne =} \KeywordTok{mean}\NormalTok{(distance)}
\NormalTok{  , }\DataTypeTok{retard_max =} \KeywordTok{max}\NormalTok{(arr_delay, }\DataTypeTok{na.rm =} \OtherTok{TRUE}\NormalTok{)}
\NormalTok{)[}\DecValTok{1}\OperatorTok{:}\DecValTok{3}\NormalTok{, ]}
\end{Highlighting}
\end{Shaded}

\begin{verbatim}
  ##   month distance_moyenne retard_max
  ## 1     1         1006.844       1272
  ## 2     2         1000.982        834
  ## 3     3         1011.987        915
\end{verbatim}

\end{frame}

\begin{frame}[fragile]{Enchaîner des opérations avec
\texttt{\%\textgreater{}\%}}

L'utilisation des verbes de \texttt{dplyr} ne prend tout son intérêt que
quand ils sont enchaînés en utilisant l'opérateur \emph{pipe}
\texttt{\%\textgreater{}\%}.

\texttt{maTable\ \%\textgreater{}\%\ maFonction(param1,\ param2)} est
équivalent à \texttt{maFonction(maTable,\ param1,\ param2)}.

\pause Ainsi, l'\textbf{enchaînement de nombreuses opérations} devient
beaucoup plus facile à mettre en \oe uvre et à comprendre.

\pause \footnotesize

\begin{Shaded}
\begin{Highlighting}[]
\NormalTok{flights }\OperatorTok
\StringTok{  }\KeywordTok{group_by}\NormalTok{(year, month, day) }\OperatorTok
\StringTok{  }\KeywordTok{summarise}\NormalTok{(}
    \DataTypeTok{retard_arrivee =} \KeywordTok{mean}\NormalTok{(arr_delay, }\DataTypeTok{na.rm =} \OtherTok{TRUE}\NormalTok{),}
    \DataTypeTok{retard_depart =} \KeywordTok{mean}\NormalTok{(dep_delay, }\DataTypeTok{na.rm =} \OtherTok{TRUE}\NormalTok{)}
\NormalTok{  ) }\OperatorTok
\StringTok{  }\KeywordTok{filter}\NormalTok{(retard_arrivee }\OperatorTok{>}\StringTok{ }\DecValTok{30} \OperatorTok{|}\StringTok{ }\NormalTok{retard_depart }\OperatorTok{>}\StringTok{ }\DecValTok{30}\NormalTok{)}
\end{Highlighting}
\end{Shaded}

\end{frame}

\begin{frame}[fragile]{Fusionner des tables avec \texttt{*\_join()}}

\texttt{dplyr} dispose de nombreuses fonctions très utiles pour
fusionner une ou plusieurs tables ensemble, qui \textbf{s'inspirent très
fortement de SQL} :

\begin{itemize}
\tightlist
\item
  \texttt{a\ \%\textgreater{}\%\ left\_join(b,\ by\ =\ "id")} : fusionne
  \texttt{a} et \texttt{b} en conservant toutes les observations de
  \texttt{a};
\item
  \texttt{a\ \%\textgreater{}\%\ right\_join(b,\ by\ =\ "id")} :
  fusionne \texttt{a} et \texttt{b} en conservant toutes les
  observations de \texttt{b};
\item
  \texttt{a\ \%\textgreater{}\%\ inner\_join(b,\ by\ =\ "id")} :
  fusionne \texttt{a} et \texttt{b} en ne conservant que les
  observations dans \texttt{a} et \texttt{b};
\item
  \texttt{a\ \%\textgreater{}\%\ full\_join(b,\ by\ =\ "id")} : fusionne
  \texttt{a} et \texttt{b} en conservant toutes les observations.
\end{itemize}

\textbf{Pour en savoir plus} Une
\href{https://cran.r-project.org/web/packages/dplyr/vignettes/two-table.html}{\underline{vignette}}
est consacrée à la présentation des fonctions de \texttt{dplyr} portant
sur deux tables.

\end{frame}

\begin{frame}[fragile]{Comparaison de base R et de \texttt{dplyr}}

\texttt{dplyr} est particulièrement intéressant pour travailler sur des
données par groupe. On compare donc l'utilisation de \texttt{tapply()}
de base R avec \texttt{group\_by()} de \texttt{dplyr}.

\footnotesize

\begin{Shaded}
\begin{Highlighting}[]
\NormalTok{df <-}\StringTok{ }\KeywordTok{data.frame}\NormalTok{(}
  \DataTypeTok{x =} \KeywordTok{rnorm}\NormalTok{(}\FloatTok{1e6}\NormalTok{)}
\NormalTok{  , }\DataTypeTok{by =} \KeywordTok{sample}\NormalTok{(}\FloatTok{1e3}\NormalTok{, }\FloatTok{1e6}\NormalTok{, }\DataTypeTok{replace =} \OtherTok{TRUE}\NormalTok{)}
\NormalTok{)}

\KeywordTok{microbenchmark}\NormalTok{(}\DataTypeTok{times =}\NormalTok{ 10L}
\NormalTok{  , }\DataTypeTok{base =} \KeywordTok{tapply}\NormalTok{(df}\OperatorTok{$}\NormalTok{x, df}\OperatorTok{$}\NormalTok{by, sum)}
\NormalTok{  , }\DataTypeTok{dplyr =}\NormalTok{ df }\OperatorTok\StringTok{ }\KeywordTok{group_by}\NormalTok{(by) }\OperatorTok\StringTok{ }\KeywordTok{summarise}\NormalTok{(}\KeywordTok{sum}\NormalTok{(x))}
\NormalTok{)}
\NormalTok{  ## Unit: milliseconds}
\NormalTok{  ##   expr      min       lq     mean   median       uq}
\NormalTok{  ##   base 40.79184 41.45713 44.89822 44.29427 47.29994}
\NormalTok{  ##  dplyr 49.23555 53.48221 59.57438 54.48100 55.97304}
\NormalTok{  ##        max neval}
\NormalTok{  ##   54.05332    10}
\NormalTok{  ##  108.10298    10}
\end{Highlighting}
\end{Shaded}

\end{frame}

\subsection{data.table : un data.frame
optimisé}\label{data.table-un-data.frame-optimise}

\begin{frame}[fragile]{Philosophie de \texttt{data.table}}

Contrairement à \texttt{dplyr}, \texttt{data.table} ne cherche pas à se
substituer à base R mais à le compléter.

Il introduit un nouveau type d'objet, le \texttt{data.table}, qui
\textbf{hérite} du \texttt{data.frame} (tout \texttt{data.table} est un
\texttt{data.frame}).

Appliqué à un \texttt{data.table}, l'opérateur \texttt{{[}} est
\textbf{enrichi et optimisé}.

\begin{Shaded}
\begin{Highlighting}[]
\KeywordTok{library}\NormalTok{(data.table)}
\NormalTok{flights_DT <-}\StringTok{ }\KeywordTok{data.table}\NormalTok{(flights)}
\end{Highlighting}
\end{Shaded}

\textbf{Pour en savoir plus} Là encore des vignettes très pédagogiques
sont disponibles sur la
\href{https://cran.r-project.org/package=data.table}{\underline{page du \textit{package}}}.

\end{frame}

\begin{frame}[fragile]{L'opérateur \texttt{{[}} du \texttt{data.table} :
\texttt{i}, \texttt{j} et \texttt{by}}

La syntaxe de l'opérateur \texttt{{[}} appliqué à un \texttt{data.table}
est la suivante (\texttt{DT} représente le \texttt{data.table}):

\centering \large

\texttt{DT{[}i,\ j,\ by{]}}

\raggedright \normalsize

\begin{itemize}
\tightlist
\item
  \texttt{i} : sélectionner des observations selon une condition;
\item
  \texttt{j} : sélectionner ou \textbf{créer} une ou plusieurs
  variables;
\item
  \texttt{by} : ventiler les traitements selon les modalités d'une ou
  plusieurs variables.
\end{itemize}

\bigskip 

\textbf{Exemple} Retard quotidien maximal au mois de janvier.

\small

\begin{Shaded}
\begin{Highlighting}[]
\NormalTok{flights_DT[}
\NormalTok{  month }\OperatorTok{==}\StringTok{ }\DecValTok{1}\NormalTok{, }\KeywordTok{max}\NormalTok{(arr_delay, }\DataTypeTok{na.rm =} \OtherTok{TRUE}\NormalTok{), by =}\StringTok{ }\NormalTok{day}
\NormalTok{]}
\end{Highlighting}
\end{Shaded}

\end{frame}

\begin{frame}[fragile]{Sélectionner des observations avec \texttt{i}}

\small
Il est beaucoup plus simple et efficace de sélectionner des observations
dans un \texttt{data.table} que dans un \texttt{data.frame}:

\begin{itemize}
\item
  il n'y a pas à répéter le nom du \texttt{data.frame} dans
  \texttt{{[}};
\item
  il est possible d'indexer un \texttt{data.table} par une ou plusieurs
  \og clés \fg{} permettant une recherche souvent plus rapide.
\end{itemize}

\footnotesize

\begin{Shaded}
\begin{Highlighting}[]
\KeywordTok{setkey}\NormalTok{(flights_DT, origin)}
\KeywordTok{microbenchmark}\NormalTok{(}\DataTypeTok{times =}\NormalTok{ 100L}
\NormalTok{  , }\DataTypeTok{base =}\NormalTok{ flights[flights}\OperatorTok{$}\NormalTok{origin }\OperatorTok{==}\StringTok{ "JFK"}\NormalTok{,]}
\NormalTok{  , }\DataTypeTok{dt1 =}\NormalTok{ flights_DT[origin }\OperatorTok{==}\StringTok{ "JFK"}\NormalTok{]}
\NormalTok{  , }\DataTypeTok{dt2 =}\NormalTok{ flights_DT[}\KeywordTok{list}\NormalTok{(}\StringTok{"JFK"}\NormalTok{)]}
\NormalTok{)}
\NormalTok{  ## Unit: milliseconds}
\NormalTok{  ##  expr      min       lq     mean   median       uq}
\NormalTok{  ##  base 43.21609 49.84061 59.34352 52.46760 58.23084}
\NormalTok{  ##   dt1 10.95107 11.80423 14.56624 13.56939 16.50279}
\NormalTok{  ##   dt2 10.47420 11.13283 14.12447 13.31107 16.29555}
\NormalTok{  ##        max neval}
\NormalTok{  ##  215.79663   100}
\NormalTok{  ##   24.36293   100}
\NormalTok{  ##   26.08380   100}
\end{Highlighting}
\end{Shaded}

\end{frame}

\begin{frame}[fragile]{Calculer des statistiques avec \texttt{j}}

L'argument \texttt{j} permet de calculer des statistiques agrégées.

\small

\begin{Shaded}
\begin{Highlighting}[]
\NormalTok{flights_DT[, j =}\StringTok{ }\KeywordTok{list}\NormalTok{(}
  \DataTypeTok{distance_moyenne =} \KeywordTok{mean}\NormalTok{(distance)}
\NormalTok{  , }\DataTypeTok{retard_max =} \KeywordTok{max}\NormalTok{(arr_delay, }\DataTypeTok{na.rm =} \OtherTok{TRUE}\NormalTok{)}
\NormalTok{)]}
\NormalTok{  ##    distance_moyenne retard_max}
\NormalTok{  ## 1:         1039.913       1272}
\end{Highlighting}
\end{Shaded}

\normalsize 

Utilisé avec \texttt{:=} il permet de les refusionner automatiquement
avec les données d'origine.

\small

\begin{Shaded}
\begin{Highlighting}[]
\NormalTok{flights_DT <-}\StringTok{ }\NormalTok{flights_DT[, j }\OperatorTok{:}\ErrorTok{=}\StringTok{ }\KeywordTok{list}\NormalTok{(}
  \DataTypeTok{distance_moyenne =} \KeywordTok{mean}\NormalTok{(distance)}
\NormalTok{  , }\DataTypeTok{retard_max =} \KeywordTok{max}\NormalTok{(arr_delay, }\DataTypeTok{na.rm =} \OtherTok{TRUE}\NormalTok{)}
\NormalTok{)]}
\end{Highlighting}
\end{Shaded}

\end{frame}

\begin{frame}[fragile]{Ventiler des traitements avec \texttt{by} et
\texttt{keyby}}

L'argument \texttt{by} de \texttt{{[}} ventile tous les traitements
renseignés dans \texttt{j} selon les modalités d'une ou plusieurs
variables.

\begin{Shaded}
\begin{Highlighting}[]
\NormalTok{flights_DT[, j =}\StringTok{ }\KeywordTok{list}\NormalTok{(}
  \DataTypeTok{distance_moyenne =} \KeywordTok{mean}\NormalTok{(distance)}
\NormalTok{  , }\DataTypeTok{retard_max =} \KeywordTok{max}\NormalTok{(arr_delay, }\DataTypeTok{na.rm =} \OtherTok{TRUE}\NormalTok{)}
\NormalTok{), by =}\StringTok{ }\NormalTok{month][}\DecValTok{1}\OperatorTok{:}\DecValTok{3}\NormalTok{,]}
\NormalTok{  ##    month distance_moyenne retard_max}
\NormalTok{  ## 1:     1         1006.844       1272}
\NormalTok{  ## 2:    10         1038.876        688}
\NormalTok{  ## 3:    11         1050.305        796}
\end{Highlighting}
\end{Shaded}

\textbf{Remarque} Par défaut, \texttt{by} ordonne les résultats dans
l'ordre des groupes dans le \texttt{data.table}. \texttt{keyby} trie les
données selon la variable d'agrégation (comme \texttt{group\_by} de
\texttt{dplyr}).

\end{frame}

\begin{frame}[fragile]{Chaîner les opérations dans un
\texttt{data.table}}

Il est très facile de chaîner les opérations sur un \texttt{data.table}
en enchaînant les \texttt{{[}}.

\begin{Shaded}
\begin{Highlighting}[]
\NormalTok{flights_DT[}
\NormalTok{  , j =}\StringTok{ }\KeywordTok{list}\NormalTok{(}
    \DataTypeTok{retard_arrivee =} \KeywordTok{mean}\NormalTok{(arr_delay, }\DataTypeTok{na.rm =} \OtherTok{TRUE}\NormalTok{)}
\NormalTok{    , }\DataTypeTok{retard_depart =} \KeywordTok{mean}\NormalTok{(dep_delay, }\DataTypeTok{na.rm =} \OtherTok{TRUE}\NormalTok{)}
\NormalTok{  )}
\NormalTok{  , keyby =}\StringTok{ }\KeywordTok{list}\NormalTok{(year, month, day)}
\NormalTok{][retard_arrivee }\OperatorTok{>}\StringTok{ }\DecValTok{30} \OperatorTok{|}\StringTok{ }\NormalTok{retard_depart }\OperatorTok{>}\StringTok{ }\DecValTok{30}\NormalTok{]}
\end{Highlighting}
\end{Shaded}

\textbf{Remarque} Ces chaînages sont possibles avec un
\texttt{data.table} mais pas avec un \texttt{data.frame}.

\end{frame}

\begin{frame}[fragile]{Comparaison de base R, \texttt{dplyr} et
\texttt{data.table}}

\footnotesize

\begin{Shaded}
\begin{Highlighting}[]
\CommentTok{# Conversion de la table de test en data.table}
\NormalTok{dt <-}\StringTok{ }\KeywordTok{data.table}\NormalTok{(df)}

\KeywordTok{microbenchmark}\NormalTok{(}\DataTypeTok{times =}\NormalTok{ 10L}
\NormalTok{  , }\DataTypeTok{base =} \KeywordTok{tapply}\NormalTok{(df}\OperatorTok{$}\NormalTok{x, df}\OperatorTok{$}\NormalTok{by, sum)}
\NormalTok{  , }\DataTypeTok{dplyr =}\NormalTok{ df }\OperatorTok\StringTok{ }\KeywordTok{group_by}\NormalTok{(by) }\OperatorTok\StringTok{ }\KeywordTok{summarise}\NormalTok{(}\KeywordTok{sum}\NormalTok{(x))}
\NormalTok{  , }\DataTypeTok{data.table =}\NormalTok{ dt[, }\KeywordTok{sum}\NormalTok{(x), }\DataTypeTok{keyby =}\NormalTok{ by]}
\NormalTok{)}
\end{Highlighting}
\end{Shaded}

\vspace{-5mm}

\begin{verbatim}
  ##         expr       lq     mean       uq
  ## 1       base 40.50007 51.00310 61.15298
  ## 2      dplyr 50.49077 58.10007 63.03932
  ## 3 data.table 22.53888 27.79945 30.58824
\end{verbatim}

\normalsize

\textbf{Pour en savoir plus} Cette discussion sur
\href{http://stackoverflow.com/questions/21435339/data-table-vs-dplyr-can-one-do-something-well-the-other-cant-or-does-poorly}{\underline{stackoverflow.com}}
(notamment entre les auteurs des \emph{packages}) aborde les avantages
et les inconvénients de \texttt{dplyr} et \texttt{data.table}.

\end{frame}

\subsection{Aller plus loin avec R}\label{aller-plus-loin-avec-r}

\begin{frame}{Les limites du logiciel}

Les outils présentés jusqu'à présent correspondent à une utilisation
\og classique \fg{} de R : production d'une enquête, redressements,
études.

~

Il arrive néanmoins que certains traitements soient rendus
\textbf{difficiles par les caractéristiques du logiciel} :

\begin{itemize}
\tightlist
\item
  travail sur des volumes de données impossibles à loger en mémoire;
\item
  temps de calcul trop longs et impossibles à réduire.
\end{itemize}

~

Dans ce genre de situations, la solution consiste en général à utiliser
R comme une \textbf{interface} vers des techniques ou langages
susceptibles de répondre au problème posé.

\end{frame}

\begin{frame}[fragile]{\large Se connecter à des bases de données}

Une autre solution pour exploiter de grands volumes de données dans R
est de l'utiliser pour \textbf{interroger des bases de données},
\emph{via} par exemple le \emph{package} \texttt{RPostgreSQL}.

\footnotesize

\begin{Shaded}
\begin{Highlighting}[]
\KeywordTok{library}\NormalTok{(RPostgreSQL)}

\CommentTok{# Connexion à la base de données maBdd}
\NormalTok{drv <-}\StringTok{ }\KeywordTok{dbDriver}\NormalTok{(}\StringTok{"PostgreSQL"}\NormalTok{)}
\NormalTok{con <-}\StringTok{ }\KeywordTok{dbConnect}\NormalTok{(drv, }\DataTypeTok{dbname =} \StringTok{"maBdd"}
\NormalTok{  , }\DataTypeTok{host =} \StringTok{"localhost"}\NormalTok{, }\DataTypeTok{port =} \DecValTok{5432}
\NormalTok{  , }\DataTypeTok{user =} \StringTok{"utilisateur"}\NormalTok{, }\DataTypeTok{password =} \StringTok{"motDePasse"}
\NormalTok{)}

\CommentTok{# Requête SQL sur la table maTable}
\KeywordTok{dbGetQuery}\NormalTok{(con, }\StringTok{"SELECT COUNT(*) FROM maTable"}\NormalTok{)}
\end{Highlighting}
\end{Shaded}

\normalsize

\textbf{Remarque} Différents \emph{packages} permettent de se connecter
à différents types de base de données : \texttt{RMySQl} pour MySQL, etc.

\end{frame}

\begin{frame}[fragile]{\large Se connecter à des bases de données avec
\texttt{dplyr}}

\texttt{dplyr} a la particularité de pouvoir fonctionner de façon
totalement transparente sur des bases de données de différents types.

\footnotesize

\begin{Shaded}
\begin{Highlighting}[]
\KeywordTok{library}\NormalTok{(dplyr)}

\CommentTok{# Connexion à la base de données maBdd}
\NormalTok{con <-}\StringTok{ }\KeywordTok{src_postgres}\NormalTok{(}
  \DataTypeTok{dbname =} \StringTok{"maBdd"}\NormalTok{, }\DataTypeTok{host =} \StringTok{"localhost"}\NormalTok{, }\DataTypeTok{port =} \DecValTok{5432}
\NormalTok{  , }\DataTypeTok{user =} \StringTok{"utilisateur"}\NormalTok{, }\DataTypeTok{password =} \StringTok{"motDePasse"}
\NormalTok{)}

\CommentTok{# Requête SQL sur la table maTable...}
\KeywordTok{tbl}\NormalTok{(con, }\StringTok{"SELECT COUNT(*) FROM maTable"}\NormalTok{)}

\CommentTok{# ... ou utilisation des verbes de dplyr}
\KeywordTok{tbl}\NormalTok{(con) }\OperatorTok\StringTok{ }\KeywordTok{summarise}\NormalTok{(}\KeywordTok{n}\NormalTok{())}
\end{Highlighting}
\end{Shaded}

\end{frame}

\begin{frame}[fragile]{\large Paralléliser des traitements avec
\texttt{parallel} (1)}

La plupart des ordinateurs possèdent aujourd'hui plusieurs c\oe urs
(\emph{core}) susceptibles de mener des traitements \textbf{en
parallèle} (8 sur chaque serveur d'AUS par exemple).

Par défaut, R n'expoite qu'un seul c\oe ur : le \emph{package}
\texttt{parallel} (mais aussi les \emph{packages} \texttt{snow} ou
\texttt{foreach} par exemple) permettent de \textbf{paralléliser des
structures du type \texttt{*apply}}.

Ce type d'opérations est composé de plusieurs étapes :

\begin{enumerate}
\def\labelenumi{\arabic{enumi}.}
\tightlist
\item
  Création et paramétrage du \og \textit{cluster} \fg{} de c\oe urs à
  utiliser (chargement des fonctions et \emph{packages} nécessaires sur
  chaque c\oe ur);
\item
  Lancement du traitement parallélisé avec \texttt{parLapply()};
\item
  Arrêt des processus du \emph{cluster} avec \texttt{stopCluster()}.
\end{enumerate}

\end{frame}

\begin{frame}[fragile]{\large Paralléliser des traitements avec
\texttt{parallel} (2)}

Dans cet exemple, on cherche à appliquer la fonction \texttt{f} à chaque
matrice de la liste \texttt{l}.

\footnotesize

\begin{Shaded}
\begin{Highlighting}[]
\KeywordTok{library}\NormalTok{(MASS)}
\NormalTok{f <-}\StringTok{ }\ControlFlowTok{function}\NormalTok{(x) }\KeywordTok{rowSums}\NormalTok{(}\KeywordTok{ginv}\NormalTok{(x))}
\NormalTok{l <-}\StringTok{ }\KeywordTok{lapply}\NormalTok{(}\DecValTok{1}\OperatorTok{:}\DecValTok{100}\NormalTok{, }\ControlFlowTok{function}\NormalTok{(x) }\KeywordTok{matrix}\NormalTok{(}\KeywordTok{runif}\NormalTok{(}\FloatTok{1e4}\NormalTok{), }\DataTypeTok{ncol =} \FloatTok{1e2}\NormalTok{))}

\CommentTok{# Création et paramétrage du cluster}
\KeywordTok{library}\NormalTok{(parallel)}
\NormalTok{cl <-}\StringTok{ }\KeywordTok{makeCluster}\NormalTok{(}\DecValTok{4}\NormalTok{)}
\KeywordTok{clusterEvalQ}\NormalTok{(cl, }\KeywordTok{library}\NormalTok{(MASS))}
\KeywordTok{clusterExport}\NormalTok{(cl, }\StringTok{"f"}\NormalTok{)}

\CommentTok{# Lancement du calcul parallélisé}
\KeywordTok{parLapply}\NormalTok{(cl, l, f)}

\CommentTok{# Arrêt des processus du cluster}
\KeywordTok{stopCluster}\NormalTok{(cl)}
\end{Highlighting}
\end{Shaded}

\end{frame}

\begin{frame}[fragile]{\large Paralléliser des traitements avec
\texttt{parallel} (3)}

\begin{Shaded}
\begin{Highlighting}[]
\KeywordTok{microbenchmark}\NormalTok{(}\DataTypeTok{times =} \DecValTok{10}
\NormalTok{  , }\KeywordTok{lapply}\NormalTok{(l, f)}
\NormalTok{  , }\KeywordTok{parLapply}\NormalTok{(cl, l, f)}
\NormalTok{)}
\NormalTok{  ## Unit: milliseconds}
\NormalTok{  ##                 expr      min       lq     mean}
\NormalTok{  ##         lapply(l, f) 685.5965 689.0597 711.2954}
\NormalTok{  ##  parLapply(cl, l, f) 394.6888 426.4845 461.8367}
\NormalTok{  ##    median       uq      max neval}
\NormalTok{  ##  697.1698 721.4244 792.8151    10}
\NormalTok{  ##  445.0535 498.3799 591.0796    10}
\end{Highlighting}
\end{Shaded}

\end{frame}

\begin{frame}[fragile]{\texttt{Rcpp} : un package R pour utiliser C++
(1)}

Le \emph{package} \texttt{Rcpp} permet d'intégrer facilement des
fonctions codées en C++ dans un programme R.

\footnotesize

\begin{Shaded}
\begin{Highlighting}[]
\KeywordTok{library}\NormalTok{(Rcpp)}
\KeywordTok{cppFunction}\NormalTok{(}\StringTok{'int add(int x, int y) \{}
\StringTok{  int result = x + y;}
\StringTok{  return result;}
\StringTok{\}'}\NormalTok{)}

\KeywordTok{add}\NormalTok{(}\DecValTok{1}\NormalTok{, }\DecValTok{2}\NormalTok{)}
\NormalTok{  ## [1] 3}
\end{Highlighting}
\end{Shaded}

\normalsize

\textbf{Remarque} Il est également possible de soumettre un fichier
contenant des fonctions C++ écrit par ailleurs à l'aide de la fonction
\texttt{sourceCpp()}.

\textbf{Pour en savoir plus}
\href{http://adv-r.had.co.nz/Rcpp.html}{\underline{\textit{Advanced R}}}

\end{frame}

\begin{frame}[fragile]{\texttt{Rcpp} : un package R pour utiliser C++
(2)}

Contrairement à R, C++ est un langage de bas niveau : les boucles y sont
en particulier extrêmement rapides.

\textbf{Exemple} Somme cumulée par colonne

\footnotesize

\begin{Shaded}
\begin{Highlighting}[]
\CommentTok{# Fonction C++}
\KeywordTok{cppFunction}\NormalTok{(}\StringTok{'NumericMatrix cumColSumsC(NumericMatrix x) \{}
\StringTok{  int nrow = x.nrow(), ncol = x.ncol();}
\StringTok{  NumericMatrix out(nrow, ncol);}
\StringTok{  for (int j = 0; j < ncol; j++) \{}
\StringTok{    double acc = 0;}
\StringTok{    for(int i = 0; i < nrow; i++)\{}
\StringTok{      acc += x(i, j);}
\StringTok{      out(i, j) = acc;}
\StringTok{    \}}
\StringTok{  \}}
\StringTok{  return out;}
\StringTok{\}'}\NormalTok{)}
\end{Highlighting}
\end{Shaded}

\end{frame}

\begin{frame}[fragile]{\texttt{Rcpp} : un package R pour utiliser C++
(3)}

\footnotesize

\begin{Shaded}
\begin{Highlighting}[]
\CommentTok{# Fonction R}
\NormalTok{cumColSumsR <-}\StringTok{ }\ControlFlowTok{function}\NormalTok{(x)\{}
  \KeywordTok{apply}\NormalTok{(x, }\DecValTok{2}\NormalTok{, cumsum)}
\NormalTok{\}}

\CommentTok{# Les deux fonctions produisent les mêmes résultats...}
\NormalTok{x <-}\StringTok{ }\KeywordTok{matrix}\NormalTok{(}\KeywordTok{rnorm}\NormalTok{(}\FloatTok{1e6}\NormalTok{), }\DataTypeTok{ncol =} \FloatTok{1e2}\NormalTok{)}
\KeywordTok{all.equal}\NormalTok{(}\KeywordTok{cumColSumsR}\NormalTok{(x), }\KeywordTok{cumColSumsC}\NormalTok{(x))}
\NormalTok{  ## [1] TRUE}

\CommentTok{# ... mais cumColSumsC() est beaucoup plus rapide !}
\KeywordTok{summary}\NormalTok{(}\KeywordTok{microbenchmark}\NormalTok{(}\DataTypeTok{times =} \DecValTok{10}
\NormalTok{  , }\KeywordTok{cumColSumsR}\NormalTok{(x)}
\NormalTok{  , }\KeywordTok{cumColSumsC}\NormalTok{(x)}
\NormalTok{))[, }\KeywordTok{c}\NormalTok{(}\StringTok{"expr"}\NormalTok{, }\StringTok{"lq"}\NormalTok{, }\StringTok{"mean"}\NormalTok{, }\StringTok{"uq"}\NormalTok{)]}
\NormalTok{  ##             expr        lq      mean       uq}
\NormalTok{  ## 1 cumColSumsR(x) 27.054698 45.593325 31.76764}
\NormalTok{  ## 2 cumColSumsC(x)  5.344443  7.852879  9.91970}
\end{Highlighting}
\end{Shaded}

\end{frame}

\end{document}
