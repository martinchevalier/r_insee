\documentclass[12pt,ignorenonframetext,]{beamer}
\setbeamertemplate{caption}[numbered]
\setbeamertemplate{caption label separator}{: }
\setbeamercolor{caption name}{fg=normal text.fg}
\beamertemplatenavigationsymbolsempty
\usepackage{lmodern}
\usepackage{amssymb,amsmath}
\usepackage{ifxetex,ifluatex}
\usepackage{fixltx2e} % provides \textsubscript
\ifnum 0\ifxetex 1\fi\ifluatex 1\fi=0 % if pdftex
  \usepackage[T1]{fontenc}
  \usepackage[utf8]{inputenc}
\else % if luatex or xelatex
  \ifxetex
    \usepackage{mathspec}
  \else
    \usepackage{fontspec}
  \fi
  \defaultfontfeatures{Ligatures=TeX,Scale=MatchLowercase}
\fi
% use upquote if available, for straight quotes in verbatim environments
\IfFileExists{upquote.sty}{\usepackage{upquote}}{}
% use microtype if available
\IfFileExists{microtype.sty}{%
\usepackage{microtype}
\UseMicrotypeSet[protrusion]{basicmath} % disable protrusion for tt fonts
}{}
\newif\ifbibliography
\hypersetup{
            pdftitle={Formation R Perfectionnement},
            pdfborder={0 0 0},
            breaklinks=true}
\urlstyle{same}  % don't use monospace font for urls
\usepackage{color}
\usepackage{fancyvrb}
\newcommand{\VerbBar}{|}
\newcommand{\VERB}{\Verb[commandchars=\\\{\}]}
\DefineVerbatimEnvironment{Highlighting}{Verbatim}{commandchars=\\\{\}}
% Add ',fontsize=\small' for more characters per line
\newenvironment{Shaded}{}{}
\newcommand{\KeywordTok}[1]{\textcolor[rgb]{0.00,0.00,1.00}{#1}}
\newcommand{\DataTypeTok}[1]{#1}
\newcommand{\DecValTok}[1]{#1}
\newcommand{\BaseNTok}[1]{#1}
\newcommand{\FloatTok}[1]{#1}
\newcommand{\ConstantTok}[1]{#1}
\newcommand{\CharTok}[1]{\textcolor[rgb]{0.00,0.50,0.50}{#1}}
\newcommand{\SpecialCharTok}[1]{\textcolor[rgb]{0.00,0.50,0.50}{#1}}
\newcommand{\StringTok}[1]{\textcolor[rgb]{0.00,0.50,0.50}{#1}}
\newcommand{\VerbatimStringTok}[1]{\textcolor[rgb]{0.00,0.50,0.50}{#1}}
\newcommand{\SpecialStringTok}[1]{\textcolor[rgb]{0.00,0.50,0.50}{#1}}
\newcommand{\ImportTok}[1]{#1}
\newcommand{\CommentTok}[1]{\textcolor[rgb]{0.00,0.50,0.00}{#1}}
\newcommand{\DocumentationTok}[1]{\textcolor[rgb]{0.00,0.50,0.00}{#1}}
\newcommand{\AnnotationTok}[1]{\textcolor[rgb]{0.00,0.50,0.00}{#1}}
\newcommand{\CommentVarTok}[1]{\textcolor[rgb]{0.00,0.50,0.00}{#1}}
\newcommand{\OtherTok}[1]{\textcolor[rgb]{1.00,0.25,0.00}{#1}}
\newcommand{\FunctionTok}[1]{#1}
\newcommand{\VariableTok}[1]{#1}
\newcommand{\ControlFlowTok}[1]{\textcolor[rgb]{0.00,0.00,1.00}{#1}}
\newcommand{\OperatorTok}[1]{#1}
\newcommand{\BuiltInTok}[1]{#1}
\newcommand{\ExtensionTok}[1]{#1}
\newcommand{\PreprocessorTok}[1]{\textcolor[rgb]{1.00,0.25,0.00}{#1}}
\newcommand{\AttributeTok}[1]{#1}
\newcommand{\RegionMarkerTok}[1]{#1}
\newcommand{\InformationTok}[1]{\textcolor[rgb]{0.00,0.50,0.00}{#1}}
\newcommand{\WarningTok}[1]{\textcolor[rgb]{0.00,0.50,0.00}{\textbf{#1}}}
\newcommand{\AlertTok}[1]{\textcolor[rgb]{1.00,0.00,0.00}{#1}}
\newcommand{\ErrorTok}[1]{\textcolor[rgb]{1.00,0.00,0.00}{\textbf{#1}}}
\newcommand{\NormalTok}[1]{#1}
\usepackage{graphicx,grffile}
\makeatletter
\def\maxwidth{\ifdim\Gin@nat@width>\linewidth\linewidth\else\Gin@nat@width\fi}
\def\maxheight{\ifdim\Gin@nat@height>\textheight0.8\textheight\else\Gin@nat@height\fi}
\makeatother
% Scale images if necessary, so that they will not overflow the page
% margins by default, and it is still possible to overwrite the defaults
% using explicit options in \includegraphics[width, height, ...]{}
\setkeys{Gin}{width=\maxwidth,height=\maxheight,keepaspectratio}

% Prevent slide breaks in the middle of a paragraph:
\widowpenalties 1 10000
\raggedbottom

\AtBeginPart{
  \let\insertpartnumber\relax
  \let\partname\relax
  \frame{\partpage}
}
\AtBeginSection{
  \ifbibliography
  \else
    \let\insertsectionnumber\relax
    \let\sectionname\relax
    \frame{\sectionpage}
  \fi
}
\AtBeginSubsection{
  \let\insertsubsectionnumber\relax
  \let\subsectionname\relax
  \frame{\subsectionpage}
}

\setlength{\parindent}{0pt}
\setlength{\parskip}{6pt plus 2pt minus 1pt}
\setlength{\emergencystretch}{3em}  % prevent overfull lines
\providecommand{\tightlist}{%
  \setlength{\itemsep}{0pt}\setlength{\parskip}{0pt}}
\setcounter{secnumdepth}{0}

\usepackage[french]{babel}
\usepackage{lmodern}
\usepackage{graphicx}
\usepackage{xcolor}
\usepackage{textcomp} 
\usepackage{amsmath, amsfonts, amssymb, amsthm}
\usepackage{booktabs,multirow}
\usepackage{setspace}
\usepackage{float}
\usepackage{pgfpages}
\usepackage{colortbl}
\usepackage{epstopdf}
\usepackage{framed}



\definecolor{shadecolor}{RGB}{248,248,248}
\definecolor{grayInsee}{HTML}{5a5758}
\definecolor{redInsee}{HTML}{ed1443}

%Everything about the notes and formatting of the notepage
\setbeamertemplate{note page}{%
	Notes personnelles
	\insertnote%
}


\setbeamertemplate{navigation symbols}{}
\usetheme{default} %Malmoe not bad
\setbeamertemplate{footline}[frame number]


%\setbeamerfont{frametitle}{size=\normalsize}
%\setbeamerfont{framesubtitle}{size=\Large}
%\setbeamercolor{frametitle}{fg=grayInsee}
%\setbeamercolor{framesubtitle}{fg=redInsee}
\setbeamercolor{title}{fg=grayInsee}
\setbeamercolor{subsection in toc}{fg=grayInsee}
\setbeamertemplate{frametitle}{%
	\large \textcolor{grayInsee}{\subsecname}
	\\ \vspace{0.1cm} \Large \textcolor{redInsee}{\insertframetitle}
}
%\setbeamertemplate{frametitle}{%
%	\large \textcolor{grayInsee}{
%		\ifx\intertsubsection\emptyset
%			\secname \\ \vspace{0.1cm} 
%		\else 
%			\subsecname \\ \vspace{0.1cm}
%		\fi
%	}
%	\Large \textcolor{redInsee}{\insertframetitle}
%}
\setbeamercolor{local structure}{fg=redInsee}

\AtBeginSection[]
{\ifnum \thesection>1
  \begin{frame}
  \vfill
  \begin{center}
  \LARGE
  \textcolor{grayInsee}{\insertsectionhead}
  \end{center}
  \vfill
  \end{frame}
\else
\fi
}

\AtBeginSubsection[]{}

\title{\Large Formation \textbf{R} Perfectionnement}

\institute{ \includegraphics[height = 2.5cm]{../figures/Logo_Insee.png}\\ ~ \\ \normalsize Martin \textsc{Chevalier} (Insee)}

\author{15-16 janvier 2018}

\date{}

\renewenvironment{Shaded}{\begin{snugshade}}{\end{snugshade}}

\newcommand{\aparte}[2]{
	{\small\textsf{\textbf{#1} #2}}
}

%\usepackage{enumitem}
%\setlist{nolistsep}

\usepackage{tikz}
\usetikzlibrary{shapes,arrows,calc, positioning}
\tikzstyle{input} = [draw, rectangle,rounded corners, text width=2.5cm, fill=green!20, node distance=0.5cm, minimum height=2em, text centered]
\tikzstyle{output} = [draw, ellipse,fill=red!20, node distance=0.5cm, minimum height=2em, text centered]
\tikzstyle{block} = [rectangle, draw, fill=blue!20, 
    text width=1.5cm, text centered, minimum height=2em, node distance = 0.5cm]
\tikzstyle{line} = [draw, -latex', shorten >=2pt, shorten <=2pt]
\tikzset{
  invisible/.style={opacity=0},
  visible on/.style={alt={#1{}{invisible}}},
  alt/.code args={<#1>#2#3}{%
    \alt<#1>{\pgfkeysalso{#2}}{\pgfkeysalso{#3}} % \pgfkeysalso doesn't change the path
  },
}

%\usepackage{pgfpages}
%\mode<handout>{
%	%\setbeamercolor{background canvas}{bg=black!20}
%	\pgfpagesuselayout{2 on 1}[border shrink=2mm]
%}

\title{Formation R Perfectionnement}
\date{}

\begin{document}
\frame{\titlepage}

\section{Réaliser des graphiques avec
R}\label{realiser-des-graphiques-avec-r}

\subsection*{Réaliser des graphiques avec R}

\begin{frame}{R et la réalisation de graphiques}

La réalisation de graphiques dans un logiciel statistique est une
opération souvent longue et complexe.

Dans la plupart des cas, l'ajustement fin des paramètres par le biais de
lignes de code relève de la gageure.

\pause R dispose néanmoins de plusieurs caractéristiques qui facilitent
la réalisation de graphiques :

\begin{itemize}
\tightlist
\item
  \textbf{souplesse} : la très grande variété des types d'objets
  simplifie les paramétrages ;
\item
  \textbf{rigueur} : la dimension fonctionnelle du langage aide à
  systématiser l'utilisation des paramètres graphiques ;
\item
  \textbf{adaptabilité} : la liberté de développement de modules
  complémentaires rend possible de profondes innovations dans la
  conception des graphiques.
\end{itemize}

\end{frame}

\begin{frame}[fragile]{Base R ou \texttt{ggplot2} ?}

Il existe aujourd'hui troix principaux paradigmes pour produire des
graphiques avec R :

\begin{itemize}
\item
  les fonctionnalités de base du logiciel du \emph{package}
  \texttt{graphics};
\item
  les fonctionnalités plus élaborées des \emph{packages} \texttt{grid}
  et \texttt{lattice} (non-abordées dans cette formation);
\item
  la \og grammaire des graphiques \fg{} du \emph{package}
  \texttt{ggplot2}.
\end{itemize}

\bigskip 

\underline{Plan de la partie}

\bigskip

\tableofcontents[currentsection, sectionstyle = hide, subsectionstyle = show/show/hide]

\end{frame}

\begin{frame}[fragile]{Données d'exemple : table \texttt{mpg} de
\texttt{ggplot2}}

\small

La plupart des exemples de cette partie sont produits à partir de la
table \texttt{mpg} du \emph{package} \texttt{ggplot2}.

\footnotesize

\begin{Shaded}
\begin{Highlighting}[]
\KeywordTok{library}\NormalTok{(ggplot2)}
\KeywordTok{dim}\NormalTok{(mpg)}
\NormalTok{  ## [1] 234  11}
\KeywordTok{names}\NormalTok{(mpg)}
\NormalTok{  ##  [1] "manufacturer" "model"        "displ"       }
\NormalTok{  ##  [4] "year"         "cyl"          "trans"       }
\NormalTok{  ##  [7] "drv"          "cty"          "hwy"         }
\NormalTok{  ## [10] "fl"           "class"}
\end{Highlighting}
\end{Shaded}

\pause \vspace{-0.4cm} \small

\begin{itemize}
\tightlist
\item
  \texttt{displ} : cylindrée;
\item
  \texttt{drv} : transmission (\texttt{f} traction, \texttt{r}
  propulsion, \texttt{4} quatre roues motrices);
\item
  \texttt{cty} et \texttt{hwy} : nombre de \emph{miles} parcourus par
  \emph{gallon} d'essence en ville et sur autoroute respectivement.
\end{itemize}

\end{frame}

\subsection{\texorpdfstring{Réaliser des graphiques avec
\texttt{graphics}}{Réaliser des graphiques avec graphics}}\label{realiser-des-graphiques-avec-graphics}

\begin{frame}[fragile]{\large Beaucoup de fonctions, des paramètres
communs}

La création de graphiques avec le \emph{package} de base
\texttt{graphics} s'appuie sur la \textbf{fonction \texttt{plot()}}
ainsi que sur des \textbf{fonctions spécifiques} :

\begin{itemize}
\tightlist
\item
  \texttt{plot(hist(x))}, \texttt{plot(density(x))} : histogrammes et
  densités;
\item
  \texttt{plot(ts)} : représentation de séries chronologiques;
\item
  \texttt{plot(x,\ y)} : nuages de points;
\item
  \texttt{barplot(table(x))} et \texttt{pie(table(x))} : diagrammes en
  bâtons et circulaires.
\end{itemize}

\pause Si ce n'est quelques \textbf{arguments spécifiques}, ces
fonctions partagent un ensemble de \textbf{paramètres graphiques
communs}.

\pause 

\textbf{Pour en savoir plus} Le site
\href{http://www.statmethods.net/graphs/}{\underline{statmethods.net}}
recense et illustre la plupart des fonctions du \emph{package}
\texttt{graphics}.

\end{frame}

\begin{frame}[fragile]{Histogrammes et densités}

Les fonctions \texttt{histogram()} et \texttt{density()} calculent les
statistiques ensuite utilisées par la fonction \texttt{plot()} pour
construire les graphiques.

\pause Arguments spécifiques à \texttt{hist()} : \vspace{-0.3cm}

\begin{itemize}
\tightlist
\item
  \texttt{breaks} : méthode pour déterminer les limites des classes;
\item
  \texttt{labels\ =\ TRUE} : ajoute l'effectif de chaque classe.
\end{itemize}

\pause Arguments spécifiques à \texttt{density()} : \vspace{-0.3cm}

\begin{itemize}
\tightlist
\item
  \texttt{bw} : largeur de la fenêtre utilisée par la fonction de
  lissage;
\item
  \texttt{kernel} : fonction de lissage utilisée.
\end{itemize}

\pause

\textbf{Remarque} L'argument \texttt{plot} de la fonction
\texttt{hist()} (\texttt{TRUE} par défaut) affiche automatiquement un
graphique, sans avoir à appeler explicitement la fonction
\texttt{plot()}.

\end{frame}

\begin{frame}[fragile]{Histogrammes et densités}

\centering \footnotesize

\begin{Shaded}
\begin{Highlighting}[]
\KeywordTok{hist}\NormalTok{(mpg}\OperatorTok{$}\NormalTok{hwy, }\DataTypeTok{breaks =} \KeywordTok{seq}\NormalTok{(}\DecValTok{10}\NormalTok{, }\DecValTok{44}\NormalTok{, }\DataTypeTok{by =} \DecValTok{2}\NormalTok{), }
     \DataTypeTok{col =} \StringTok{"lightblue"}\NormalTok{, }\DataTypeTok{labels =} \OtherTok{TRUE}\NormalTok{)}
\end{Highlighting}
\end{Shaded}

\includegraphics[height=7cm]{partie2_files/figure-beamer/unnamed-chunk-3-1}

\end{frame}

\begin{frame}[fragile]{Histogrammes et densités}

\centering \footnotesize

\begin{Shaded}
\begin{Highlighting}[]
\KeywordTok{plot}\NormalTok{(}\KeywordTok{density}\NormalTok{(mpg}\OperatorTok{$}\NormalTok{hwy, }\DataTypeTok{bw =} \FloatTok{0.5}\NormalTok{, }\DataTypeTok{kernel =} \StringTok{"gaussian"}\NormalTok{))}
\end{Highlighting}
\end{Shaded}

\includegraphics[height=7cm]{partie2_files/figure-beamer/unnamed-chunk-4-1}

\end{frame}

\begin{frame}[fragile]{Séries chronologiques avec \texttt{plot(ts)}}

\centering \footnotesize

\begin{Shaded}
\begin{Highlighting}[]
\KeywordTok{class}\NormalTok{(AirPassengers)}
\NormalTok{  ## [1] "ts"}
\KeywordTok{plot}\NormalTok{(AirPassengers)}
\end{Highlighting}
\end{Shaded}

\includegraphics[height=6cm]{partie2_files/figure-beamer/unnamed-chunk-5-1}

\end{frame}

\begin{frame}[fragile]{Nuages de points avec \texttt{plot(x,\ y)}}

\centering \footnotesize

\begin{Shaded}
\begin{Highlighting}[]
\KeywordTok{plot}\NormalTok{(mpg}\OperatorTok{$}\NormalTok{displ, mpg}\OperatorTok{$}\NormalTok{hwy)}
\end{Highlighting}
\end{Shaded}

\includegraphics[height=6.5cm]{partie2_files/figure-beamer/unnamed-chunk-6-1}

\end{frame}

\begin{frame}[fragile]{Diagrammes en bâtons et circulaires}

La fonction \texttt{table()} permet de calculer les statistiques
utilisées ensuite par \texttt{barplot()} et \texttt{pie()} pour
construire les graphiques.

\pause Arguments spécifiques à \texttt{barplot()} : \vspace{-3mm}

\begin{itemize}
\tightlist
\item
  \texttt{horiz} : construit le graphique horizontalement;
\item
  \texttt{names.arg} : nom à afficher près des barres.
\end{itemize}

\pause Arguments spécifiques à \texttt{pie()} : \vspace{-3mm}

\begin{itemize}
\tightlist
\item
  \texttt{labels} : noms à afficher à côté des portions de disque;
\item
  \texttt{clockwise} : sens dans lequel sont représentées les modalités;
\item
  \texttt{init.angle} : point de départ en degrés.
\end{itemize}

\pause

\textbf{Remarque} Quand \texttt{barplot()} est appliqué à un tri croisé,
la couleur des barres varie et les paramètres deviennent utiles :
\vspace{-3mm}

\begin{itemize}
\tightlist
\item
  \texttt{beside} : position des barres;
\item
  \texttt{legend.text} : ajoute une légende avec le texte indiqué.
\end{itemize}

\end{frame}

\begin{frame}[fragile]{Diagrammes en bâtons et circulaires}

\centering \footnotesize

\begin{Shaded}
\begin{Highlighting}[]
\NormalTok{uni <-}\StringTok{ }\KeywordTok{table}\NormalTok{(mpg}\OperatorTok{$}\NormalTok{drv)}
\NormalTok{lab <-}\StringTok{ }\KeywordTok{c}\NormalTok{(}\StringTok{"4 roues"}\NormalTok{, }\StringTok{"Traction"}\NormalTok{, }\StringTok{"Propulsion"}\NormalTok{)}
\KeywordTok{barplot}\NormalTok{(uni, }\DataTypeTok{names.arg =}\NormalTok{ lab)}
\end{Highlighting}
\end{Shaded}

\includegraphics[height=6cm]{partie2_files/figure-beamer/unnamed-chunk-7-1}

\end{frame}

\begin{frame}[fragile]{Diagrammes en bâtons et circulaires}

\centering \footnotesize

\begin{Shaded}
\begin{Highlighting}[]
\KeywordTok{pie}\NormalTok{(uni, }\DataTypeTok{labels =} \KeywordTok{paste0}\NormalTok{(lab, }\StringTok{"}\CharTok{\textbackslash{}n}\StringTok{"}\NormalTok{, uni)}
\NormalTok{    , }\DataTypeTok{init.angle =} \DecValTok{90}\NormalTok{, }\DataTypeTok{clockwise =} \OtherTok{TRUE}\NormalTok{)}
\end{Highlighting}
\end{Shaded}

\includegraphics[height=6cm]{partie2_files/figure-beamer/unnamed-chunk-8-1}

\end{frame}

\begin{frame}[fragile]{Diagrammes en bâtons et circulaires}

\centering \footnotesize

\begin{Shaded}
\begin{Highlighting}[]
\NormalTok{bi <-}\StringTok{ }\KeywordTok{table}\NormalTok{(mpg}\OperatorTok{$}\NormalTok{drv, mpg}\OperatorTok{$}\NormalTok{year)}
\KeywordTok{barplot}\NormalTok{(bi, }\DataTypeTok{horiz =} \OtherTok{TRUE}\NormalTok{, }\DataTypeTok{beside =} \OtherTok{TRUE}\NormalTok{, }\DataTypeTok{legend.text =}\NormalTok{ lab)}
\end{Highlighting}
\end{Shaded}

\includegraphics[height=6cm]{partie2_files/figure-beamer/unnamed-chunk-9-1}

\end{frame}

\begin{frame}[fragile]{Couleur, forme et taille des objets}

Plusieurs paramètres permettent de modifier la couleur, la forme ou la
taille des éléments qui composent un graphique:

\begin{itemize}
\tightlist
\item
  \pause \texttt{pch} : entier ou caractère spécial indiquant la forme
  des points à représenter.
\end{itemize}

\includegraphics{partie2_files/figure-beamer/unnamed-chunk-10-1.pdf}

\begin{itemize}
\tightlist
\item
  \pause \texttt{col} : valeur indiquant la couleur du contour des
  formes utilisées. Peut être un entier (recyclé au-delà de 8), un nom
  ou un code RGB hexadécimal (du type \texttt{"\#FF1111"}). \small  
\end{itemize}

\includegraphics{partie2_files/figure-beamer/unnamed-chunk-11-1.pdf}

Pour certaines formes (\texttt{pch} entre 21 et 25), il est également
possible de modifier la couleur de remplissage avec \texttt{bg}.

\end{frame}

\begin{frame}[fragile]{Couleur, forme et taille des objets}

\textbf{Remarque} : la palette de couleurs accessibles en utilisant des
entiers est réduite. Il est possible de l'étendre considérablement
\emph{via} la fonction \texttt{colors()}.

\small

\begin{Shaded}
\begin{Highlighting}[]
\KeywordTok{colors}\NormalTok{()[}\DecValTok{1}\OperatorTok{:}\DecValTok{3}\NormalTok{]}
\NormalTok{  ## [1] "white"        "aliceblue"    "antiquewhite"}
\KeywordTok{length}\NormalTok{(}\KeywordTok{colors}\NormalTok{())}
\NormalTok{  ## [1] 657}
\KeywordTok{grep}\NormalTok{(}\StringTok{"blue"}\NormalTok{, }\KeywordTok{colors}\NormalTok{(), }\DataTypeTok{value =} \OtherTok{TRUE}\NormalTok{)[}\DecValTok{1}\OperatorTok{:}\DecValTok{3}\NormalTok{]}
\NormalTok{  ## [1] "aliceblue" "blue"      "blue1"}
\end{Highlighting}
\end{Shaded}

\pause \normalsize
- \texttt{cex} : utilisé dans une fonction \texttt{plot()}, \texttt{cex}
permet d'ajuster la taille des points qui le composent.

\includegraphics{partie2_files/figure-beamer/unnamed-chunk-13-1.pdf}

\end{frame}

\begin{frame}[fragile]{Couleur, forme et taille des objets}

La fonction \texttt{legend()} permet d'ajouter une légende.
\footnotesize \center

\pause \vspace{-0.4cm}

\begin{Shaded}
\begin{Highlighting}[]
\NormalTok{t <-}\StringTok{ }\KeywordTok{factor}\NormalTok{(mpg}\OperatorTok{$}\NormalTok{drv}
\NormalTok{  , }\DataTypeTok{labels =} \KeywordTok{c}\NormalTok{(}\StringTok{"4 roues"}\NormalTok{, }\StringTok{"Traction"}\NormalTok{, }\StringTok{"Propulsion"}\NormalTok{))}
\KeywordTok{plot}\NormalTok{(mpg}\OperatorTok{$}\NormalTok{displ, mpg}\OperatorTok{$}\NormalTok{hwy, }\DataTypeTok{pch =} \DecValTok{21}\NormalTok{, }\DataTypeTok{col =}\NormalTok{ t, }\DataTypeTok{bg =}\NormalTok{ t)}
\KeywordTok{legend}\NormalTok{(}\StringTok{"topright"}\NormalTok{, }\DataTypeTok{legend =} \KeywordTok{unique}\NormalTok{(t), }\DataTypeTok{pch =} \DecValTok{21}
\NormalTok{  , }\DataTypeTok{col =} \KeywordTok{unique}\NormalTok{(t), }\DataTypeTok{pt.bg =} \KeywordTok{unique}\NormalTok{(t))}
\end{Highlighting}
\end{Shaded}

\includegraphics{partie2_files/figure-beamer/unnamed-chunk-14-1.pdf}

\end{frame}

\begin{frame}[fragile]{Titres, texte et axes}

Les titres sont paramétrés à l'aide des fonctions suivantes :

\vspace{-0.3cm} - \texttt{main} pour ajouter le titre principal; -
\texttt{xlab} et \texttt{ylab} pour ajouter des titres aux axes.

\pause La fonction \texttt{text()} permet d'ajouter du texte sur le
graphique en le positionnant par ses coordonnées, éventuellement avec un
décalage (pour nommer des points par exemple).

\pause Il est également possible de paramétrer les axes :

\vspace{-0.3cm} - \texttt{xlim} et \texttt{ylim} spécifient les valeurs
minimales et maximales de chaque axe; - \texttt{axis()} est une fonction
qui permet d'ajouter un axe personnalisé.

\pause 

\textbf{Remarque} Pour produire un graphique sans axe et les rajouter
après, utiliser l'option \texttt{axes\ =\ FALSE} de la fonction
\texttt{plot()}.

\end{frame}

\begin{frame}[fragile]{Combinaison de plusieurs graphiques}

Par défaut l'utilisation de la fonction \texttt{plot()} produit un
nouveau graphique.

\pause Pour superposer différents graphiques, le plus simple est de
commencer par une instruction \texttt{plot()} puis de la compléter :

\begin{itemize}
\tightlist
\item
  avec \texttt{points()} pour ajouter des points;
\item
  avec \texttt{lines()} pour ajouter des lignes;
\item
  avec \texttt{abline()} pour ajouter des lignes d'après une équation;
\item
  avec \texttt{curve()} pour ajouter des courbes d'après une équation.
\end{itemize}

\pause 

\textbf{Exemple} Ajout d'une droite de régression au graphique de
\texttt{hwy} par \texttt{displ}.

\end{frame}

\begin{frame}[fragile]{Combinaison de plusieurs graphiques}

\footnotesize

\begin{Shaded}
\begin{Highlighting}[]
\NormalTok{reg <-}\StringTok{ }\KeywordTok{lm}\NormalTok{(hwy }\OperatorTok{~}\StringTok{ }\NormalTok{displ, }\DataTypeTok{data =}\NormalTok{ mpg)}
\KeywordTok{plot}\NormalTok{(mpg}\OperatorTok{$}\NormalTok{displ, mpg}\OperatorTok{$}\NormalTok{hwy)}
\KeywordTok{abline}\NormalTok{(}\DataTypeTok{a =}\NormalTok{ reg}\OperatorTok{$}\NormalTok{coefficients[}\DecValTok{1}\NormalTok{], }\DataTypeTok{b =}\NormalTok{ reg}\OperatorTok{$}\NormalTok{coefficients[}\DecValTok{2}\NormalTok{])}
\end{Highlighting}
\end{Shaded}

\includegraphics{partie2_files/figure-beamer/unnamed-chunk-15-1.pdf}

\end{frame}

\begin{frame}[fragile]{Paramètres généraux et disposition (1)}

Utilisée en dehors de la fonction \texttt{plot()}, la fonction
\texttt{par()} permet de définir l'ensemble des paramètres graphiques
globaux.

\pause Ses mots-clés les plus importants sont :

\begin{itemize}
\item
  \texttt{mfrow} : permet de disposer plusieurs graphiques côte-à-côte.

\begin{Shaded}
\begin{Highlighting}[]
\KeywordTok{par}\NormalTok{(}\DataTypeTok{mfrow =} \KeywordTok{c}\NormalTok{(}\DecValTok{1}\NormalTok{, }\DecValTok{2}\NormalTok{)) }\CommentTok{# 1 ligne et 2 colonnes}
\KeywordTok{par}\NormalTok{(}\DataTypeTok{mfrow =} \KeywordTok{c}\NormalTok{(}\DecValTok{3}\NormalTok{, }\DecValTok{2}\NormalTok{)) }\CommentTok{# 3 lignes et 2 colonnes}
\KeywordTok{par}\NormalTok{(}\DataTypeTok{mfrow =} \KeywordTok{c}\NormalTok{(}\DecValTok{1}\NormalTok{, }\DecValTok{1}\NormalTok{)) }\CommentTok{# 1 ligne et 1 colonne}
\end{Highlighting}
\end{Shaded}
\item
  \texttt{cex} : coefficient multiplicatif pour modifier la taille de
  l'ensemble des textes et symboles utilisés dans les graphiques (1 par
  défaut).
\end{itemize}

\pause 

\textbf{Pour en savoir plus} La
\href{http://stat.ethz.ch/R-manual/R-devel/library/graphics/html/par.html}{page
d'aide} de la fonction \texttt{par()} détaille toutes ces options.

\end{frame}

\begin{frame}[fragile]{Paramètres généraux et disposition (2)}

\begin{Shaded}
\begin{Highlighting}[]
\KeywordTok{par}\NormalTok{(}\DataTypeTok{mfrow =} \KeywordTok{c}\NormalTok{(}\DecValTok{1}\NormalTok{, }\DecValTok{2}\NormalTok{))}
\KeywordTok{plot}\NormalTok{(mpg}\OperatorTok{$}\NormalTok{displ, mpg}\OperatorTok{$}\NormalTok{hwy)}
\KeywordTok{plot}\NormalTok{(AirPassengers)}
\end{Highlighting}
\end{Shaded}

\includegraphics{partie2_files/figure-beamer/unnamed-chunk-17-1.pdf}

\end{frame}

\begin{frame}[fragile]{Exportation}

Pour exporter des graphiques depuis R, la démarche consiste à rediriger
le flux de production du graphiques vers un fichier à l'aide d'une
fonction du \emph{package} \texttt{grDevices}. Par exemple :

\pause 

\begin{Shaded}
\begin{Highlighting}[]
\KeywordTok{png}\NormalTok{(}\StringTok{"monGraphique.png"}\NormalTok{, }\DataTypeTok{width =} \DecValTok{10}\NormalTok{, }\DataTypeTok{height =} \DecValTok{8}
\NormalTok{    , }\DataTypeTok{unit =} \StringTok{"cm"}\NormalTok{, }\DataTypeTok{res =} \DecValTok{600}\NormalTok{)}
\KeywordTok{plot}\NormalTok{(mpg}\OperatorTok{$}\NormalTok{displ, mpg}\OperatorTok{$}\NormalTok{hwy)}
\KeywordTok{dev.off}\NormalTok{()}
\end{Highlighting}
\end{Shaded}

\pause Dans ce contexte, les fonctions les plus utiles sont :
\texttt{png()}, \texttt{jpeg()} et \texttt{pdf()}. En particulier,
\texttt{pdf()} permet de conserver le caractère vectoriel des graphiques
dans R.

\pause 

\textbf{Remarque} Les graphiques peuvent également facilement être
exportés depuis RStudio en utilisant les menus spécialement conçus à cet
effet.

\end{frame}

\subsection{\texorpdfstring{Réaliser des graphiques avec
\protect\texttt{ggplot2}}{Réaliser des graphiques avec }}\label{realiser-des-graphiques-avec}

\begin{frame}[fragile]{\large L'implémentation d'une grammaire des
graphiques}

Le \emph{package} \texttt{graphics} permet de réaliser une grande
quantité de graphiques mais présente deux limites importantes :

\begin{itemize}
\tightlist
\item
  les fonctions qui le composent forment une casuistique complexe;
\item
  il n'est pas possible d'inventer de nouvelles représentations à partir
  des fonctions existantes.
\end{itemize}

\pause Ce sont ces limites que tente de dépasser le \emph{package}
\texttt{ggplot2} en implémentant une \textbf{grammaire des graphiques}

Comme les éléments du langage, les \textbf{composants élémentaires} d'un
graphique doivent pouvoir être \textbf{réassemblés} pour produire de
\textbf{nouvelles représentations.}

\pause 

\textbf{Pour aller plus loin} \textsc{Wilkinson L.} (2005)
\textit{The Grammar of Graphics}, Springer,
\href{https://github.com/hadley/ggplot2-book}{\underline{ggplot2: elegant graphics for data analysis}}

\end{frame}

\begin{frame}[fragile]{\large Les trois composants essentiels d'un
graphique}

La construction d'un graphique avec \texttt{ggplot2} fait intervenir
trois composants essentiels (d'après Wickham, \emph{ibid.}, 2.3) :

\begin{itemize}
\tightlist
\item
  le \texttt{data.frame} dans lequel sont stockées les données à
  représenter ;
\item
  des correspondances esthétiques (\emph{aesthetic mappings}) entre des
  variables et des propriétés visuelles;
\item
  au moins une couche (\emph{layer}) décrivant comment représenter les
  observations.
\end{itemize}

\pause 

\textbf{Exemple} \emph{Miles per gallon} sur l'autoroute en fonction de
la cylindrée.

\center \small 

\begin{Shaded}
\begin{Highlighting}[]
\KeywordTok{ggplot}\NormalTok{(}\DataTypeTok{data =}\NormalTok{ mpg, }\DataTypeTok{mapping =} \KeywordTok{aes}\NormalTok{(}\DataTypeTok{x =}\NormalTok{ displ, }\DataTypeTok{y =}\NormalTok{ hwy)) }\OperatorTok{+}
\StringTok{  }\KeywordTok{geom_point}\NormalTok{()}
\end{Highlighting}
\end{Shaded}

\end{frame}

\begin{frame}[fragile]{\large Rappel : le même graphique avec base R}

\centering \footnotesize

\begin{Shaded}
\begin{Highlighting}[]
\KeywordTok{plot}\NormalTok{(mpg}\OperatorTok{$}\NormalTok{displ, mpg}\OperatorTok{$}\NormalTok{hwy)}
\end{Highlighting}
\end{Shaded}

\includegraphics[height=6.5cm]{partie2_files/figure-beamer/unnamed-chunk-21-1}

\end{frame}

\begin{frame}[fragile]{\large Les trois composants essentiels d'un
graphique}

\center \small 

\begin{Shaded}
\begin{Highlighting}[]
\KeywordTok{ggplot}\NormalTok{(}\DataTypeTok{data =}\NormalTok{ mpg, }\DataTypeTok{mapping =} \KeywordTok{aes}\NormalTok{(}\DataTypeTok{x =}\NormalTok{ displ, }\DataTypeTok{y =}\NormalTok{ hwy)) }\OperatorTok{+}
\StringTok{  }\KeywordTok{geom_point}\NormalTok{()}
\end{Highlighting}
\end{Shaded}

\includegraphics[width=0.9\linewidth]{partie2_files/figure-beamer/unnamed-chunk-22-1}

\end{frame}

\begin{frame}{Couleur, forme et taille des objets}

Pour faire varier l'aspect visuel des éléments représentés en fonction
de données, il suffit
d'\textbf{associer une variable à l'attribut de couleur, de taille ou de forme}
dans la fonction \texttt{aes()}.

\textcolor{white}{Selon le type des variables utilisées pour les correspondances esthétiques, \textbf{les échelles sont continues ou discrètes}.}

\textcolor{white}{Quand la même variable est utilisée dans plusieurs correspondances esthétiques, \textbf{les échelles qui lui correspondent sont fusionnées}.}

\textcolor{white}{Au-delà des correspondances esthétiques dans la fonction \texttt{aes()}, \textbf{l'aspect visuel peut être ajusté directement dans la fonction \texttt{geom\_*}}.}

\end{frame}

\begin{frame}[fragile]{Couleur, forme et taille des objets}

\footnotesize \center

\begin{Shaded}
\begin{Highlighting}[]
\KeywordTok{ggplot}\NormalTok{(mpg, }\KeywordTok{aes}\NormalTok{(displ, hwy, }\DataTypeTok{colour =}\NormalTok{ cyl, }\DataTypeTok{shape =}\NormalTok{ drv)) }\OperatorTok{+}
\StringTok{  }\KeywordTok{geom_point}\NormalTok{()}
\end{Highlighting}
\end{Shaded}

\includegraphics[width=0.9\linewidth]{partie2_files/figure-beamer/unnamed-chunk-23-1}

\end{frame}

\begin{frame}{Couleur, forme et taille des objets}

Pour faire varier l'aspect visuel des éléments représentés en fonction
de données, il suffit
d'\textbf{associer une variable à l'attribut de couleur, de taille ou de forme}
dans la fonction \texttt{aes()}.

Selon le type des variables utilisées pour les correspondances
esthétiques, \textbf{les échelles sont continues ou discrètes}.

\textcolor{white}{Quand la même variable est utilisée dans plusieurs correspondances esthétiques, \textbf{les échelles qui lui correspondent sont fusionnées}.}

\textcolor{white}{Au-delà des correspondances esthétiques dans la fonction \texttt{aes()}, \textbf{l'aspect visuel peut être ajusté directement dans la fonction \texttt{geom\_*}}.}

\end{frame}

\begin{frame}[fragile]{Couleur, forme et taille des objets}

\footnotesize \center

\begin{Shaded}
\begin{Highlighting}[]
\KeywordTok{ggplot}\NormalTok{(mpg, }\KeywordTok{aes}\NormalTok{(displ, hwy, }\DataTypeTok{colour =} \KeywordTok{as.factor}\NormalTok{(cyl)}
\NormalTok{  , }\DataTypeTok{shape =}\NormalTok{ drv)) }\OperatorTok{+}
\StringTok{  }\KeywordTok{geom_point}\NormalTok{()}
\end{Highlighting}
\end{Shaded}

\includegraphics[width=0.9\linewidth]{partie2_files/figure-beamer/unnamed-chunk-24-1}

\end{frame}

\begin{frame}{Couleur, forme et taille des objets}

Pour faire varier l'aspect visuel des éléments représentés en fonction
de données, il suffit
d'\textbf{associer une variable à l'attribut de couleur, de taille ou de forme}
dans la fonction \texttt{aes()}.

Selon le type des variables utilisées pour les correspondances
esthétiques, \textbf{les échelles sont continues ou discrètes}.

Quand la même variable est utilisée dans plusieurs correspondances
esthétiques,
\textbf{les échelles qui lui correspondent sont fusionnées}.

\textcolor{white}{Au-delà des correspondances esthétiques dans la fonction \texttt{aes()}, \textbf{l'aspect visuel peut être ajusté directement dans la fonction \texttt{geom\_*}}.}

\end{frame}

\begin{frame}[fragile]{Couleur, forme et taille des objets}

\footnotesize \center

\begin{Shaded}
\begin{Highlighting}[]
\KeywordTok{ggplot}\NormalTok{(mpg, }\KeywordTok{aes}\NormalTok{(displ, hwy, }\DataTypeTok{colour =} \KeywordTok{as.factor}\NormalTok{(cyl)}
\NormalTok{  , }\DataTypeTok{shape =} \KeywordTok{as.factor}\NormalTok{(cyl))) }\OperatorTok{+}
\StringTok{  }\KeywordTok{geom_point}\NormalTok{()}
\end{Highlighting}
\end{Shaded}

\includegraphics[width=0.9\linewidth]{partie2_files/figure-beamer/unnamed-chunk-25-1}

\end{frame}

\begin{frame}{Couleur, forme et taille des objets}

Pour faire varier l'aspect visuel des éléments représentés en fonction
de données, il suffit
d'\textbf{associer une variable à l'attribut de couleur, de taille ou de forme}
dans la fonction \texttt{aes()}.

Selon le type des variables utilisées pour les correspondances
esthétiques, \textbf{les échelles sont continues ou discrètes}.

Quand la même variable est utilisée dans plusieurs correspondances
esthétiques,
\textbf{les échelles qui lui correspondent sont fusionnées}.

Au-delà des correspondances esthétiques dans la fonction \texttt{aes()},
\textbf{l'aspect visuel peut être ajusté directement dans la fonction \texttt{geom\_*}}.

\end{frame}

\begin{frame}[fragile]{Couleur, forme et taille des objets}

\footnotesize \center

\begin{Shaded}
\begin{Highlighting}[]
\KeywordTok{ggplot}\NormalTok{(mpg, }\KeywordTok{aes}\NormalTok{(displ, hwy)) }\OperatorTok{+}
\StringTok{  }\KeywordTok{geom_point}\NormalTok{(}\DataTypeTok{colour =} \StringTok{"red"}\NormalTok{, }\DataTypeTok{size =} \DecValTok{8}\NormalTok{, }\DataTypeTok{alpha =} \FloatTok{0.5}\NormalTok{)}
\end{Highlighting}
\end{Shaded}

\includegraphics[width=0.9\linewidth]{partie2_files/figure-beamer/unnamed-chunk-26-1}

\end{frame}

\begin{frame}[fragile]{Combinaison de plusieurs graphiques}

\footnotesize \center

\begin{Shaded}
\begin{Highlighting}[]
\KeywordTok{ggplot}\NormalTok{(mpg, }\KeywordTok{aes}\NormalTok{(displ, hwy)) }\OperatorTok{+}
\StringTok{  }\KeywordTok{geom_point}\NormalTok{() }\OperatorTok{+}\StringTok{ }\KeywordTok{geom_smooth}\NormalTok{()}
\NormalTok{  ## `geom_smooth()` using method = 'loess'}
\end{Highlighting}
\end{Shaded}

\includegraphics[width=0.9\linewidth]{partie2_files/figure-beamer/unnamed-chunk-27-1}

\end{frame}

\begin{frame}[fragile]{Combinaison de plusieurs graphiques}

\footnotesize \center

\begin{Shaded}
\begin{Highlighting}[]
\KeywordTok{ggplot}\NormalTok{(mpg, }\KeywordTok{aes}\NormalTok{(displ, hwy)) }\OperatorTok{+}
\StringTok{  }\KeywordTok{geom_point}\NormalTok{() }\OperatorTok{+}\StringTok{ }\KeywordTok{geom_smooth}\NormalTok{(}\DataTypeTok{method =} \StringTok{"lm"}\NormalTok{, }\DataTypeTok{se =} \OtherTok{FALSE}\NormalTok{)}
\end{Highlighting}
\end{Shaded}

\includegraphics[width=0.9\linewidth]{partie2_files/figure-beamer/unnamed-chunk-28-1}

\end{frame}

\begin{frame}[fragile]{Combinaison de plusieurs graphiques}

\footnotesize \center

\begin{Shaded}
\begin{Highlighting}[]
\KeywordTok{ggplot}\NormalTok{(mpg, }\KeywordTok{aes}\NormalTok{(displ, hwy, }\DataTypeTok{colour =}\NormalTok{ drv)) }\OperatorTok{+}
\StringTok{  }\KeywordTok{geom_point}\NormalTok{() }\OperatorTok{+}\StringTok{ }\KeywordTok{geom_smooth}\NormalTok{(}\DataTypeTok{method =} \StringTok{"lm"}\NormalTok{, }\DataTypeTok{se =} \OtherTok{FALSE}\NormalTok{)}
\end{Highlighting}
\end{Shaded}

\includegraphics[width=0.9\linewidth]{partie2_files/figure-beamer/unnamed-chunk-29-1}

\end{frame}

\begin{frame}[fragile]{\large Le fonctionnement en \og couches \fg{} de
\texttt{ggplot2}}

La construction d'un graphique dans \texttt{ggplot2} repose sur la
superposition de couches (\emph{layer}) \textbf{conçues indépendamment}
mais \textbf{réconciliées en fin d'opération}.

\pause Chaque couche est composée de cinq éléments :

\begin{itemize}
\tightlist
\item
  un \texttt{data.frame} (\texttt{data});
\item
  une ou plusieurs correspondances esthétiques (\texttt{mapping});
\item
  une transformation statistique (\texttt{stat});
\item
  un objet géométrique (\texttt{geom});
\item
  un paramètre d'ajustement de la position (\texttt{position}).
\end{itemize}

C'est la \textbf{fonction \texttt{layer()}} qui articule ces cinq
éléments.

\pause 

\textbf{Les fonctions \texttt{geom\_*} vues précédemment sont des appels
pré-paramétrées de \texttt{layer()}}.

\end{frame}

\begin{frame}[fragile]{\large Le fonctionnement en \og couches \fg{} de
\texttt{ggplot2}}

\emph{Un graphique à une couche}

\footnotesize \centering

\begin{Shaded}
\begin{Highlighting}[]
\KeywordTok{ggplot}\NormalTok{() }\OperatorTok{+}\StringTok{ }\KeywordTok{layer}\NormalTok{(}
  \DataTypeTok{data =}\NormalTok{ mpg, }\DataTypeTok{mapping =} \KeywordTok{aes}\NormalTok{(displ, hwy), }\DataTypeTok{stat =} \StringTok{"identity"}
\NormalTok{  , }\DataTypeTok{geom =} \StringTok{"point"}\NormalTok{, }\DataTypeTok{position =} \StringTok{"identity"}
\NormalTok{)}
\end{Highlighting}
\end{Shaded}

\includegraphics[width=0.8\linewidth]{partie2_files/figure-beamer/unnamed-chunk-30-1}

\end{frame}

\begin{frame}[fragile]{\large Le fonctionnement en \og couches \fg{} de
\texttt{ggplot2}}

\emph{Un graphique à deux couches}

\footnotesize \centering

\begin{Shaded}
\begin{Highlighting}[]
\KeywordTok{ggplot}\NormalTok{() }\OperatorTok{+}\StringTok{ }\KeywordTok{layer}\NormalTok{(}
  \DataTypeTok{data =}\NormalTok{ mpg, }\DataTypeTok{mapping =} \KeywordTok{aes}\NormalTok{(displ, hwy), }\DataTypeTok{stat =} \StringTok{"identity"}
\NormalTok{  , }\DataTypeTok{geom =} \StringTok{"point"}\NormalTok{, }\DataTypeTok{position =} \StringTok{"identity"}
\NormalTok{) }\OperatorTok{+}\StringTok{ }\KeywordTok{layer}\NormalTok{(}
  \DataTypeTok{data =}\NormalTok{ mpg, }\DataTypeTok{mapping =} \KeywordTok{aes}\NormalTok{(displ, hwy), }\DataTypeTok{stat =} \StringTok{"smooth"}
\NormalTok{  , }\DataTypeTok{geom =} \StringTok{"line"}\NormalTok{, }\DataTypeTok{position =} \StringTok{"identity"}
\NormalTok{  , }\DataTypeTok{params =} \KeywordTok{list}\NormalTok{(}\DataTypeTok{method =} \StringTok{"lm"}\NormalTok{, }\DataTypeTok{formula =}\NormalTok{ y }\OperatorTok{~}\StringTok{ }\NormalTok{x)}
\NormalTok{)}
\end{Highlighting}
\end{Shaded}

\vfill

\vfill

\end{frame}

\begin{frame}{\large Le fonctionnement en \og couches \fg{} de
\texttt{ggplot2}}

\emph{Un graphique à deux couches}

\centering

\includegraphics[width=1\linewidth]{partie2_files/figure-beamer/unnamed-chunk-32-1}

\end{frame}

\begin{frame}[fragile]{\large Le fonctionnement en \og couches \fg{} de
\texttt{ggplot2}}

\emph{Mise en facteur dans \texttt{ggplot()} de \texttt{data} et
\texttt{mapping}}

\footnotesize \vspace{-1mm}

\begin{Shaded}
\begin{Highlighting}[]
\KeywordTok{ggplot}\NormalTok{(}\DataTypeTok{data =}\NormalTok{ mpg, }\DataTypeTok{mapping =} \KeywordTok{aes}\NormalTok{(displ, hwy)) }\OperatorTok{+}\StringTok{ }\KeywordTok{layer}\NormalTok{(}
  \DataTypeTok{stat =} \StringTok{"identity"}\NormalTok{, }\DataTypeTok{geom =} \StringTok{"point"}\NormalTok{, }\DataTypeTok{position =} \StringTok{"identity"}
\NormalTok{) }\OperatorTok{+}\StringTok{ }\KeywordTok{layer}\NormalTok{(}
  \DataTypeTok{stat =} \StringTok{"smooth"}\NormalTok{, }\DataTypeTok{geom =} \StringTok{"line"}\NormalTok{, }\DataTypeTok{position =} \StringTok{"identity"}
\NormalTok{  , }\DataTypeTok{params =} \KeywordTok{list}\NormalTok{(}\DataTypeTok{method =} \StringTok{"lm"}\NormalTok{, }\DataTypeTok{formula =}\NormalTok{ y }\OperatorTok{~}\StringTok{ }\NormalTok{x)}
\NormalTok{)}
\end{Highlighting}
\end{Shaded}

\normalsize \vspace{-3mm}

\emph{Remplacement de \texttt{layer()} par des alias pré-paramétrés}

\footnotesize \vspace{-1mm}

\begin{Shaded}
\begin{Highlighting}[]
\KeywordTok{ggplot}\NormalTok{(}\DataTypeTok{data =}\NormalTok{ mpg, }\DataTypeTok{mapping =} \KeywordTok{aes}\NormalTok{(displ, hwy)) }\OperatorTok{+}\StringTok{ }
\StringTok{  }\KeywordTok{geom_point}\NormalTok{() }\OperatorTok{+}\StringTok{ }\KeywordTok{geom_smooth}\NormalTok{(}\DataTypeTok{method =} \StringTok{"lm"}\NormalTok{, }\DataTypeTok{se =} \OtherTok{FALSE}\NormalTok{)}
\end{Highlighting}
\end{Shaded}

\begin{Shaded}
\begin{Highlighting}[]
\KeywordTok{ggplot}\NormalTok{(}\DataTypeTok{data =}\NormalTok{ mpg, }\DataTypeTok{mapping =} \KeywordTok{aes}\NormalTok{(displ, hwy)) }\OperatorTok{+}\StringTok{ }
\StringTok{  }\KeywordTok{geom_point}\NormalTok{() }\OperatorTok{+}\StringTok{ }\KeywordTok{stat_smooth}\NormalTok{(}\DataTypeTok{method =} \StringTok{"lm"}\NormalTok{, }\DataTypeTok{se =} \OtherTok{FALSE}\NormalTok{)}
\end{Highlighting}
\end{Shaded}

\end{frame}

\begin{frame}[fragile]{\large Le fonctionnement en \og couches \fg{} de
\texttt{ggplot2}}

À chaque fonction \texttt{geom\_*()} est assocée un paramètre
\texttt{stat} par défaut, et à chaque fonction \texttt{stat\_*()} un
\texttt{geom} par défaut.

\footnotesize \center

\begin{Shaded}
\begin{Highlighting}[]
\KeywordTok{ggplot}\NormalTok{(}\DataTypeTok{data =}\NormalTok{ mpg, }\DataTypeTok{mapping =} \KeywordTok{aes}\NormalTok{(displ, hwy)) }\OperatorTok{+}\StringTok{ }
\StringTok{  }\KeywordTok{geom_point}\NormalTok{(}\DataTypeTok{colour =} \StringTok{"red"}\NormalTok{, }\KeywordTok{aes}\NormalTok{(}\DataTypeTok{size =}\NormalTok{ cyl)) }\OperatorTok{+}\StringTok{ }
\StringTok{  }\KeywordTok{stat_smooth}\NormalTok{(}\DataTypeTok{geom =} \StringTok{"point"}\NormalTok{, }\DataTypeTok{method =} \StringTok{"lm"}\NormalTok{, }\DataTypeTok{se =} \OtherTok{FALSE}
\NormalTok{    , }\DataTypeTok{colour =} \StringTok{"blue"}\NormalTok{, }\DataTypeTok{shape =} \DecValTok{2}\NormalTok{)}
\end{Highlighting}
\end{Shaded}

\includegraphics[width=0.9\linewidth]{partie2_files/figure-beamer/unnamed-chunk-36-1}

\end{frame}

\begin{frame}[fragile]{\large Le fonctionnement en \og couches \fg{} de
\texttt{ggplot2}}

\footnotesize \center

\begin{Shaded}
\begin{Highlighting}[]
\KeywordTok{ggplot}\NormalTok{(mpg, }\KeywordTok{aes}\NormalTok{(displ, hwy)) }\OperatorTok{+}\StringTok{ }
\StringTok{  }\KeywordTok{geom_point}\NormalTok{(}\KeywordTok{aes}\NormalTok{(}\DataTypeTok{colour =}\NormalTok{ drv)) }\OperatorTok{+}\StringTok{ }
\StringTok{  }\KeywordTok{stat_smooth}\NormalTok{(}\DataTypeTok{method =} \StringTok{"lm"}\NormalTok{, }\DataTypeTok{se =} \OtherTok{FALSE}\NormalTok{)}
\end{Highlighting}
\end{Shaded}

\includegraphics[width=0.9\linewidth]{partie2_files/figure-beamer/unnamed-chunk-37-1}

\end{frame}

\begin{frame}[fragile]{\large Le fonctionnement en \og couches \fg{} de
\texttt{ggplot2}}

\footnotesize \center

\begin{Shaded}
\begin{Highlighting}[]
\KeywordTok{ggplot}\NormalTok{(mpg, }\KeywordTok{aes}\NormalTok{(displ, hwy)) }\OperatorTok{+}\StringTok{ }
\StringTok{  }\KeywordTok{geom_point}\NormalTok{(}\KeywordTok{aes}\NormalTok{(}\DataTypeTok{shape =}\NormalTok{ drv), }\DataTypeTok{colour =} \StringTok{"red"}\NormalTok{) }\OperatorTok{+}\StringTok{ }
\StringTok{  }\KeywordTok{stat_smooth}\NormalTok{(}\KeywordTok{aes}\NormalTok{(}\DataTypeTok{colour =}\NormalTok{ class), }\DataTypeTok{method =} \StringTok{"lm"}\NormalTok{, }\DataTypeTok{se =} \OtherTok{FALSE}\NormalTok{)}
\end{Highlighting}
\end{Shaded}

\includegraphics[width=0.9\linewidth]{partie2_files/figure-beamer/unnamed-chunk-38-1}

\end{frame}

\begin{frame}[fragile]{Histogrammes et densités}

\footnotesize \center

\vspace{-0.3cm}

\begin{Shaded}
\begin{Highlighting}[]
\KeywordTok{ggplot}\NormalTok{(mpg, }\KeywordTok{aes}\NormalTok{(hwy)) }\OperatorTok{+}\StringTok{ }\KeywordTok{geom_histogram}\NormalTok{()}
\end{Highlighting}
\end{Shaded}

\includegraphics[width=0.8\linewidth]{partie2_files/figure-beamer/unnamed-chunk-39-1}

\pause \raggedright \small \vspace{-0.3cm}

\textbf{Remarque} Le positionnement des classes des histogrammes semble
perturbé dans les dernières versions de \texttt{ggplot2} : le paramètre
\texttt{boundary} permet de corriger ce problème (\emph{cf.}
\href{http://stackoverflow.com/questions/37876096/geom-histogram-wrong-bins}{\underline{cette discussion}}).

\end{frame}

\begin{frame}[fragile]{Histogrammes et densités}

\footnotesize \center

\begin{Shaded}
\begin{Highlighting}[]
\KeywordTok{ggplot}\NormalTok{(mpg, }\KeywordTok{aes}\NormalTok{(hwy, }\DataTypeTok{colour =}\NormalTok{ drv, }\DataTypeTok{fill =}\NormalTok{ drv)) }\OperatorTok{+}\StringTok{ }
\StringTok{  }\KeywordTok{geom_histogram}\NormalTok{()}
\end{Highlighting}
\end{Shaded}

\includegraphics[width=0.9\linewidth]{partie2_files/figure-beamer/unnamed-chunk-40-1}

\end{frame}

\begin{frame}[fragile]{Histogrammes et densités}

\footnotesize \center

\begin{Shaded}
\begin{Highlighting}[]
\KeywordTok{ggplot}\NormalTok{(mpg, }\KeywordTok{aes}\NormalTok{(hwy)) }\OperatorTok{+}\StringTok{ }\KeywordTok{geom_density}\NormalTok{(}\DataTypeTok{bw =} \FloatTok{0.5}\NormalTok{)}
\end{Highlighting}
\end{Shaded}

\includegraphics[width=0.9\linewidth]{partie2_files/figure-beamer/unnamed-chunk-41-1}

\end{frame}

\begin{frame}[fragile]{Histogrammes et densités}

\footnotesize \center

\begin{Shaded}
\begin{Highlighting}[]
\KeywordTok{ggplot}\NormalTok{(mpg, }\KeywordTok{aes}\NormalTok{(hwy, }\DataTypeTok{colour =}\NormalTok{ drv, }\DataTypeTok{fill =}\NormalTok{ drv)) }\OperatorTok{+}\StringTok{ }
\StringTok{  }\KeywordTok{geom_density}\NormalTok{(}\DataTypeTok{bw =} \FloatTok{0.5}\NormalTok{, }\DataTypeTok{alpha =} \FloatTok{0.5}\NormalTok{)}
\end{Highlighting}
\end{Shaded}

\includegraphics[width=0.9\linewidth]{partie2_files/figure-beamer/unnamed-chunk-42-1}

\end{frame}

\begin{frame}[fragile]{Séries temporelles}

\footnotesize \center

\begin{Shaded}
\begin{Highlighting}[]
\KeywordTok{ggplot}\NormalTok{(economics, }\KeywordTok{aes}\NormalTok{(date, unemploy }\OperatorTok{/}\StringTok{ }\NormalTok{pop)) }\OperatorTok{+}
\StringTok{  }\KeywordTok{geom_line}\NormalTok{()}
\end{Highlighting}
\end{Shaded}

\includegraphics[width=0.9\linewidth]{partie2_files/figure-beamer/unnamed-chunk-43-1}

\end{frame}

\begin{frame}[fragile]{Diagrammes en bâtons et circulaires}

\footnotesize \center

\begin{Shaded}
\begin{Highlighting}[]
\KeywordTok{ggplot}\NormalTok{(mpg, }\KeywordTok{aes}\NormalTok{(drv, }\DataTypeTok{colour =}\NormalTok{ drv, }\DataTypeTok{fill =}\NormalTok{ drv)) }\OperatorTok{+}\StringTok{ }
\StringTok{  }\KeywordTok{geom_bar}\NormalTok{()}
\end{Highlighting}
\end{Shaded}

\includegraphics[width=0.9\linewidth]{partie2_files/figure-beamer/unnamed-chunk-44-1}

\end{frame}

\begin{frame}[fragile]{Diagrammes en bâtons et circulaires}

\footnotesize \center

\begin{Shaded}
\begin{Highlighting}[]
\KeywordTok{library}\NormalTok{(scales)}
\KeywordTok{ggplot}\NormalTok{(mpg, }\KeywordTok{aes}\NormalTok{(drv, }\DataTypeTok{fill =}\NormalTok{ drv)) }\OperatorTok{+}\StringTok{ }
\StringTok{  }\KeywordTok{geom_bar}\NormalTok{(}\KeywordTok{aes}\NormalTok{(}\DataTypeTok{y =}\NormalTok{ (..count..)}\OperatorTok{/}\KeywordTok{sum}\NormalTok{(..count..))) }\OperatorTok{+}
\StringTok{  }\KeywordTok{scale_y_continuous}\NormalTok{(}\DataTypeTok{labels=}\NormalTok{percent) }\OperatorTok{+}
\StringTok{  }\KeywordTok{scale_fill_brewer}\NormalTok{(}\DataTypeTok{palette=}\StringTok{"Blues"}\NormalTok{)}
\end{Highlighting}
\end{Shaded}

\includegraphics[width=0.7\linewidth]{partie2_files/figure-beamer/unnamed-chunk-45-1}

\end{frame}

\begin{frame}[fragile]{Diagrammes en bâtons et circulaires}

\footnotesize \center

\begin{Shaded}
\begin{Highlighting}[]
\NormalTok{g <-}\StringTok{ }\KeywordTok{ggplot}\NormalTok{(mpg, }\KeywordTok{aes}\NormalTok{(}\DataTypeTok{x =} \StringTok{""}\NormalTok{, }\DataTypeTok{fill =}\NormalTok{ drv, }\DataTypeTok{colour =}\NormalTok{ drv)) }\OperatorTok{+}\StringTok{ }
\StringTok{  }\KeywordTok{geom_bar}\NormalTok{(}\DataTypeTok{width =} \DecValTok{1}\NormalTok{)}
\NormalTok{g}
\end{Highlighting}
\end{Shaded}

\includegraphics[width=0.75\linewidth]{partie2_files/figure-beamer/unnamed-chunk-46-1}

\end{frame}

\begin{frame}[fragile]{Diagrammes en bâtons et circulaires}

\footnotesize \center

\begin{Shaded}
\begin{Highlighting}[]
\NormalTok{g }\OperatorTok{+}\StringTok{ }\KeywordTok{coord_polar}\NormalTok{(}\DataTypeTok{theta =} \StringTok{"y"}\NormalTok{) }\OperatorTok{+}\StringTok{ }\KeywordTok{theme_minimal}\NormalTok{() }\OperatorTok{+}
\StringTok{  }\KeywordTok{scale_fill_grey}\NormalTok{() }\OperatorTok{+}\StringTok{ }\KeywordTok{scale_colour_grey}\NormalTok{()}
\end{Highlighting}
\end{Shaded}

\includegraphics[width=0.5\linewidth]{partie2_files/figure-beamer/unnamed-chunk-47-1}

\pause \raggedright \small

\textbf{Pour aller plus loin} Une page du site
\href{http://www.sthda.com/french/wiki/ggplot2-graphique-en-camembert-guide-de-demarrage-rapide-logiciel-r-et-visualisation-de-donnees}{\underline{sthda.com}}
explique (en français) comment produire un diagramme circulaire complet
avec \texttt{ggplot2}.

\end{frame}

\begin{frame}[fragile]{Diagrammes en bâtons et circulaires}

\footnotesize \center

\begin{Shaded}
\begin{Highlighting}[]
\KeywordTok{ggplot}\NormalTok{(mpg, }\KeywordTok{aes}\NormalTok{(drv, }\DataTypeTok{fill =} \KeywordTok{as.factor}\NormalTok{(year))) }\OperatorTok{+}\StringTok{ }
\StringTok{  }\KeywordTok{geom_bar}\NormalTok{()}
\end{Highlighting}
\end{Shaded}

\includegraphics[width=0.9\linewidth]{partie2_files/figure-beamer/unnamed-chunk-48-1}

\end{frame}

\begin{frame}[fragile]{Diagrammes en bâtons et circulaires}

\footnotesize \center

\begin{Shaded}
\begin{Highlighting}[]
\KeywordTok{ggplot}\NormalTok{(mpg, }\KeywordTok{aes}\NormalTok{(drv, }\DataTypeTok{fill =} \KeywordTok{as.factor}\NormalTok{(year))) }\OperatorTok{+}\StringTok{ }
\StringTok{  }\KeywordTok{geom_bar}\NormalTok{(}\DataTypeTok{position =} \StringTok{"fill"}\NormalTok{)}
\end{Highlighting}
\end{Shaded}

\includegraphics[width=0.9\linewidth]{partie2_files/figure-beamer/unnamed-chunk-49-1}

\end{frame}

\begin{frame}[fragile]{Diagrammes en bâtons et circulaires}

\footnotesize \center

\begin{Shaded}
\begin{Highlighting}[]
\KeywordTok{ggplot}\NormalTok{(mpg, }\KeywordTok{aes}\NormalTok{(}\KeywordTok{as.factor}\NormalTok{(year), }\DataTypeTok{fill =}\NormalTok{ drv)) }\OperatorTok{+}\StringTok{ }
\StringTok{  }\KeywordTok{geom_bar}\NormalTok{(}\DataTypeTok{position =} \StringTok{"dodge"}\NormalTok{) }\OperatorTok{+}\StringTok{ }
\StringTok{  }\KeywordTok{coord_flip}\NormalTok{()}
\end{Highlighting}
\end{Shaded}

\includegraphics[width=0.9\linewidth]{partie2_files/figure-beamer/unnamed-chunk-50-1}

\end{frame}

\begin{frame}[fragile]{Boîtes à moustaches et assimilés}

\footnotesize \center

\begin{Shaded}
\begin{Highlighting}[]
\KeywordTok{ggplot}\NormalTok{(mpg, }\KeywordTok{aes}\NormalTok{(}\DataTypeTok{x =}\NormalTok{ drv, }\DataTypeTok{y =}\NormalTok{ hwy)) }\OperatorTok{+}\StringTok{ }
\StringTok{  }\KeywordTok{geom_boxplot}\NormalTok{(}\DataTypeTok{coef =} \FloatTok{1.5}\NormalTok{)}
\end{Highlighting}
\end{Shaded}

\includegraphics[width=0.9\linewidth]{partie2_files/figure-beamer/unnamed-chunk-51-1}

\end{frame}

\begin{frame}[fragile]{Boîtes à moustaches et assimilés}

\footnotesize \center

\begin{Shaded}
\begin{Highlighting}[]
\KeywordTok{ggplot}\NormalTok{(mpg, }\KeywordTok{aes}\NormalTok{(}\DataTypeTok{x =}\NormalTok{ drv, }\DataTypeTok{y =}\NormalTok{ hwy)) }\OperatorTok{+}\StringTok{ }
\StringTok{  }\KeywordTok{geom_jitter}\NormalTok{()}
\end{Highlighting}
\end{Shaded}

\includegraphics[width=0.9\linewidth]{partie2_files/figure-beamer/unnamed-chunk-52-1}

\end{frame}

\begin{frame}[fragile]{Boîtes à moustaches et assimilés}

\footnotesize \center

\begin{Shaded}
\begin{Highlighting}[]
\KeywordTok{ggplot}\NormalTok{(mpg, }\KeywordTok{aes}\NormalTok{(}\DataTypeTok{x =}\NormalTok{ drv, }\DataTypeTok{y =}\NormalTok{ hwy)) }\OperatorTok{+}\StringTok{ }
\StringTok{  }\KeywordTok{geom_violin}\NormalTok{()}
\end{Highlighting}
\end{Shaded}

\includegraphics[width=0.9\linewidth]{partie2_files/figure-beamer/unnamed-chunk-53-1}

\end{frame}

\begin{frame}[fragile]{Titres et axes}

\footnotesize \center

\begin{Shaded}
\begin{Highlighting}[]
\KeywordTok{ggplot}\NormalTok{(mpg, }\KeywordTok{aes}\NormalTok{(displ, hwy)) }\OperatorTok{+}\StringTok{ }\KeywordTok{geom_point}\NormalTok{() }\OperatorTok{+}\StringTok{ }
\StringTok{  }\KeywordTok{ggtitle}\NormalTok{(}\StringTok{"Mon titre avec un retour }\CharTok{\textbackslash{}n}\StringTok{à la ligne"}\NormalTok{) }\OperatorTok{+}
\StringTok{  }\KeywordTok{xlab}\NormalTok{(}\StringTok{"Cylindrée"}\NormalTok{) }\OperatorTok{+}\StringTok{ }\KeywordTok{ylab}\NormalTok{(}\StringTok{"Miles per gallon"}\NormalTok{) }\OperatorTok{+}
\StringTok{  }\KeywordTok{coord_cartesian}\NormalTok{(}\DataTypeTok{xlim =} \KeywordTok{c}\NormalTok{(}\DecValTok{0}\NormalTok{,}\DecValTok{10}\NormalTok{), }\DataTypeTok{ylim =} \KeywordTok{c}\NormalTok{(}\DecValTok{0}\NormalTok{, }\DecValTok{100}\NormalTok{))}
\end{Highlighting}
\end{Shaded}

\includegraphics[width=0.9\linewidth]{partie2_files/figure-beamer/unnamed-chunk-54-1}

\end{frame}

\begin{frame}[fragile]{Disposition : le \emph{facetting}}

\footnotesize \center

\begin{Shaded}
\begin{Highlighting}[]
\KeywordTok{ggplot}\NormalTok{(mpg, }\KeywordTok{aes}\NormalTok{(displ, hwy)) }\OperatorTok{+}
\StringTok{  }\KeywordTok{geom_point}\NormalTok{() }\OperatorTok{+}\StringTok{ }\KeywordTok{geom_smooth}\NormalTok{(}\DataTypeTok{method =} \StringTok{"lm"}\NormalTok{, }\DataTypeTok{se =} \OtherTok{FALSE}\NormalTok{) }\OperatorTok{+}\StringTok{ }
\StringTok{  }\KeywordTok{facet_wrap}\NormalTok{(}\OperatorTok{~}\NormalTok{manufacturer, }\DataTypeTok{nrow =} \DecValTok{3}\NormalTok{)}
\end{Highlighting}
\end{Shaded}

\includegraphics[width=0.9\linewidth]{partie2_files/figure-beamer/unnamed-chunk-55-1}

\end{frame}

\begin{frame}[fragile]{Disposition : le \emph{facetting}}

\footnotesize \center

\begin{Shaded}
\begin{Highlighting}[]
\KeywordTok{ggplot}\NormalTok{(mpg, }\KeywordTok{aes}\NormalTok{(displ, hwy)) }\OperatorTok{+}
\StringTok{  }\KeywordTok{geom_point}\NormalTok{() }\OperatorTok{+}\StringTok{ }\KeywordTok{geom_smooth}\NormalTok{(}\DataTypeTok{method =} \StringTok{"lm"}\NormalTok{, }\DataTypeTok{se =} \OtherTok{FALSE}\NormalTok{) }\OperatorTok{+}\StringTok{ }
\StringTok{  }\KeywordTok{facet_grid}\NormalTok{(drv}\OperatorTok{~}\NormalTok{class)}
\end{Highlighting}
\end{Shaded}

\includegraphics[width=1\linewidth]{partie2_files/figure-beamer/unnamed-chunk-56-1}

\end{frame}

\begin{frame}[fragile]{Sauvegarde et exportation}

Le résultat de la fonction \texttt{ggplot()} pouvant être stocké dans un
objet R, il est possible de le sauvegarder tel quel avec \texttt{save()}
ou \texttt{saveRDS()} et de le réutiliser par la suite dans R.

\begin{Shaded}
\begin{Highlighting}[]
\NormalTok{g <-}\StringTok{ }\KeywordTok{ggplot}\NormalTok{(mpg, }\KeywordTok{aes}\NormalTok{(displ, hwy)) }\OperatorTok{+}\StringTok{ }\KeywordTok{geom_point}\NormalTok{()}
\KeywordTok{saveRDS}\NormalTok{(g, }\DataTypeTok{file =} \StringTok{"g.rds"}\NormalTok{)}
\end{Highlighting}
\end{Shaded}

\pause La fonction \texttt{ggsave()} simplifie l'export de graphiques en
dehors de R. Par défaut, elle sauvegarde le dernier graphique produit.

\begin{Shaded}
\begin{Highlighting}[]
\NormalTok{g }\OperatorTok{+}\StringTok{ }\KeywordTok{geom_smooth}\NormalTok{(}\DataTypeTok{method =} \StringTok{"lm"}\NormalTok{, }\DataTypeTok{se =} \OtherTok{FALSE}\NormalTok{)}
\KeywordTok{ggsave}\NormalTok{(}\StringTok{"monGraphique.pdf"}\NormalTok{)}
\KeywordTok{ggsave}\NormalTok{(}\StringTok{"monGraphique.png"}\NormalTok{)}
\end{Highlighting}
\end{Shaded}

\end{frame}

\end{document}
