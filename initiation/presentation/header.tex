% Packages à charger
\usepackage[french]{babel}
\usepackage{lmodern}
\usepackage{graphicx}
\usepackage{xcolor}
\usepackage{textcomp} 
\usepackage{amsmath, amsfonts, amssymb, amsthm}
\usepackage{booktabs,multirow}
\usepackage{setspace}
\usepackage{float}
\usepackage{pgfpages}
\usepackage{colortbl}
\usepackage{epstopdf}
\usepackage{framed}
\usepackage{etoolbox}
\usepackage{tikz}
\usepackage{csquotes}


% Définition de couleurs utilisées par la suite
\definecolor{shadecolor}{RGB}{248,248,248}
\definecolor{grayInsee}{HTML}{5a5758}
\definecolor{redInsee}{HTML}{ed1443}

% Personnalisation du thème du beamer
\usetheme{default}
\setbeamertemplate{navigation symbols}{}
\setbeamertemplate{footline}[frame number]
\setbeamercolor{title}{fg=grayInsee}
\setbeamercolor{section in toc}{fg=redInsee}
\setbeamercolor{subsection in toc}{fg=grayInsee}
\setbeamertemplate{frametitle}{%
	\large \textcolor{grayInsee}{\subsecname}
	\\ \vspace{0.1cm} \Large \textcolor{redInsee}{\insertframetitle}
}
\setbeamercolor{local structure}{fg=redInsee}

% Instruction spécifiques à tikz
\usetikzlibrary{shapes,arrows,calc, positioning}
\tikzstyle{input} = [draw, rectangle,rounded corners, text width=2.5cm, fill=green!20, node distance=0.5cm, minimum height=2em, text centered]
\tikzstyle{output} = [draw, ellipse,fill=red!20, node distance=0.5cm, minimum height=2em, text centered]
\tikzstyle{block} = [rectangle, draw, fill=blue!20, 
    text width=1.5cm, text centered, minimum height=2em, node distance = 0.5cm]
\tikzstyle{line} = [draw, -latex', shorten >=2pt, shorten <=2pt]
\tikzset{
  invisible/.style={opacity=0},
  visible on/.style={alt={#1{}{invisible}}},
  alt/.code args={<#1>#2#3}{%
    \alt<#1>{\pgfkeysalso{#2}}{\pgfkeysalso{#3}} % \pgfkeysalso doesn't change the path
  },
}

% Personnalisation des débuts de partie et de sous-partie
\AtBeginSection[]
{\ifnum \thesection>0
  \begin{frame}
  \vfill
  \begin{center}
  \LARGE
  \textcolor{grayInsee}{\insertsectionhead}
  \end{center}
  \vfill
  \end{frame}
\else
\fi
\subsection*{\secname}
}
\AtBeginSubsection[]{}

% Affichage d'un fond gris derrière les exemples de code
\ifcsmacro{Shaded}{
  \renewenvironment{Shaded}{\begin{snugshade}}{\end{snugshade}}
}

% Page de garde
\institute{
	\includegraphics[height = 2.5cm]{Logo_Insee.png} \\ ~ \\ 
	\normalsize Martin \textsc{Chevalier} (Insee)
}
\author{8 et 9 juin 2017}
\date{}

% Commande outil pour les exemples, etc.
\newcommand{\aparte}[2]{
	{\small\textsf{\textcolor{redInsee}{\textbf{#1}} #2}}
}

\newcommand{\strong}[1]{\textbf{\textcolor{redInsee}{#1}}}

\newcommand{\link}[1]{\textcolor{redInsee}{\underline{#1}}}
